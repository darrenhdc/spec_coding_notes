\chapter{Spec Coding 与工程实践}

\small

\section{Spec Coding vs Vibe Coding:范式差异}

\subsection{核心概念对比}

Vibe Coding 依赖直觉与即时反馈,通过对话迭代逐步构建功能。Spec Coding 则强调先明确规格(spec),再制定计划(plan),最后执行实现(implement)。

\textbf{关键差异}:
\begin{itemize}
  \item Vibe Coding:决策与实现混合,每句话都可能改变代码走向
  \item Spec Coding:对话阶段与执行阶段分离,决策在前,执行在后
\end{itemize}

\subsection{Spec Coding 的优势}

\begin{enumerate}
  \item \textbf{减少返工}:spec 提前明确目标,避免后期理解偏差
  \item \textbf{可中断恢复}:决策已文档化,换人/换 agent 可继续
  \item \textbf{可审计性}:每次改动都能追溯到明确的 spec 条目
  \item \textbf{防止 drift}:agent 不能随意偏离 spec 定义
\end{enumerate}

\subsection{适用场景}

Spec Coding 适合:
\begin{itemize}
  \item 已知目标、需要稳妥交付的项目
  \item 会长期维护、多次修改的代码库
  \item 多人协作、需要明确约束的场景
\end{itemize}

Vibe Coding 更适合:
\begin{itemize}
  \item 快速原型、探索性开发
  \item 自己都不知道要什么的时候
  \item 一次性脚本、临时工具
\end{itemize}

\section{Claude Code 中的文档体系}

\subsection{PRD / plan / CLAUDE.md 的本质}

在 Claude Code 中,这三个文件本质上是\textbf{长期 context},而非真正的 system prompt。

\begin{description}
  \item[PRD.md] 产品需求文档,作为参考上下文加载
  \item[plan.md] 实施计划,同样作为参考材料
  \item[CLAUDE.md] 项目级约束规则,优先级最高,接近"会话级 system context"
\end{description}

\subsection{记忆遗忘问题}

Claude Code 的文档存在以下问题:
\begin{enumerate}
  \item \textbf{跨会话遗忘}:关闭后重新打开,需要重新加载
  \item \textbf{上下文溢出}:新对话会挤压旧信息
  \item \textbf{实现 drift}:agent 会以当前代码为事实源,而非文档
\end{enumerate}

\subsection{应对策略}

\begin{enumerate}
  \item \textbf{锚点最小化}:只保留必须不变的硬约束(5-10条)
  \item \textbf{提升到 CLAUDE.md}:绝对不能违反的行为规则
  \item \textbf{Re-anchor Prompt}:关键修改前强制重新阅读文档
  \item \textbf{阶段化 plan}:只保留当前阶段的计划
\end{enumerate}

\section{Droid 的系统级差异}

\subsection{文档在 Droid 中的定位}

\begin{table}[h]
\centering
\begin{tabular}{lll}
\hline
\textbf{文件} & \textbf{Claude Code} & \textbf{Droid} \\
\hline
PRD.md & 参考材料 & 系统真理(Authoritative State) \\
plan.md & 设计说明 & 可执行结构(Execution Graph) \\
CLAUDE.md & 说明文件 & Agent 宪法(Policy Layer) \\
\hline
\end{tabular}
\end{table}

\subsection{核心差异}

Droid 的设计假设是"agent 默认不可信",因此:
\begin{itemize}
  \item spec 拥有系统级否决权
  \item plan 是执行约束,而非建议
  \item agent 行为规则是系统的一部分,不可随意覆盖
\end{itemize}

Claude Code 的三文件体系是"把工程纪律显式写出来给 AI 看",Droid 则是"把纪律写进系统本身"。

\section{Spec 的存储与连续性}

\subsection{Spec 的三种载体层}

\begin{enumerate}
  \item \textbf{Repo-level spec}:Git 仓库中的 Markdown/YAML 文件(最稳定)
  \item \textbf{Project-level spec}:Droid 内部 spec state(系统态)
  \item \textbf{Session/View state}:当前执行上下文(不可追溯)
\end{enumerate}

\subsection{判断 Spec 是否在仓库}

\textbf{三问判定法}:
\begin{enumerate}
  \item Spec 能否被另一个账号完整恢复?
  \item Spec 是否存在可版本化、可 diff 的实体?
  \item Agent 引用 spec 时是"引用文件"还是"模糊转述"?
\end{enumerate}

\subsection{防止 Spec 失效的条件}

把 spec 放进 repo 只是"存储正确",必须同时满足:
\begin{enumerate}
  \item 明确宣告 spec = 唯一真理源
  \item 强制"引用义务":任何实现必须引用 spec 条目
  \item spec 可被 diff/review
  \item 冲突时显式处理,不自动调和
\end{enumerate}

\section{spec-kit 工具}

\subsection{spec-kit 的价值}

spec-kit 不是"让 Claude 更聪明",而是把靠自觉维护的 PRD/plan/CLAUDE.md 流程,变成:
\begin{itemize}
  \item 可执行
  \item 可审计
  \item 可持续
\end{itemize}

\subsection{与 Claude Code 的配合}

\textbf{spec-kit + Claude Code} 的特点:
\begin{itemize}
  \item 强制 artifact 链条(不能跳步)
  \item 明确"哪一步可以问不确定问题"
  \item spec/plan/tasks 都是机器可解析的
\end{itemize}

\textbf{手写文档}的特点:
\begin{itemize}
  \item 完全可控
  \item 高度贴合个人习惯
  \item 容易出现 drift、不一致
\end{itemize}

\subsection{何时使用 spec-kit}

\begin{itemize}
  \item[\textbf{✅ 推荐}] 多 feature 并行、长期项目、需要反复改 spec
  \item[\textbf{⚠️ 可选}] Solo 项目、中等复杂度、已有稳定模板
  \item[\textbf{❌ 不推荐}] 快速原型、一次性脚本、明确要 vibe coding
\end{itemize}

\section{Git 工作流与 Spec}

\subsection{Feature = Git 分支约定}

"Feature = Git 分支"不是 Git 的常识,而是被广泛采用的工程约定。其价值在于:
\begin{itemize}
  \item 最简单的隔离单位
  \item 天然的回滚边界
  \item 与 PR/Review 完美对齐
  \item 与 spec-driven 极度契合
\end{itemize}

\subsection{Pull Request 的本质}

PR 不是 Git 概念,而是协作平台的"变更提议机制"。其核心价值在于:
\begin{itemize}
  \item 提供意图边界:"这次改动想干嘛"
  \item 提供讨论空间:对话不污染 commit 历史
  \item 提供安全闸门:review 后才 merge
\end{itemize}

\subsection{Merge 策略对比}

\begin{table}[h]
\centering
\begin{tabular}{lll}
\hline
\textbf{策略} & \textbf{feature 内 commit} & \textbf{main 上 commit} \\
\hline
普通 merge & 原样保留 & 原样保留 \\
squash merge & 保留在 feat & 不进入 main \\
rebase + FF & 改写后保留 & 线性进入 main \\
\hline
\end{tabular}
\end{table}

\textbf{推荐}:在 spec-driven 工作流中,squash merge 是最常见、最稳妥的选择。它让 main 保持清晰、线性、可读。

\subsection{Squash 的本质}

Squash = 把多个 commit "压扁"为一个 commit。其目的是:
\begin{itemize}
  \item main 只保留"做了什么"
  \item 不保留"中间怎么折腾的"
  \item 历史被整理过,只留下对外可读的结果
\end{itemize}

\textbf{误解澄清}:squash 不是"只能压缩顶部一段 commit",而是"必须是连续、可重写的一段历史"。

\section{spec-kit 实战流程}

\subsection{标准工作流}

\begin{enumerate}
  \item \texttt{specify init . --ai claude}:初始化
  \item \texttt{/speckit.constitution}:写项目宪法
  \item \texttt{/speckit.specify}:写功能 spec
  \item \texttt{/speckit.plan}:生成技术方案
  \item \texttt{/speckit.tasks}:生成任务拆分
  \item \texttt{/speckit.implement}:让 Claude 写代码
\end{enumerate}

\subsection{与 Vibe Coding 的交互差异}

\textbf{旧范式(Vibe Coding)}:对话 = 即执行,决策和实现混在一起

\textbf{新范式(Spec-Driven)}:
\begin{itemize}
  \item 对话阶段:讨论 spec(what/why)和 plan(how),不写代码
  \item 执行阶段:\texttt{/speckit.implement} 按计划执行
  \item 角色:从"即时批准者"变为"规范制定者"
\end{itemize}

\subsection{Commit 与 Merge 时机}

\textbf{关键原则}:
\begin{itemize}
  \item \texttt{/speckit.implement} 不会自动 commit
  \item implement 和 merge 不是原子操作
  \item merge 永远是人类的权力边界
\end{itemize}

\textbf{推荐节奏}:
\begin{enumerate}
  \item spec/plan/tasks 完成后 commit
  \item implement 执行
  \item review(git diff / PR)
  \item 满意后 commit implementation
  \item merge(squash 推荐)
\end{enumerate}

\section{Spec 的三种形态}

\subsection{形态分类}

\begin{enumerate}
  \item \textbf{工作态 spec}(不进 git):
  \begin{itemize}
    \item 本地笔记、对话、临时 markdown
    \item 用于思考、对齐、驱动实现
    \item 不要求持久化,squash 后自然消失
  \end{itemize}

  \item \textbf{协作态 spec}(进 git,有生命周期):
  \begin{itemize}
    \item docs/wip/、specs/、design/ 目录
    \item PR 中 review,merge 时决定是否留下
  \end{itemize}

  \item \textbf{契约态 spec}(必须长期存在):
  \begin{itemize}
    \item docs/adr/、docs/api/、README
    \item 必须进 main,长期维护,是团队记忆
  \end{itemize}
\end{enumerate}

\subsection{判断准则}

"两问法"决定 spec 去向:
\begin{enumerate}
  \item 这个 spec 对"feature 完成后"还有用吗?
  \item 如果我明天请假,别人需要它吗?
\end{enumerate}

\begin{itemize}
  \item 两个都否 → 不 commit / 删除
  \item 任一是 → commit 到 main(或 issue/PR)
\end{itemize}

\section{Claude Skills 焦虑与真相}

\subsection{Skills 的本质}

所谓的"Claude skills"本质是:
\begin{itemize}
  \item 一组固定的 prompt 模板
  \item 一些 if/when/condition
  \item 一些行为约束
  \item 一点"人类写的经验总结"
\end{itemize}

claude.md 的作用是:在每次对话/任务时,把这些规则"注入上下文"。

\subsection{省 Token 的真相}

Skills 省 token 的前提是:你原本就打算反复注入同一堆指令。

但在 spec-coding 里,你写的是明确 spec、明确任务,Claude 不需要靠"skills 猜"。spec-coding 消耗 token 是因为你在"外包思考",而不是因为它低效。

\subsection{Skills 的不可靠性}

\begin{enumerate}
  \item \textbf{没有稳定接口}:模型一升级,行为就可能变化
  \item \textbf{经验叠加}:是"我觉得这样好",不是明确契约
  \item \textbf{不可迁移}:博主的项目背景与你的完全不同
  \item \textbf{营销载体}:制造信息差、暗示稀缺、承诺捷径
\end{enumerate}

\subsection{工程判断}

真正强的 LLM 使用者,往往 skills 很少,spec 很清楚。因为模型最擅长明确任务、明确约束,最不擅长模糊经验、玄学规则。

\textbf{判断准则}:"如果模型换了、上下文变了、团队换人了,这套方法还成立吗?"
\begin{itemize}
  \item 成立 → 工程方法
  \item 不成立 → marketing/hack
\end{itemize}

\section{Rebase 的本质}

\subsection{机制说明}

Git rebase = 把"我这一串 commit"从"旧的 base"重新接到"新的 base"上。

\begin{verbatim}
原始状态:
  main:  A ── B ── C
           \
  feat:           D ── E

rebase 后:
  main:  A ── B ── C
           \
  feat:               D' ── E'
\end{verbatim}

\subsection{与 Stash 的关系}

stash 是"可选的前置操作",不是 rebase 的一部分:
\begin{itemize}
  \item 工作区干净 → 直接 rebase
  \item 有未提交修改 → Git 会拒绝
  \item 所以:git stash → git rebase → git stash pop
\end{itemize}

\subsection{适用时机}

\begin{itemize}
  \item[\textbf{✅}] 只在自己的 feature 分支上
  \item[\textbf{✅}] 提交还没被别人依赖时
  \item[\textbf{❌}] 不要在 main/已共享分支上 rebase
\end{itemize}

\section{.specify/ 目录定位}

\subsection{本质澄清}

.specify/ 的本质是:LLM 的工作目录/私有工具状态/临时输入输出。

它更像 .idea/、.vscode/、.cursor/:
\begin{itemize}
  \item ❌ 不稳定
  \item ❌ 工具相关
  \item ❌ 不保证长期语义
  \item ✅ 可随工具版本变化
\end{itemize}

\subsection{不应放的内容}

形态 2(协作态 spec)和形态 3(契约态 spec),都不应该放在 .specify/ 里,而应该放在项目可见、可 review 的目录(./docs、./design、./adr 等)。

\textbf{原因}:
\begin{enumerate}
  \item .specify/ 是 specify 工具的实现细节,工程 spec 必须工具无关
  \item 新工程师会看 README/docs/adr/,不会默认翻 .specify/
  \item review/diff/blame 语义错误
\end{enumerate}

\subsection{正确位置}

\begin{itemize}
  \item 工具私有状态 → .specify/、.claude/
  \item 临时工作 spec → 不进 git、对话、note
  \item 协作态 spec → docs/wip/、specs/(PR 时决定去留)
  \item 契约态 spec → docs/adr/、docs/api/、必须进 main
\end{itemize}

\section{Vibe、Spec 与 Skills 的关系}

\subsection{三者定位}

\begin{itemize}
  \item \textbf{vibe}:直觉/感觉(非长期、非跨 feature)
  \item \textbf{spec}:一次性决策(非长期、非跨 feature)
  \item \textbf{skills}:经验沉淀(长期、跨 feature)
\end{itemize}

\subsection{相互关系}

skills 是 vibe 的长期结晶;spec 是 vibe 的一次性落地。

\textbf{心法}:vibe 给方向,spec 给确定性,skills 让下一次 vibe 更强。

\textbf{判断句}:"我现在写的这段东西,是为了'这次能做对',还是为了'下次能更快 vibe 对'?"
\begin{itemize}
  \item 这次 → spec
  \item 下次 → skills
\end{itemize}

\section{工程实践总结}

\subsection{Spec Coding 的核心原则}

\begin{enumerate}
  \item 不迷信 skills、不迷信 agent
  \item spec 只服务确定性
  \item git 是最终事实源
  \item 所有不确定性留在分支,所有确定性进 main
\end{enumerate}

\subsection{终极心法}

临时 spec 是工作记忆,main 只保留长期记忆;squash merge 的意义,不仅是压缩 commit,也是压缩思考噪音。

\textbf{安全原则}:"所有不确定性,都留在分支里;所有确定性,才进 main。"
