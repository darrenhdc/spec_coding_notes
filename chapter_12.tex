% chapter_12.tex - Refactored
% Lines 100398 to 109533 of original content
{\small

\subsection*{一、底层库场景下的 AI Agent 工程判断}

在 transformers 这种“接口经常变、体量巨大、语义靠源码而不是注释”的库里,用 AI agent / vibe coding 的“性价比和安全性”反而低于古法手写 + 文档 + grep。

\subsubsection*{1.1 场景特征分析}

模型/底层库代码的特点:
\begin{itemize}
    \item 接口是软约定(不总写清楚)
    \item 行为靠实现细节
    \item 一点点错 = silent bug / 数值错 / 性能灾难
    \item API 多、历史包袱重、同名函数多版本
    \item 文档永远落后于实现
    \item 接口不稳定、breaking change 频繁
    \item PR / release / main 分支行为不一致
\end{itemize}

transformers 库的特殊性:
\begin{itemize}
    \item 代码路径深、side effect 多
    \item inheritance / mixin 爆炸
    \item hook / callback 横飞
\end{itemize}

\subsubsection*{1.2 为什么 AI agent 在此类场景下不可靠}

AI agent 的知识几乎必然是“滞后 + 混合态”的:
\begin{itemize}
    \item 混合多个版本
    \item 不知道哪些已废弃
    \item 很容易“合理胡说”
\end{itemize}

AI agent 不会像人一样:
\begin{itemize}
    \item rg class\_name 跳进源码
    \item 看 call graph
    \item 追 commit diff
\end{itemize}

即使能,也做得慢、不可靠、容易 hallucinate。

\subsubsection*{1.3 “古法开发”的实际含义}

不是不用工具,而是用对工具:
\begin{itemize}
    \item IDE + LSP
    \item grep / ripgrep
    \item 看源码
    \item 看 commit history
    \item 跑最小复现脚本
    \item 手写改动
\end{itemize}

这是“底层工程的 vibe”。

\subsubsection*{1.4 适用域不同}

Spec Agent 适用:
\begin{itemize}
    \item 自动化
    \item 多文件协作
\end{itemize}

古法开发适用:
\begin{itemize}
    \item 单点接口
    \item 高频变更
    \item 模型行为一致性
\end{itemize}

两者面对的是两个完全不同的工程地形。

\subsubsection*{1.5 一句话总结}

在接口频繁变化、行为靠源码定义、错误不显性的模型库里,AI agent 的不确定性会被放大;所以经验工程师选择古法开发,不是保守,而是风控。

\subsection*{二、AI Agent 的 LSP 能力分析}

\subsubsection*{2.1 AI agent 确实可以接 LSP}

现在的 agent 通常能用:
\begin{itemize}
    \item go-to-definition
    \item find references
    \item hover / type info
    \item 基础 symbol 索引
\end{itemize}

但“有 LSP ≠ 好用 ≠ 可信”。差别在于谁在主导、谁在兜底。

\subsubsection*{2.2 人用 LSP vs AI 用 LSP}

人用 LSP 是:
\begin{itemize}
    \item 有明确怀疑点
    \item 带着假设跳转
    \item 看完实现立刻判断对错
\end{itemize}

AI 用 LSP 是:
\begin{itemize}
    \item 被动调用
    \item 局部查询
    \item 很难形成长期、稳定的心智模型
\end{itemize}

\subsubsection*{2.3 LSP 无法覆盖的场景}

transformers 里常见问题:
\begin{itemize}
    \item 动态 dispatch
    \item mixin / multiple inheritance
    \item hook / callback
    \item runtime condition 分支
\end{itemize}

这些很多只能靠:
\begin{itemize}
    \item 顺着源码读
    \item 跑代码验证
    \item 看历史改动背景
\end{itemize}

LSP 只能给“入口”,不能给“结论”。

\subsubsection*{2.4 commit diff 是关键短板}

人类工程师会习惯性地:
\begin{itemize}
    \item 看最近 commit
    \item 看 PR 讨论
    \item 看 breaking change 说明
\end{itemize}

AI agent 即使“能看”,也通常:
\begin{itemize}
    \item 不会默认去看
    \item 不知道哪些 diff 是关键
    \item 不擅长用 diff 修正已有假设
\end{itemize}

\subsubsection*{2.5 错误发现与兜底}

人写代码时:
\begin{itemize}
    \item 心里知道“我可能错了”
    \item 会主动 double check
\end{itemize}

AI agent:
\begin{itemize}
    \item 一旦形成错误假设
    \item 很难自发推翻
    \item 容易把错一路传播下去
\end{itemize}

在 silent bug 场景下尤其危险。

\subsection*{三、DeepWiki:AI 文档工具}

\subsubsection*{3.1 DeepWiki 是什么}

DeepWiki 是一个把 GitHub 仓库自动索引成“可对话文档”的工具。它会解析 repo 的代码、结构、README、关键文件,让你可以直接用自然语言问:
\begin{itemize}
    \item 这个仓库是干嘛的?
    \item 某个模块怎么用?
    \item 某个类/函数在哪里定义、为什么这么设计?
\end{itemize}

\subsubsection*{3.2 核心价值}

DeepWiki 的核心价值是“读库”,不是“写库”。它擅长的事情是:
\begin{itemize}
    \item 帮你快速理解一个大型 repo
    \item 定位代码位置和模块职责
    \item 总结当前版本的结构和约定
\end{itemize}

它不等同于 AI agent 自动改代码,而是更像一个“随时可问的 repo 专家”。

\subsubsection*{3.3 与古法开发的关系}

DeepWiki + 古法开发是“互补关系”。在 transformers 这类库里,更安全的工作流是:
\begin{itemize}
    \item 人类主导决策和改动
    \item IDE/LSP/grep 做精确定位
    \item DeepWiki 用来快速理解整体和上下文
\end{itemize}

而不是:
\begin{itemize}
    \item agent 自动生成改动
    \item 人类事后兜底
\end{itemize}

\subsection*{四、TRAE 国内版与地域访问控制}

\subsubsection*{4.1 403 错误的原因}

TRAE 国内版官网是强地域访问控制。403 说明被判定为“非支持区域”。这和账号无关,就是区域策略。

香港 IP ≠ 国内 IP。TRAE 国内版官网下载入口页面通常是按“大陆 / 非大陆”直接切的。

\subsubsection*{4.2 可行的解决方案}

方式 1(最稳,推荐):国内网络代下载
\begin{itemize}
    \item 找一个在中国大陆的朋友 / 同事或一台国内云服务器
    \item 打开 https://www.trae.com.cn/
    \item 下载对应的 macOS / Windows 安装包
    \item 通过文件传给你
\end{itemize}

方式 2:国内机器远程使用
\begin{itemize}
    \item 国内机器安装 Trae 国内版
    \item 你用远程桌面 / VS Code Remote 操作
\end{itemize}

方式 3:换成“只用豆包 API,不用 Trae”
\begin{itemize}
    \item 直接走豆包 API
    \item 在你当前环境自己封工具/agent
    \item 完全绕开 Trae 的地域问题
\end{itemize}

\subsubsection*{4.3 在国内安装后带到香港的使用}

可以在国内下载并安装 Trae 国内版,但回到香港后,“能不能用”取决于你用到 Trae 的哪一层能力。

分层说明:
\begin{enumerate}
    \item 客户端本身:✅ 可以用。不会启动即崩,不会强制卸载。
    \item 登录与基础功能:⚠️ 大概率可用,但不保证
    \item 豆包模型调用:⚠️ 取决于当时策略(关键)
    \item 更新与新功能:❌ 风险最高
\end{enumerate}

经验结论:在国内下载并完成首次登录,回到香港继续日常使用,但准备一个 fallback(国内机器 / 国内网络)。

\subsection*{五、TRAE 命令行版本问题}

\subsubsection*{5.1 结论}

截至目前(2026-01-06),TRAE 没有官方提供的、独立的命令行(CLI)版本。

\subsubsection*{5.2 原因}

TRAE 的定位是「IDE + 内嵌 AI 能力」:
\begin{itemize}
    \item TRAE 本质是一个图形化 IDE(类似 VS Code 的分支/封装)
    \item AI 能力是强绑定在 IDE 交互里的
    \item 不是一个独立可调用的 CLI 工具
\end{itemize}

TRAE 的卖点是:
\begin{itemize}
    \item 上下文感知(编辑器状态)
    \item 文件树
    \item diff / patch
    \item 人在回路里的交互
\end{itemize}

这些能力强依赖 IDE 运行态。

\subsubsection*{5.3 替代方案}

如果你真正想要的是 CLI 可脚本化、可自动化、能跑在服务器 / CI / SSH 里,那现实选择是:

1)直接用模型 API(豆包 / OpenAI / Claude / DeepSeek,自己封 prompt / workflow)

2)用已有的 agent CLI 框架(OpenAI Codex CLI、aider、continue(CLI 模式)、自建脚本 + LLM)

\subsection*{六、GitHub Copilot 功能详解}

\subsubsection*{6.1 New chat}

最基础的 Copilot Chat,会话级对话窗口。

能力特点:
\begin{itemize}
    \item 基于当前打开的文件 / 选区回答问题
    \item 代码解释、补全、重写
    \item 不自动跨文件执行修改
    \item 不长期记忆上下文(换 chat 就没了)
\end{itemize}

适合:临时问问题、解释某段代码、小范围修改建议

不适合:多文件重构、有状态的长期任务、复杂工程决策

\subsubsection*{6.2 Agents}

Copilot 的“多步任务执行模式”,但仍然是交互式 agent。

能力特点:
\begin{itemize}
    \item 可以规划步骤
    \item 可以连续读写多个文件
    \item 可以执行“我打算做 A → 再做 B”
    \item 每一步通常需要你确认或可回滚
\end{itemize}

本质:IDE 内的“半自动助手”、人在回路(human-in-the-loop)

适合:小到中等规模改动、明确目标的 refactor、自动生成测试/文档

不适合:无人值守执行、大规模一次性不可逆改动、高频自动化任务

\subsubsection*{6.3 Spaces}

一个“上下文容器”,不是 agent 本身。可以理解成给 Copilot / Agent 的一个“长期记忆 + 工作区”。

能装什么:
\begin{itemize}
    \item 当前 repo
    \item 特定子目录
    \item 文档、说明、约束
    \item 自定义背景信息(比如 coding style)
\end{itemize}

作用:让 Agent 在多轮对话中保持一致理解,避免每次都重新解释项目背景

适合:中大型项目、多天/多周的连续工作、需要稳定上下文的 agent 任务

不适合:一次性 quick question

\subsubsection*{6.4 Spark(Preview)}

GitHub 正在试验的“AI 驱动的高层工作流”,目前还是 Preview。

目标:从“想法 / 需求”推导出代码结构、文件骨架、初始实现

能力形态:
\begin{itemize}
    \item 比 Agents 更高一层
    \item 偏“项目生成 / scaffold”
    \item 偏 product / workflow,而不是写单个函数
\end{itemize}

适合:新项目起步、PoC / demo、从零生成结构化代码

不适合:成熟项目的精细修改、需要严格一致性的底层代码、生产级自动化

\subsubsection*{6.5 一句话总结}

New chat:随问随答
Agents:人在回路的多步助手
Spaces:给 AI 一个“长期脑子”
Spark:试验性的 AI 项目生成器

它们都不是命令行工具,而是围绕 IDE 设计的交互式能力。

\subsection*{七、spec-kit 与 Spec-Driven Development}

\subsubsection*{7.1 spec-kit + Claude Code vs 手写 PRD + plan}

手写 PRD + plan + CLAUDE.md 的特点:
\begin{itemize}
    \item 优点:完全可控、没有隐式流程、高度贴合工程习惯、Claude Code 行为“可预期”
    \item 问题:强依赖个人 discipline,容易出现 PRD 和 plan 漂移、plan 和实现不一致、忘了补 checklist;多 feature / 多人时一致性迅速下降
\end{itemize}

spec-kit + Claude Code 的核心差异:
\begin{itemize}
    \item 把你现在靠自觉在做的事情,变成“强制、可追踪、可复用的流程”
    \item 强制 artifact 链条(不能跳步):constitution → spec → clarify → plan → tasks → implement
    \item 明确“哪一步可以问不确定问题”
    \item spec / plan / tasks 都是“机器可解析”的
\end{itemize}

\subsubsection*{7.2 spec-kit 作为“外挂程序”的可靠性}

靠谱的地方:
\begin{itemize}
    \item 它不劫持模型:spec-kit 不替代 Claude Code、不包一层“魔法 agent”、只是生成目录结构/Markdown artifacts/shell 脚本
    \item 它是“可删的”:可以保留 specs / plan / tasks.md,删除 CLI 和脚本,Claude Code 依然能继续用
\end{itemize}

CLAUDE.md:给人 + 给模型看的指导

spec-kit 的 artifacts:给 agent 当“状态机”的

这让 Claude Code 不会反复“忘记自己做过什么”、不容易重构时推翻前面的决定。

\subsubsection*{7.3 对比总结}

你现在的做法是“高质量手工 Spec-Driven”,spec-kit 是把这套方法变成“流程化 + 工具化 + 可复制”。

} % end small
