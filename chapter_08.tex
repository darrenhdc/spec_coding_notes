% chapter_08.tex
% Lines 63853 to 72988 of original content

{\small

\section*{Chapter 8: LSP, Spec Coding, and System Design}


% =================================================================

\subsection*{PRD.md, plan.md, CLAUDE.md: Three-Layer Control System}

\subsubsection*{One-Sentence Overview}

\begin{itemize}
  \item PRD.md = 为什么要做(What \& Why)
  \item plan.md = 打算怎么做(How, at 设计层)
  \item CLAUDE.md = Claude 在这个 repo 里"该如何当一个工程师"
\end{itemize}

三者不是同一层级的文档,而是三层控制面。

\subsubsection*{PRD.md — 产品事实层(Intent / Truth)}

PRD 是"需求的真理来源",不是给 Claude 发挥的,而是用来约束 Claude 的。

\textbf{解决的问题:}
\begin{itemize}
  \item 要做什么功能
  \item 用户是谁
  \item 行为边界在哪里
  \item \textbf{不做什么(非常重要)}
\end{itemize}

\textbf{特性:}稳定(不频繁改)、非技术为主、人 \& AI 都要遵守

\textbf{典型内容:}
\begin{verbatim}
# Product Requirements

## Goals
- 用户可以创建 / 编辑 / 删除任务

## Non-Goals
- 不支持多租户
- 不支持权限系统

## User Stories
- As a user, I can ...
\end{verbatim}

\textbf{不该有的东西:}技术选型、文件结构、类名/函数名

\subsubsection*{plan.md — 技术决策层(Design / Architecture)}

plan.md 是"在不违反 PRD 的前提下,我们决定如何实现"。这是工程设计文档。

\textbf{解决的问题:}
\begin{itemize}
  \item 用什么技术栈
  \item 模块怎么拆
  \item 数据流 / 控制流
  \item 关键 trade-off
\end{itemize}

\textbf{特性:}比 PRD 更易变、强技术导向、给 Claude 一个"不要乱设计"的边界

\textbf{典型内容:}
\begin{verbatim}
# Implementation Plan

## Architecture
- Frontend: React
- Backend: FastAPI

## Key Decisions
- 使用 REST 而不是 GraphQL(原因:简单)

## Constraints
- 不引入 ORM
- 保持 API 向后兼容
\end{verbatim}

\textbf{不该有的东西:}具体代码实现、逐行逻辑、prompt 指令

\subsubsection*{CLAUDE.md — Agent 行为控制层(Meta / Governance)}

这是 Claude Code 特有、也是最容易被低估的文件。

CLAUDE.md 不是"项目文档",而是"Claude 的操作手册"。

\textbf{解决的问题:}
\begin{itemize}
  \item Claude 在这个 repo 里应该怎么工作
  \item 什么时候可以改代码
  \item 什么时候必须先问
  \item 使用什么风格 / 规则
\end{itemize}

这是对 AI 的"constitution"。

\textbf{典型内容:}
\begin{verbatim}
# Claude Instructions

## General Rules
- 不得引入新依赖,除非明确说明
- 修改前必须先分析影响范围

## Coding Style
- 使用函数式风格
- 所有 public API 必须有注释

## Workflow
- 修改前先总结计划
- 不允许直接大规模重构
\end{verbatim}

\textbf{不适合放的内容:}产品需求(那是 PRD)、技术架构(那是 plan)、临时聊天指令(那是 prompt)

\subsubsection*{Control Direction Comparison}

\begin{tabular}{l|l|l|l}
文件 & 控制谁 & 控制什么 & 稳定性 \\ \hline
PRD.md & 人 + AI & 产品边界 & ✅ 高 \\
plan.md & AI 为主 & 技术实现方式 & ⚠️ 中 \\
CLAUDE.md & Claude & 行为 / 决策方式 & ✅ 高 \\
\end{tabular}

可以理解为:PRD = 法律,plan = 施工图,CLAUDE.md = 施工队操作规范。

\subsubsection*{How They Work Together in Claude Code}

理想的 Claude Code 开发流程:
\begin{enumerate}
  \item Claude 读取 CLAUDE.md → 知道「我该怎么当一个工程师」
  \item Claude 参考 PRD.md → 知道「我不能越界」
  \item Claude 遵循 plan.md → 知道「我该怎么实现」
  \item Claude 才会:分析、提出修改、实现代码
\end{enumerate}

\subsubsection*{Common Errors}

\textbf{错误 1:}把指令写进 PRD —— "你应该用 React 实现这个功能" → 这是 plan,不是 PRD。

\textbf{错误 2:}把 prompt 临时规则塞进 plan —— "你修改前要先确认类型" → 这是 CLAUDE.md。

\textbf{错误 3:}没有 CLAUDE.md,却抱怨 Claude "乱来" → 没有行为约束,AI 必然发挥。

\subsubsection*{Spec-kit Mapping}

如果用 spec-kit:
\begin{itemize}
  \item spec-kit 文件 → 对应概念
  \item spec.md → PRD(功能级)
  \item plan.md → plan
  \item constitution.md ≈ CLAUDE.md
\end{itemize}

\textbf{一句话总结:}PRD.md 决定"能不能做",plan.md 决定"怎么做",CLAUDE.md 决定"Claude 做事像不像一个靠谱工程师"。

% =================================================================

\subsection*{Implicit vs Explicit: What Happens Without PRD/plan/CLAUDE.md?}

\subsubsection*{Core Conclusion}

会"看起来像在执行",但不是在"可靠地执行"。

换句话说:
\begin{itemize}
  \item ✅ Claude Code 在没有 PRD / plan / CLAUDE.md 的情况下,会隐式模拟一个类似流程
  \item ❌ 但这个流程是:不稳定的、不可审计的、不可继承的、不可复现的
\end{itemize}

\subsubsection*{What Claude Code Actually Does Without Documents}

Claude Code 并不是"无流程",而是:把 PRD / plan / 行为规范,全都临时塞进了对话上下文里。

可以理解为:
\begin{verbatim}
prompt + 当前 repo
= 即时 PRD
= 即时 plan
= 即时 CLAUDE.md
\end{verbatim}

这些东西存在,但:只存在于当前对话、不可见、不可复用、不可校验。

\subsubsection*{Why It Feels Like It Works}

\textbf{1️⃣ 模型本身内建了「优秀工程师的隐式模式」}

Claude 非常擅长:先解释理解、再提出方案、最后写代码。这是模型训练出来的"默认行为"。

\textbf{2️⃣ Claude Code 自动做 repo 分析(含 LSP / AST)}

即使你没说,它也会:扫描项目结构、猜测架构、找入口点。这让它看起来像是有 plan。

\textbf{3️⃣ 人类 prompt 本身往往是"半个 PRD"}

你通常会说:"我要实现 X,要求 Y,不要 Z"。这已经是一个简化版 PRD 了。

\subsubsection*{The Distortion Points}

\textbf{❌ 1️⃣ PRD 是"隐式的",所以会漂移}

你第一轮说的"不做 Z",下一轮对话没再提,Claude Code 很可能:忘了、或重新假设。这不是 bug,是上下文的自然衰减。

\textbf{❌ 2️⃣ plan 是"即时生成的",不是长期约束}

Claude Code 的 plan是:"我这一次觉得怎么实现最好",而不是"我们项目约定的实现方式"。所以下一次它可能:用另一种架构、引入你不想要的依赖。

\textbf{❌ 3️⃣ Claude 的"行为规范"每次都会重算}

没有 CLAUDE.md:Claude 每次都会重新决定:我是激进还是保守?要不要重构?要不要加新抽象?

👉 这就是你有时觉得它"太聪明反被聪明误"的原因。

\subsubsection*{When Implicit Process Breaks Down}

这种"隐式流程"在以下条件下确实够用:项目小、文件少、需求稳定、主要是你一个人在用、你随时盯着它。

在这种条件下:Claude Code ≈ 高级 pair programmer。

\textbf{什么时候它会开始"明显失控"?}
\begin{itemize}
  \item 项目超过 ~20–30 个文件
  \item 开始重构 / 返工
  \item 多次跨会话继续开发
  \item 你想让它"记住之前的决定"
\end{itemize}

这时你会发现:它忘了之前说过的话、行为风格不一致、开始引入"没约定过的东西"。

\subsubsection*{The Real Value of PRD/plan/CLAUDE.md}

不是让 Claude"更聪明",而是让 Claude"更稳定"。它们的作用是:把"隐式、一次性的智能"变成"显式、可复用的约束"。

\subsubsection*{Accurate Analogy}

不写 PRD / plan / CLAUDE.md 的 Claude Code ≈ 一个经验丰富、但没有设计文档的高级工程师。

他能:快速写代码、解决问题。

但:难以交接、难以回溯、难以规模化。

\textbf{一句话总结:}Claude Code 在没有 PRD / plan / CLAUDE.md 时,会"临时模拟"一个开发流程;但一旦你需要稳定性、可继承性和长期一致性,这些文件就从"可选"变成了"必需"。

% =================================================================

\subsection*{Hallucination vs Drift: Two Types of AI Errors}

\subsubsection*{Definitions}

\textbf{Hallucination =} 模型在"不知道或不确定"的情况下,仍然自信地生成看起来合理、但实际上是错误或不存在的信息。中文常译为:幻觉 / 虚构 / 编造事实。

在你这个上下文里:
\begin{itemize}
  \item hallucination:编造事实
  \item drift:偏离约束
\end{itemize}

两者经常一起出现,但不是一回事。

\subsubsection*{Typical Manifestations}

\textbf{代码场景:}
\begin{itemize}
  \item 编造不存在的函数 / API
  \item 猜错类型签名
  \item 说"这个函数在 X 文件里",但实际上不存在
  \item 生成看起来很对、但项目里根本没有的模块
\end{itemize}

\textbf{文档 / spec 场景:}
\begin{itemize}
  \item 声称"PRD 要求了某行为",但 PRD 里根本没写
  \item 引用不存在的设计决策
\end{itemize}

\subsubsection*{Relationship Between Hallucination and Drift}

\begin{tabular}{l|l|l}
维度 & Hallucination & Drift \\ \hline
本质 & 编造不存在的事实 & 偏离原有约束 \\
时间维度 & 往往是单次 & 通常是累积的 \\
是否一定错误 & ✅ 是 & ❌ 不一定马上错 \\
是否和记忆有关 & ❌ 不一定 & ✅ 强相关 \\
是否可被 LSP 抑制 & ✅ 很大程度 & ❌ 只能部分 \\
\end{tabular}

\textbf{一个非常典型的组合场景:}
\begin{enumerate}
  \item 没有明确 spec / LSP 约束
  \item 模型猜了一个 API(hallucination)
  \item 后续修改都基于这个错误假设
  \item 整个实现逐步偏离真实设计(drift)
\end{enumerate}

👉 hallucination 往往是 drift 的起点。

\subsubsection*{How LSP/Spec-Coding Reduces Hallucination}

\textbf{✅ LSP 对 hallucination 的抑制作用:}

LSP 提供:符号表、类型系统、引用关系。这些是:本地的、确定的、不靠模型"猜"的。

👉 你在用 LSP-first prompt,其实是在反 hallucination。

\textbf{✅ spec 对 hallucination 的抑制作用:}

spec 明确:做什么、不做什么。减少:自由发挥空间、编造需求的可能。

\subsubsection*{Standard English Expressions}

\textbf{名词:}hallucination, model hallucination, LLM hallucination

\textbf{动词:}hallucinate —— "The model hallucinated an API that doesn't exist."

\textbf{工程文档常见说法:}
\begin{itemize}
  \item "This response contains hallucinated APIs."
  \item "To reduce hallucination, rely on LSP-based analysis."
  \item "The agent should explicitly say 'unknown' instead of hallucinating."
\end{itemize}

\subsubsection*{Key Sentence}

"LSP-first workflows significantly reduce hallucination, while explicit specs are necessary to prevent drift."

\textbf{一句话总结:}Hallucination 是"凭空编造",Drift 是"逐步跑偏";前者靠 LSP 和事实源解决,后者靠 spec 和结构化流程解决。

% =================================================================

\subsection*{Droid's Internalization: PRD/plan/CLAUDE.md as System Components}

\subsubsection*{Core Conclusion}

在 Droid 中:PRD / plan / CLAUDE.md 不再是三个并列文件,而是被提升为「系统级状态」与「agent 约束层」。

也可以说:
\begin{itemize}
  \item Claude Code:你显式写文档,Claude 读
  \item Droid:系统原生维护这些"角色",你只是在编辑其中一部分
\end{itemize}

\subsubsection*{Mapping Table}

\begin{tabular}{l|l|l}
你熟悉的文件 & 在 Droid 中对应的是 & 本质变化 \\ \hline
PRD.md & Product / Feature Spec(系统真理层) & 从"文档"→"主事实源" \\
plan.md & Implementation Plan Graph(执行图) & 从"说明"→"可执行结构" \\
CLAUDE.md & Agent Policy / Constitution(行为宪法) & 从"说明文件"→"硬约束" \\
\end{tabular}

\subsubsection*{PRD.md in Droid: Spec as System Truth}

在 Droid 里,PRD 不是一个普通 Markdown 文件,而是:被解析、被结构化、被持续校验。

可以理解为:PRD → Feature Spec → 系统的"事实模型"。

\textbf{在 Droid 中,PRD 的特性:}
\begin{itemize}
  \item ✅ 是唯一真理来源
  \item ✅ 每一行 spec 都是:可引用、可追踪、可验证
  \item ✅ 实现必须能:指回 spec,否则会被认为是"越界行为"
\end{itemize}

📌 这一步是 spec-coding 的核心差异。

\textbf{和 Claude Code 的关键不同:}

\begin{tabular}{l|l}
Claude Code & Droid \\ \hline
PRD 是"参考材料" & PRD 是"约束条件" \\
可以被忽略 & 不能被违反 \\
只在 prompt 里生效 & 在系统中长期生效 \\
\end{tabular}

\subsubsection*{plan.md in Droid: Executable Implementation Graph}

在 Claude Code + spec-kit 里:plan.md 是设计说明,给 Claude "参考"。

在 Droid 里:plan 是 agent 要"执行"的对象,而不是"阅读"的对象。

Droid 的 plan 本质上是一个依赖图 / 任务图,包含:顺序、并行、前置条件、回滚点。

可以理解为:
\begin{verbatim}
plan.md(人类可读)
↓
Droid 内部:
Implementation DAG / State Machine
\end{verbatim}

\textbf{关键差异一句话:}Claude Code 的 plan 是"建议",Droid 的 plan 是"执行约束"。

\subsubsection*{CLAUDE.md in Droid: Agent Constitution as Policy}

这是最本质、也最容易忽略的一点。

在 Claude Code:CLAUDE.md = "Claude,请你在这个项目里这么做"。

在 Droid:Agent 的行为规则不是文件,而是系统的一部分。

\textbf{Droid 的 Agent Policy 包含:}
\begin{itemize}
  \item 什么时候可以改代码
  \item 什么时候必须先分析
  \item 什么时候必须拒绝
  \item 如何处理不确定性
  \item 如何防 hallucination / drift
\end{itemize}

这些规则:❌ 你不能随意覆盖、❌ 不是 prompt、✅ 是 agent 的"操作系统"。

📌 这也是为什么 Droid 看起来"更保守、更慢",但更稳。

\subsubsection*{Intuitive Analogy}

\textbf{Claude Code 世界:}
\begin{verbatim}
你(人)
 ├─ PRD.md        ← 给 Claude 看
 ├─ plan.md       ← 给 Claude 看
 └─ CLAUDE.md     ← 教 Claude 怎么做事
\end{verbatim}

\textbf{Droid 世界:}
\begin{verbatim}
Droid 系统
 ├─ Spec Engine        ← PRD 内生
 ├─ Planning Engine    ← plan 内生
 └─ Policy Engine      ← CLAUDE.md 内生

你只是在编辑"输入",不是在教 agent 怎么工作。
\end{verbatim}

\subsubsection*{Paradigm Shift}

你前面所有问题,其实都指向这一点:
\begin{itemize}
  \item Claude Code + spec-kit 是"把人类工程流程教给 AI"
  \item Droid 是"把工程流程变成 AI 系统本身"
\end{itemize}

\textbf{一句话总结:}在 Droid 中,PRD 变成了系统真理,plan 变成了可执行结构,CLAUDE.md 变成了 agent 的宪法;你不再"教 AI 怎么做事",而是在"向一个有工程纪律的系统输入事实"。

% =================================================================

\subsection*{Context vs System Prompt: Claude Code vs Droid Memory}

\subsubsection*{Claude Code's Mechanism}

在 Claude Code 中:PRD.md / plan.md / CLAUDE.md 本质上都是「长期 context」,不是每次提问的 system prompt。

\textbf{1️⃣ CLAUDE.md — 最接近 system prompt,但仍不是"每次重置"}

Claude Code 会:在会话开始时读取 CLAUDE.md、把它注入到一个高优先级上下文。

但它不是:OpenAI API 那种「每次请求都重新塞 system prompt」。

更准确的说法是:CLAUDE.md 是"会话级 system context",不是"请求级 system prompt"。

📌 含义:
\begin{itemize}
  \item 同一个 Claude Code 会话里:✅ 一直有效
  \item 跨会话:❌ 不保证 100\% 一致(但通常会重新加载)
\end{itemize}

\textbf{2️⃣ PRD.md / plan.md — 明确是 context memory}

Claude Code 会:在 repo 中扫描这些文件、作为"参考上下文"加载。

但它们:❌ 优先级低于 CLAUDE.md、❌ 低于你当前 prompt、❌ 会受到 context 窗口限制。

所以它们是:显式、可读、但"软约束"的 context。

\textbf{3️⃣ 为什么你感觉「它好像一直记得」?}

因为 Claude Code 做了三件事:会话不频繁重置、对 repo 结构有缓存、模型本身有很强的工程先验。

👉 这让它"看起来像系统级记忆",但本质仍然是 context + prompt + 模型习惯。

\textbf{用一句工程化的话总结 Claude Code:}Claude Code = 长 context + 高质量模型 + 软文档约束。

\subsubsection*{Droid's Mechanism}

Droid 不"要求"你写 PRD.md / plan.md / CLAUDE.md 这三个文件。

但注意:不是"不需要这些概念",而是"不需要你用文件去表达"。

\textbf{1️⃣ PRD.md → 系统内生的 Spec / Feature Model}

你输入需求,系统会:结构化、拆解、存成 spec state。这个 spec:✅ 长期存在、✅ 可追踪、✅ 不随会话消失。

👉 PRD 不再是"文档",而是"数据库里的事实"。

\textbf{2️⃣ plan.md → 可执行的 Planning Graph}

你不写:"plan.md 请你照着做"。而是:Droid 自动生成、并作为执行约束。你甚至不能随意跳过 plan。

\textbf{3️⃣ CLAUDE.md → Agent 的硬编码 Policy}

这是最大差异点。在 Droid 中,agent 的行为规则:不是 prompt、不是文档、不是可随意覆盖的,而是:系统级 policy,类似"操作系统内核规则"。

👉 你不"教 Droid 怎么做事",你只是向一个有纪律的系统提供输入。

\subsubsection*{Clear Comparison}

\textbf{Claude Code 世界:}
\begin{verbatim}
约束 = 文档 + prompt + 你的盯梢
记忆 = context
稳定性 = 人工维护
\end{verbatim}

\textbf{Droid 世界:}
\begin{verbatim}
约束 = 系统内生
记忆 = 状态存储
稳定性 = 系统保证
\end{verbatim}

\textbf{终极判断:}Claude Code 的三文件体系,是"把工程纪律显式写出来给 AI 看";Droid 的设计,是"假设 AI 不可信,于是把纪律写进系统本身"。

\textbf{一句话总结:}在 Claude Code 中,PRD / plan / CLAUDE.md 是长期 context,而非真正的 system prompt;在 Droid 中,这三者不再是文档,而是 agent 系统的内生组成部分。

} % end small
