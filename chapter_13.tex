% chapter_13.tex
\small

\section*{Spec-Kit 与 Claude Skills 深度解析}

\subsection*{一、spec-kit vs PRD+plan+CLAUDE.md 的本质差异}

你现在说的「PRD + plan + CLAUDE.md」其实是成熟团队的"手工 Spec-Driven"。先明确一点:你描述的这套做法,本质上已经是 \textbf{Spec-Driven Development(SDD)}了。区别不在\textbf{理念},而在\textbf{机制}。

\subsubsection*{1. 手写 PRD + plan + CLAUDE.md 的特点}

\textbf{优点:}完全可控、没有隐式流程、高度贴合你自己的工程习惯、Claude Code 行为"可预期"。

\textbf{问题:}强依赖个人 discipline,很容易出现 PRD 和 plan 漂移、plan 和实现不一致、忘了补 checklist。多 feature / 多人时一致性迅速下降。这套方式在「一个人或小团队」里很好,在规模化时很累。

\subsubsection*{2. spec-kit + Claude Code 的核心差异}

spec-kit 干的不是"帮你想需求",而是把你现在靠自觉在做的事情,变成"强制、可追踪、可复用的流程"。

\textbf{强制 artifact 链条(不能跳步):}
\begin{verbatim}
constitution → spec → clarify → plan → tasks → implement
\end{verbatim}

Claude Code \textbf{不能直接从"想法 → 写代码"},每一步都必须有对应文件。

\textbf{明确"哪一步可以问不确定问题":}例如 /speckit.clarify 明确告诉 agent:现在是补需求,不是写方案。而你手写流程里,这个阶段经常被跳过或混杂。

\textbf{spec / plan / tasks 都是"机器可解析"的:}CLAUDE.md 是给人 + 给模型看的指导;spec-kit 的 artifacts 是给 agent 当"状态机"的。这让 Claude Code 不会反复"忘记自己做过什么",不容易重构时推翻前面的决定。

\subsubsection*{3. spec-kit 作为"外挂程序"的可靠性}

\textbf{靠谱的地方:}spec-kit 不替代 Claude Code,不包一层"魔法 agent",它只是生成目录结构、Markdown artifacts、shell / ps 脚本。你随时可以不用它,项目不报废。如果哪天你觉得太重不值,可以保留 specs / plan / tasks.md,删除 CLI 和脚本,Claude Code 依然能继续用。它已经被 GitHub 内部 / Copilot 团队在用,这是给 Copilot / Claude Code / Codex CLI 真实用的,不是 demo 玩具。

\textbf{不适合的地方:}流程很"啰嗦",如果你是快速试错、hackathon、一次性脚本,spec-kit \textbf{明显过重}。它不会让 Claude 更聪明,如果你现在的问题是"Claude 老是写烂代码",spec-kit \textbf{救不了你},它只解决忘步骤、漂移、自相矛盾的问题。对已经非常成熟的个人流程,如果你 PRD 写得很清楚、plan 非常工程化、CLAUDE.md 用得很好,spec-kit 对你是 \textbf{20\% 提升,不是 2× 提升}。

\subsubsection*{4. 什么时候推荐使用 spec-kit}

\textbf{强烈推荐:}多 feature 并行、长期项目、会反复改 spec、需要 Claude Code 长时间保持一致认知、想把"好习惯"变成"默认行为"。

\textbf{可选:}solo 项目、中等复杂、已经有稳定 PRD + plan 模板,可以只学 spec-kit 的结构,不一定用 CLI。

\textbf{不推荐:}快速原型、一次性脚本、明确知道"我就是要 vibe coding"。

\textbf{总结:}spec-kit 不是"Claude 的外挂大脑",而是把你现在靠自觉维护的 PRD / plan / CLAUDE.md 流程,变成一套可执行、可审计、可持续的 Spec-Driven 工程流水线;如果你已经是成熟工程实践者,它不会颠覆你,但会帮你少犯错、少返工。

\subsection*{二、Claude Skills 的启动方式}

Claude 项目的 skill 并不是"写在 Claude 里",而是写在「Claude 所运行的宿主环境」中。所以答案不是一个文件名,而是一个\textbf{分层答案}。

\subsubsection*{1. Claude 启动 Skill 的 3 种常见方式}

\textbf{方式一:Slash Command(你最可能用到的)}

这是 \textbf{spec-kit / Claude Code / Copilot} 里最常见的形式。你看到的是 /speckit.specify、/speckit.plan、/speckit.tasks 等。实际发生的是:环境(CLI / IDE)注册了一个 skill,告诉 Claude 当用户输入 /speckit.specify,你必须使用指定模板、写入指定文件、遵守指定规则。Claude 不是"自由回答",而是进入这个 skill 的执行模式,按 skill 的 contract 输出。Slash command = 显式启动 skill,Claude \textbf{不会自己随便用},必须你或系统明确触发。

\textbf{方式二:Tool / Function Call(你看不到,但在发生)}

这是 \textbf{Claude API / Copilot / Droid 内部}最常见的。Claude 在内部做的是类似:
\begin{verbatim}
{
  "tool_call": {
    "name": "analyze_repo_structure",
    "arguments": { "depth": 3 }
  }
}
\end{verbatim}

系统告诉 Claude 你现在"可以用"这些 tools(skills),Claude 在推理过程中判断这个问题我不能靠文本回答,我需要调用一个 tool,然后自动触发。这类 Skill 的特点:自动、不需要你输入 /something、常见于 Droid 自动 agent、无人值守流程。你对它的控制力最弱,但自动化最高。

\textbf{方式三:约定式 Skill("像 Skill,但不是显式 Skill")}

这是你现在可能已经在用、但没意识到的。例如 CLAUDE.md:你在 CLAUDE.md 里写 "When reviewing code: Always check for concurrency issues; Always suggest tests; Never modify public APIs without confirmation"。这本质上是一个"软 Skill",它没有明确名字、明确调用指令,但它定义了何时触发、如何执行、什么不能做。Claude 在每轮对话中都会"启动"它。

\subsubsection*{2. Claude 什么时候决定启动 Skill}

\textbf{用户显式要求:}例如 /speckit.plan,强制启动。

\textbf{系统提示强制要求:}例如"当需要修改文件时,必须使用 file\_edit skill",Claude \textbf{不能直接输出文本}。

\textbf{推理中发现"必须用工具":}例如要读 repo、要跑测试、要查文件,Claude 自动 tool call(如果允许)。

\subsubsection*{3. 为什么你会感觉"Claude 有时候不用 Skill"}

因为 Skill 不是默认能力。Claude 只有在被允许、被提示、被强制时才会用,否则它会退回到"自由生成模式"(最危险、也最灵活)。

\textbf{判断口诀:}你能"点名叫它干活"的,才叫 Skill;它"自己想干就干"的,是 Agent 内部能力。

\subsection*{三、.specify/ 目录的位置和作用}

\textbf{结论:}.specify/ 目录是存在于「项目 root 目录」里的普通文件夹,不是存在于 Claude、spec-kit、IDE 或宿主机某个隐式位置。它就是你 repo 里的文件。

\subsubsection*{1. .specify/ 的实际位置}

假设你的项目结构是:
\begin{verbatim}
my-project/
├── src/
├── package.json
├── CLAUDE.md
└── .specify/
    ├── memory/
    │   └── constitution.md
    └── specs/
        └── 001-feature/
            ├── spec.md
            └── plan.md
\end{verbatim}

那它满足:和 src/、package.json 同级;在 Git 仓库里(除非你 .gitignore 掉);你可以 cat 它、git diff 它、PR review 它。它不是某个隐藏的"AI 状态目录"。

\subsubsection*{2. 为什么很多人会误以为它"在宿主机"}

\textbf{错觉来源 1:}是 CLI 创建的 → 被误以为是"工具私有数据"。specify init 是 CLI 命令,但它做的只是在当前目录创建 .specify/、写模板文件。它不会写到 \textasciitilde{}/.config、\textasciitilde{}/.claude、IDE cache。

\textbf{错觉来源 2:}Claude "记得"它 → 让人以为是内存。Claude Code 每轮对话都会读取 .specify/memory/constitution.md、读取当前 feature 的 spec / plan。这会让人感觉"这是 Claude 的记忆",但实际上只是 Claude 每次都重新读项目文件。

\textbf{错觉来源 3:}.specify/ 是隐藏目录(以点开头),在 macOS / Linux 里 Finder / ls 默认不显示,新手很容易忽略。但这只是 Unix 命名习惯,不是"私有"。

\subsubsection*{3. 它和 CLAUDE.md 的关系}

\begin{itemize}
\item \textbf{CLAUDE.md}:项目 root,全局、跨 feature 的长期规则
\item \textbf{.specify/memory/constitution.md}:项目 root 下,spec-kit 流程中的"宪法"
\item \textbf{.specify/specs/*/spec.md}:项目 root 下某个 feature 的功能定义
\item \textbf{.specify/specs/*/plan.md}:项目 root 下该 feature 的技术方案
\end{itemize}

全部都在项目里,没有"跑到宿主机"的东西。

\subsubsection*{4. 什么时候它"看起来不在项目里"(特殊情况)}

\textbf{特殊情况 1:}你不是在 repo root 运行 agent。如果你在 cd src/ 然后 claude,Claude 可能读不到 .specify/,需要你设置 SPECIFY\_FEATURE。但文件依然在项目 root。

\textbf{特殊情况 2:}非 Git 项目。spec-kit 依赖 Git 分支来判断 feature。如果你没用 Git,需要手动指定 feature,但 .specify/ 仍然在当前目录。

\subsubsection*{5. 工程判断}

.specify/ 是"项目知识库",不是 agent cache。这意味着:可以 code review、可以回滚、可以跨 agent 使用、可以长期维护。而不是:临时状态、只能被某个 AI 读懂。

\subsection*{四、Claude 项目 Skill 的存储位置}

Claude 项目的 skill 分为三层:

\subsubsection*{1. 项目级「软 Skill」——写在项目文件里(最常见)}

\textbf{CLAUDE.md(最重要):}位置是 repo/CLAUDE.md。它是 Claude 的"项目宪法"、持久生效、每次对话都会加载。本质是一个「无名字、自动启动的 Skill 集合」。例如你写的不是提示,而是能力约束:
\begin{verbatim}
## Architecture Rules
- Do not introduce new services without approval
- Prefer composition over inheritance

## Coding Rules
- Always add tests for public functions
- Never modify migrations retroactively
\end{verbatim}
Claude 在\textbf{任何阶段}都会"启动这些 skills"。

\textbf{项目内 Markdown artifacts(spec / plan / checklist):}
\begin{verbatim}
.specify/
  ├── memory/constitution.md
  ├── specs/001-feature/spec.md
  ├── specs/001-feature/plan.md
\end{verbatim}
它们的作用是限定 Claude 当前阶段该干什么、约束"现在是 spec,不是实现"、约束"现在是 plan,不是写代码"。这是\textbf{阶段性 skill}。

\subsubsection*{2. 显式 Skill(你能"点名"的那种)}

\textbf{Slash Command Skills(spec-kit 的那类):}不在你的 repo 里,在 spec-kit / IDE 插件 / CLI 工具里。例如 /speckit.plan、/speckit.tasks。它们定义在 spec-kit 的 Python / Shell 脚本 agent integration layer。你只看到 /speckit.plan,但实际是环境注册了一个 skill,并把 contract 告诉 Claude。

\subsubsection*{3. 真正的「硬 Skill」(你一般写不到)}

\textbf{Tool / Function Skills(API 层):}位置是 Claude API 调用时注册、IDE / Agent 内部代码。例如:
\begin{verbatim}
tools: [
  {
    "name": "edit_file",
    "input_schema": { ... }
  }
]
\end{verbatim}
这些不是你写在项目里的,是宿主(Claude Code / Copilot / Droid)写的。这是 Claude 能"动文件 / 跑命令"的原因。

\textbf{判断口诀:}能进 Git 的,是你能控制的 skill;不能进 Git 的,是工具给 Claude 的能力。

\subsection*{五、spec-kit 实战 Demo}

\subsubsection*{Demo 目标}

做一个命令行 Todo 应用(Python),支持:add / list / done,数据存本地 JSON,不搞 Web、不搞框架。这个 demo 的重点\textbf{不是功能},而是流程。

\subsubsection*{前置条件确认}

已安装 specify-cli、已安装 Claude Code(或你正在用支持 slash command 的 Claude)、当前目录是一个未初始化的项目。

\subsubsection*{Step 0:初始化 spec-kit(只做一次)}

在项目 root 目录执行:
\begin{verbatim}
specify init . --ai claude
\end{verbatim}

会创建 .specify/、创建模板文件、初始化 Git(如果你没 --no-git)。此时目录结构大致是:
\begin{verbatim}
todo-demo/
├── .specify/
│   ├── memory/
│   │   └── constitution.md
│   ├── scripts/
│   └── templates/
├── CLAUDE.md
└── README.md
\end{verbatim}

到这一步为止,\textbf{还没写任何业务代码},完全正常。

\subsubsection*{Step 1:启动 Claude Code(关键)}

在\textbf{项目 root 目录}启动 Claude Code,例如 \texttt{claude} 或你 IDE 里的 Claude Code 面板。你需要确认能看到 /speckit.constitution、/speckit.specify 等命令。

\subsubsection*{Step 2:写「项目宪法」(constitution)}

在 Claude 里输入:
\begin{verbatim}
/speckit.constitution
Create simple, pragmatic principles:
- Prefer clarity over abstraction
- Keep dependencies minimal
- Write code that is easy to read and modify
- Use basic Python standard library only
- Favor small functions and explicit logic
\end{verbatim}

这一步会生成 / 更新 .specify/memory/constitution.md。你可以打开看看,它就是 Claude\textbf{长期遵守的规则}。

\subsubsection*{Step 3:写功能 spec(最关键的一步)}

在 Claude 里输入:
\begin{verbatim}
/speckit.specify
Build a simple command-line Todo application.
Users can:
- Add a todo item with text
- List all todo items
- Mark a todo item as completed

Todos are stored locally on disk.
This is a single-user tool and does not require authentication.
The focus is correctness and simplicity, not performance.
\end{verbatim}

执行后你会看到 .specify/specs/001-todo-cli/spec.md。\textbf{重点:现在还没有任何技术细节}。

\subsubsection*{Step 4:生成技术方案(plan)}

现在你才告诉 Claude\textbf{怎么实现}。输入:
\begin{verbatim}
/speckit.plan
Implement this as a Python CLI tool.
Use argparse for command parsing.
Persist todos in a local JSON file.
The application should be runnable via `python todo.py`.
\end{verbatim}

会生成 .specify/specs/001-todo-cli/ 目录下的 spec.md、plan.md 以及可能的 data-model.md / research.md。打开 plan.md 看一眼,你会发现结构清晰、没写代码,但已经"像工程文档"。

\subsubsection*{Step 5:生成任务拆分(tasks)}

现在让 Claude 把 plan 拆成\textbf{可执行步骤}。输入:
\begin{verbatim}
/speckit.tasks
\end{verbatim}

会生成 .specify/specs/001-todo-cli/tasks.md。你会看到类似:Create data model、Implement JSON storage、Implement add command、Implement list command、Implement complete command。\textbf{这是 Claude 接下来"唯一允许执行的事情清单"}。

\subsubsection*{Step 6:让 Claude 真正写代码}

最后一步,输入:
\begin{verbatim}
/speckit.implement
\end{verbatim}

Claude 会按 tasks.md 顺序执行、创建 todo.py、写代码、可能运行 Python 命令、遇到错误会修。最终你应该看到:
\begin{verbatim}
todo-demo/
├── todo.py
├── todos.json   (运行后生成)
└── .specify/
\end{verbatim}

你可以手动测试:
\begin{verbatim}
python todo.py add "Buy milk"
python todo.py list
python todo.py done 1
\end{verbatim}

\subsubsection*{你应该"体感到"的 3 件事(非常重要)}

\textbf{1. Claude 没有跳步:}不能直接写代码,每一步都有文件产物。

\textbf{2. 你可以随时打断 / 修改:}改 spec、重新 plan、重新 tasks。

\textbf{3. 这些不是"Claude 的记忆":}全在 repo 里,换一个 agent 也能继续。

\subsubsection*{进阶练习(可选)}

你可以试试问 Claude:
\begin{verbatim}
/speckit.analyze
\end{verbatim}

或:
\begin{verbatim}
Review the spec and plan and tell me if anything is over-engineered.
\end{verbatim}

\textbf{一句话总结:}这个 demo 的价值不在 Todo,而在于你已经完整跑通了一次「Spec-Driven + Claude Code」的工程闭环。

