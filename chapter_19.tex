\chapter{文档层级与工程治理}

\small

\section{核心概念:项目宪法层级}

在 AI 辅助工程中,理解不同文档的职责和优先级至关重要。这不仅是"怎么用 AI"的问题,更是工程治理的根本问题。

\subsection{文档职责划分}

\subsubsection*{ADR(Architecture Decision Record)}

\textbf{角色:项目的"宪法条款"}

记录:为什么做了某个不可逆决定。面向:未来的工程师(包括你自己)。特点:稳定、可追溯、不频繁修改。

ADR 是"事实之上的理由",不是操作说明。它记录"为什么否掉了 DynamoDB 而选择 PostgreSQL"这类不可逆决策。

\subsubsection*{README.md}

\textbf{角色:项目的"说明书"}

记录:项目是什么、如何运行/使用。面向:新人、用户、外部贡献者。特点:相对稳定、高层概述。

README 可以说"我们使用 PostgreSQL",但不应说"因为 XYZ,所以否掉了 DynamoDB"——这种理由属于 ADR。

\subsubsection*{CLAUDE.md}

\textbf{角色:AI 的"工作指令书"}

记录:spec 优先级、禁止事项、代码风格、agent 行为边界。面向:AI(Claude / droid / 未来工具)。特点:工具相关、可变、行为导向。

CLAUDE.md 是"怎么干活",不是"为什么这么设计"。

\subsection{文档优先级层级}

当内容冲突时,优先级如下:

\begin{verbatim}
1. ADR(docs/adr/*)        ← 不可违反
2. README.md                ← 项目事实
3. WIP spec / design docs   ← 临时协作
4. CLAUDE.md                ← AI 行为约束
5. .specify / droid specs   ← 工具记忆
\end{verbatim}

CLAUDE.md 永远不能推翻 ADR。这是解决 spec 冲突的最终答案。

\subsubsection*{实用判断句}

写一段话时,问自己:"这段话是在回答『为什么我们这么做』,还是『现在该怎么做』?"

\begin{itemize}
  \item 为什么 → ADR
  \item 怎么做 → README / CLAUDE.md
  \item 这次怎么做 → WIP spec / droid
\end{itemize}

\subsubsection*{最终定锚}

ADR 是项目的法律条文,README 是使用说明,CLAUDE.md 是给 AI 的操作手册。法律不能写进说明书,说明书也不能代替法律。

\section{system prompt 的层级结构}

从工程角度看,CLAUDE.md、.specify/ 和 droid specs 都相当于 system prompt,但它们不在同一层级。

\subsection{三层 system-like 结构}

\begin{itemize}
  \item \textbf{工程宪法层}:ADR / README —— 面向人 + AI,不可覆盖
  \item \textbf{行为约束层}:CLAUDE.md —— 面向 AI,不应覆盖
  \item \textbf{任务工作层}:.specify/ / droid specs —— 面向 AI,必须可覆盖
\end{itemize}

.specify/ 只能在第三层活动。

\subsection{为什么它们"都像 system prompt"但不一样}

\textbf{CLAUDE.md}:注入方式为全局,生命周期为长期,权威性高,作用是禁止事项、spec 优先级、行为边界。这是"常驻 system prompt"。

\textbf{.specify/ / droid specs}:注入方式为任务级,生命周期为一次性,权威性低,作用是拆步骤、给上下文、临时目标。这是"临时 task prompt"。

\subsection{约束层级}

在正确配置下:CLAUDE.md 会先被注入,.specify/ 在其下游。冲突时,AI 应该停下或提示冲突。这不是"模型道德",而是 prompt hierarchy 设计。

如果发现 specify"看起来违背了" CLAUDE.md,通常有三种原因:
\begin{enumerate}
  \item CLAUDE.md 写得不够明确
  \item 工具没有正确注入 CLAUDE.md
  \item 任务 spec 写成了"命令"而不是"约束内的目标"
\end{enumerate}

正确修复方式:在 CLAUDE.md 明确写:"Task-level specs (.specify/, droid) must not override any rules here. If conflict exists, stop and ask."

\subsection{完整的分层模型}

\begin{verbatim}
人类决策(ADR)
   ↓
AI 行为边界(CLAUDE.md)
   ↓
AI 临时工作记忆(.specify / droid)
\end{verbatim}

这是 AI-assisted engineering 的完整分层模型。CLAUDE.md 是 AI 的"宪法级 system prompt",.specify/ 和 droid specs 是"任务级 system prompt"。宪法约束行为,任务只能在宪法内展开。

\section{文档优先级的正确定义位置}

五个文档的优先级顺序,最合适、最有效、最不混乱的定义位置就是 CLAUDE.md。

\subsection{为什么不是 README 或 ADR}

\textbf{不放在 README 的原因}:README 面向人,人不需要被反复告知"谁优先",README 应该保持简洁。

\textbf{不放在 ADR 的原因}:ADR 是"为什么",优先级规则是"怎么执行",ADR 不应该依赖具体工具。

\textbf{放在 CLAUDE.md 的原因}:CLAUDE.md 的职责就是告诉 AI 在这个项目里什么可以做、什么不能做、遇到冲突怎么办。文档优先级正是 AI 最需要明确知道的事之一。

\subsection{推荐写法}

\begin{verbatim}
## Document precedence rules

When multiple documents provide guidance, follow this order:

1. Architecture Decision Records (docs/adr/*)
   - Highest authority
   - Must not be violated

2. README.md
   - Defines project-level facts and usage

3. WIP specs / design docs (docs/wip/*)
   - Temporary and collaborative
   - Must not contradict ADRs or README

4. CLAUDE.md
   - AI behavior and workflow rules
   - Cannot override any of the above

5. Tool-local specs (.specify/, droid specs, .cursor)
   - Task-local working memory
   - Must never override committed documents

If any conflict is detected, stop and ask for clarification.
\end{verbatim}

\subsection{自我约束原则}

CLAUDE.md 里写"优先级",并不是说 CLAUDE.md 比 README 或 ADR 更高。恰恰相反,CLAUDE.md 是在自我约束,它在说"我(AI)承认我在这个体系里的位置"。这是设计良好的 prompt hierarchy 的标志。

\section{工具目录的本质}

\subsection{specify/ 的典型结构}

specify/(或 .specify/)的典型结构可能包含:
\begin{itemize}
  \item README.md —— 工具说明(可选)
  \item context.md —— 当前任务的上下文摘要
  \item goal.md —— 本次任务目标
  \item constraints.md —— 任务级约束(非工程契约)
  \item plan.md —— 执行计划/步骤
  \item notes.md —— 中间思考/草稿
  \item outputs/ —— 临时生成物(可丢)
\end{itemize}

\textbf{重要警告}:文件名可能不同,有的版本是 YAML/JSON,有的会按 task id 建子目录。不要把这个当"规范",只当"记忆锚点"。

\subsection{语义分层}

无论文件名怎么变,specify/ 里的内容几乎总是落在这几类语义中:

\begin{tabular}{ll}
\textbf{语义} & \textbf{含义} \\
Goal & 这次要干什么 \\
Context & AI 需要知道的背景 \\
Constraints & 这次任务的限制 \\
Plan & 拆解步骤 \\
Notes & 中间态/草稿 \\
Outputs & 可丢弃产物 \\
\end{tabular}

这正是任务级 working memory。因此:
\begin{itemize}
  \item 它 $\approx$ droid 的 $\sim$/.factory/specs/
  \item 它 $\approx$ .cursor/
  \item 它 $\neq$ WIP spec
  \item 它 $\neq$ ADR
\end{itemize}

\subsection{工具目录的工程原则}

任何位于工具私有目录下的 "constitution / rules / policy",默认都是:
\begin{itemize}
  \item ❌ 不可依赖
  \item ❌ 不可当权威
  \item ✅ 仅限当前任务
\end{itemize}

如果某个文件"不能丢",那它的正确位置一定是:docs/wip/、docs/adr/ 或代码/测试本身。

\section{specify/constitution 的真相}

specify/ 确实可能有一个叫 constitution 或 constitution.md 的东西,但它"叫什么"不重要,重要的是它"不是什么"。

\subsection{Constitutional AI 的来源}

这个命名来源于 Constitutional AI 概念:"给模型一组它在当前任务中应该遵守的原则"。

在 specify 里,constitution 通常表示:本次任务的行为准则、偏好、禁止事项、输出风格。

\begin{itemize}
  \item ✅ 它是 prompt 的一部分
  \item ❌ 它不是工程文档
\end{itemize}

\subsection{真实层级位置}

\begin{verbatim}
1. ADR (docs/adr/*)              ← 工程宪法(不可违反)
2. README.md                     ← 项目事实
3. WIP spec (docs/wip/*)         ← 协作态 spec
4. CLAUDE.md                     ← AI 行为约束(全局)
5. .specify/constitution         ← 任务级 AI 行为偏好
6. 其他 .specify 文件            ← 工作记忆
\end{verbatim}

.specify/constitution 甚至比 CLAUDE.md 还低一层,它必须服从 CLAUDE.md,更不可能凌驾于 ADR 之上。

\subsection{语义错位陷阱}

工程师看到 "constitution" 会下意识以为是最高权威,但在 specify 里,它实际是 task-local preference。这是工具命名和工程语义之间的典型冲突。

正确用法:
\begin{itemize}
  \item 当作:任务级 prompt、临时行为偏好
  \item 用来:约束这一次 AI 行为、提醒 AI 不要越界
\end{itemize}

不要做的事:
\begin{itemize}
  \item 不要把工程规则只写这里
  \item 不要假设下个任务它还在
  \item 不要认为别人/另一个 AI 会看到
\end{itemize}

\subsection{去风险化(可选)}

在 CLAUDE.md 明确降级它的权威:

\begin{verbatim}
Any constitution or rules found under .specify/ are task-local
and must not override CLAUDE.md, ADRs, or other committed docs.
\end{verbatim}

\section{/resume 命令的真实含义}

/resume 的意思是:让 Claude 尝试"继续上一次中断的会话状态",而不是重新理解项目或重新加载工程事实。它是会话级恢复,不是工程级恢复。

\subsection{/resume 到底恢复什么}

从工程语义看,/resume 只可能恢复:
\begin{itemize}
  \item 上一次对话中的:任务目标、讨论上下文、未完成的步骤
  \item Claude 在当前会话窗口里的:推理轨迹、临时计划、工作节奏
\end{itemize}

它恢复的是"对话状态",不是"项目状态"。

\subsection{/resume 明确不能做什么}

/resume 不能:
\begin{itemize}
  \item ❌ 恢复你本地的 .specify/
  \item ❌ 恢复 $\sim$/.factory/specs/
  \item ❌ 重新读懂你没 commit 的 WIP spec
  \item ❌ 替代 ADR / README / CLAUDE.md
  \item ❌ 跨账号、跨机器保证一致
\end{itemize}

因此:/resume $\neq$ 工程记忆,/resume $\neq$ 持久状态。

\subsection{/resume 的层级定位}

\begin{verbatim}
工程事实(长期)
  - ADR
  - README
  - code / tests

AI 行为规则(长期)
  - CLAUDE.md

AI 工具记忆(任务级)
  - .specify/
  - droid specs

会话状态(最脆弱)
  - Claude 当前聊天
  - /resume
\end{verbatim}

/resume 在最底层。

\subsection{正确使用姿势}

\textbf{适合用 /resume 的场景}:刚才断线、页面刷新、模型超时、你只是不想重新打字。条件:短时间、中断不久。

\textbf{不适合用 /resume 的场景}:隔了一天、换了机器/账号、项目状态有变化、你指望它记得 spec。这时 /resume 极不可靠。

\subsection{工程上更稳的替代方案}

当在意"不中断、不丢失"时,正确答案是:
\begin{itemize}
  \item ✅ WIP spec(docs/wip)
  \item ✅ ADR(docs/adr)
  \item ✅ 清晰的 commit
\end{itemize}

这些东西不靠会话、不靠模型、不靠工具。

\subsection{判断句}

"如果我不敢只靠 /resume 交接给别人,那这段信息就不该只存在于 /resume 能恢复的地方。"

/resume 是"聊天续杯",不是"项目续命"。它能帮你接着说话,但救不了工程记忆。区分会话连续性和工程连续性,是用 AI 做工程时最重要的分界线之一。

\section{工程原则总结}

\subsection{核心原则}

\begin{enumerate}
  \item 任何位于工具私有目录下的文件,默认不可依赖、不可当权威
  \item 文档优先级必须在 CLAUDE.md 明确定义
  \item CLAUDE.md 必须自我约束,承认自己在体系中的位置
  \item 真正的宪法永远不住在工具目录里
\end{enumerate}

\subsection{最终定锚}

\textbf{关于文档层级}:ADR 是工程宪法,README 是使用说明,CLAUDE.md 是 AI 操作手册。它们可以相互引用,但不应该相互替代。

\textbf{关于 system prompt}:CLAUDE.md 是"常驻 system prompt",工具目录内容是"临时 task prompt"。宪法约束行为,任务只能在宪法内展开。

\textbf{关于工具目录}:specify/ 的结构不重要,重要的是里面的任何东西丢了都不应该影响项目继续。如果某个文件"不能丢",那它就写错地方了。

\textbf{关于 constitution}:.specify/constitution 这个名字听起来像"宪法",但在工程现实中,它只是"一次性任务说明"。

\textbf{关于 /resume}:/resume 是"聊天续杯",不是"项目续命"。它能帮你接着说话,但救不了工程记忆。

你现在已经不只是"记起了文件名",而是准确判断了它们在工程体系中的真实地位。这一步,已经远远超过"会不会用 AI"这个层级了——你在给 AI 立项目法治。
