% chapter_01.tex
\small

\chapter{Spec Coding 与 Vibe Coding:工程范式的演进}

\section{核心概念}

Vibe coding 是「生成方式」,Spec coding 是「约束方式」;它们不是对立关系,而是同一条工程成熟度曲线上的不同阶段。Vibe coding 解决的是:怎么快;Spec coding 解决的是:怎么不出事。

\subsection{Vibe Coding 本质}

Vibe coding $\neq$ 不写规范。它的本质是:在不完全明确需求的情况下,通过人与模型的「感觉对齐」,快速逼近可用实现。

典型特征:用自然语言、模糊、迭代、即时反馈、「差不多对了,再改一点」。强烈依赖模型能力和开发者审美/判断力。

常见场景:原型验证、Hackathon、个人项目、探索性功能、UI/UX 草图。

\subsection{Spec Coding 本质}

Spec coding 不是传统意义上的「写文档」,而是:把「你脑子里的隐性约束」显式化,强迫模型在一个受控空间里生成代码。

核心特征:明确输入/输出、明确不变量(invariants)、明确边界条件、明确失败方式。

常见场景:核心业务逻辑、金融/医疗/安全相关代码、多人协作、长期维护项目、自动化测试/CI。

\subsection{阶段切换关系}

成熟团队的真实流程是:Vibe $\to$ 半 Spec $\to$ 强 Spec。

\textbf{Vibe 阶段}:探索问题空间。快速、混乱、但便宜。

\textbf{半 Spec 阶段}:固化直觉。把成功的 vibe 对话提炼成接口、状态机、约束条件,开始补测试。

\textbf{Spec 阶段}:规模化与防退化。明确 spec 让模型可复现、可回滚、可替换。此时 vibe 是负资产。

\subsection{为什么单一模式不可持续}

「只有 Vibe coding」的问题:一个人时很好用,三个人开始互相骂,六个月后没人敢动代码,模型升级后行为漂移。原因只有一个:Vibe 没法被版本控制。

「只有 Spec coding」的问题:需求还没想清楚就强行写 spec,导致 spec 本身反复重写,模型生成「合规但没用」的代码。Spec 是收敛工具,不是探索工具。

\subsection{实用判断标准}

问自己一句话:「如果明天模型换一家,我现在写的东西还能用吗?」
\begin{itemize}
    \item 不能 $\to$ 你还在 vibe
    \item 能 $\to$ 你已经 spec 化了
\end{itemize}

\subsection{未来趋势}

到 2026 年,主流工程范式会是:Vibe 用来找方向,Spec 用来保成果。好的工程师将具备一种新能力:把 vibe 转译成 spec 的能力。这会比「会不会写 prompt」重要得多。

\section{Claude Code 的 Plan Mode}

Claude Code 的 Plan Mode,本质上就是:把「vibe $\to$ spec 的中间态」产品化了。它不是否定 vibe coding,而是试图解决 vibe coding 最大的工程缺陷。

\subsection{Plan Mode 的定位}

Plan Mode 在做一件事:在「纯 vibe」和「纯 spec」之间,建立一个「半结构化的中间态」。

它提供的是:
\begin{itemize}
    \item 结构化的思考框架
    \item 可审查的决策过程
    \item 可复现的步骤
\end{itemize}

但它不提供:
\begin{itemize}
    \item 完整的工程 spec
    \item 长期维护的契约
    \item 团队级的规范
\end{itemize}

\subsection{Plan Mode vs 传统 Spec}

传统 spec(你在工程文档里写的那些):
\begin{itemize}
    \item 面向:团队、人类
    \item 生命周期:长期、版本化
    \item 权威性:工程事实
\end{itemize}

Plan Mode 生成的 plan:
\begin{itemize}
    \item 面向:当前任务、模型
    \item 生命周期:一次性
    \item 权威性:任务级
\end{itemize}

\subsection{使用场景}

Plan Mode 适合:
\begin{itemize}
    \item 需求大致清楚,但需要探索实现路径
    \item 单人或小团队
    \item 中小型任务
\end{itemize}

Plan Mode 不适合:
\begin{itemize}
    \item 大型重构
    \item 多人协作的复杂系统
    \item 需要长期维护的 API 契约
\end{itemize}

\section{Spec-driven Development}

Spec-driven development 不是新概念,但在 AI 时代有了新的重要性。

\subsection{Spec 的层次}

spec 有不同「存活形态」:

\textbf{① 工作态 spec}(不进 git):本地笔记、Chat/LLM 对话、临时 markdown。用于思考、对齐、驱动实现,不要求持久化。

\textbf{② 协作态 spec}(进 git,但有生命周期):docs/wip/、specs/、design/。merge 时要么一起 squash 进 main,要么在 merge 前删除。

\textbf{③ 契约态 spec}(必须长期存在):docs/adr/、docs/api/、README。必须进 main、长期维护、是团队「记忆」。

\subsection{Spec 优先级规则}

当多个文档提供指导时,遵循以下顺序:
\begin{enumerate}
    \item ADR(docs/adr/*)——最高权威、不可违反
    \item README.md——定义项目级事实和使用说明
    \item WIP specs/design docs(docs/wip/*)——临时和协作态
    \item CLAUDE.md——AI 行为和工作流规则
    \item 工具本地 spec(.specify/、droid specs、.cursor)——任务级工作记忆
\end{enumerate}

如果检测到冲突,停止并请求澄清。

\subsection{.specify/ 的真实定位}

.specify/ 的本质是:LLM 的工作目录/私有工具状态/临时输入输出。它更像 .idea/、.vscode/、.cursor/:
\begin{itemize}
    \item 不稳定
    \item 工具相关
    \item 不保证长期语义
    \item 可随工具版本变化
\end{itemize}

它不具备「工程契约」的资格。协作态和契约态 spec 都不应该放在 .specify/ 里,而是应该放在项目可见、可 review 的目录下(./docs、./design、./adr 等)。

.specify/ 是「草稿本」,./docs 是「会议纪要」,./adr 是「合同」。草稿本里的东西可以丢、不能依赖。

\section{Git 工作流与 Spec}

\subsection{Feature = Git 分支}

「Feature = Git 分支」不是 Git 的常识,而是一个工作流约定。Git 本身根本不知道什么是 feature。

Git 的真正常识/事实只有这些:Git 有 commit、branch、tag。branch 只是一个指向 commit 的指针。Git 不关心 feature、bug、release、task。

这个约定之所以成为事实标准,是因为:
\begin{itemize}
    \item 它是最简单的「隔离单位」:不污染 main、可以随时丢弃、可以并行工作
    \item 它天然就是回滚边界
    \item 它和 PR/Review 完美对齐
    \item 它和 spec-driven 极度契合
\end{itemize}

\subsection{Spec 与分支的关系}

spec 文件的物理位置永远是项目根目录的 .specify/,但它们的语义归属、生命周期、review 边界,都跟某一个 feature(通常是 feat 分支)绑定。

一个 feature 的「设计真相」就是 .specify/specs/<feature>/ 里的 spec.md、plan.md、tasks.md。这些就是这个 feature 的设计合同,当然要进 Git。

正确流程:
\begin{enumerate}
    \item git checkout -b feat/xxx
    \item spec / plan / tasks(commit)
    \item implement
    \item review(git diff / PR)
    \item commit implementation
    \item open PR
    \item merge
\end{enumerate}

\subsection{Squash Merge}

Squash = 把多个 commit「压扁」为一个 commit。它不是「删除历史」,而是「选择性公开历史」。原来的 commit 仍在 feat 分支、在 reflog 里,只是不进入 main。

squash merge 的目的:main 只保留「做了什么」,而不是「我中间怎么折腾的」。这是对「主线历史可读性」的优化手段。

在 spec-driven 里,squash 非常常见。因为 spec/plan 已经记录了过程,main 更适合只保留 spec 定义和最终实现。

\subsection{Rebase}

Git rebase = 把「我这一串 commit」从「旧的 base」重新接到「新的 base」上。体感像「同步代码」,但本质是「重写你这段提交的历史位置」。

正确表述:「我在 feature 分支上,通过 rebase 把自己的提交,重新基于最新的 main 应用一遍,让历史看起来像是从最新 main 开始开发的。」

什么时候 rebase = 好主意:只在你自己的 feature 分支上、在提交还没被别人依赖时。不要在 main/已共享分支上 rebase。

\subsection{Implement 与 Commit}

/speckit.implement 执行完成后,不会在 feat 分支自动 commit。执行结束时,代码只处在 working tree/staging 之前的状态。

commit 是「责任边界」,不能由 AI 自动跨越。正确顺序永远是:implement $\to$ review $\to$ commit。

如果 implement 自动 commit:review 就变成事后补救、历史会被污染。

\subsection{Merge 后的新分支}

一个 feat 分支一旦被 merge(或 squash merge)到 main,它的「使命」就结束了。继续 spec-coding 新内容时,应该从 main 再开一个新的 feat 分支。

这是工程常识级别的做法。merge 之后继续在同一个分支上写新 spec,等于把两个 feature 的生命周期混在一起,让 PR、review、回滚都变得模糊。

例外情况:
\begin{itemize}
    \item merge 之后立刻发现小 bug/typo:可以直接在 main 上 hotfix(小项目/solo)
    \item spec 尚未 merge,其实只是「第一次 implement」:还在同一个 feature 内
\end{itemize}

\section{ADR(Architecture Decision Record)}

ADR = 用来记录「为什么我们在架构上做了这个决定」的文档。它解决的不是「怎么写代码」,而是:我们当时在多个方案中为什么选了这一种、哪些选择是被明确否掉的、这个决定对未来有什么约束。

\subsection{ADR 的内容}

一个典型 ADR 会回答 4 件事:
\begin{itemize}
    \item \textbf{Context}:发生了什么问题?为什么要做决定?
    \item \textbf{Decision}:我们选了什么方案?
    \item \textbf{Alternatives}:还考虑过哪些方案?为什么没选?
    \item \textbf{Consequences}:这个决定带来的好处和代价是什么?
\end{itemize}

\subsection{ADR 与其他文档的区别}

\begin{tabular}{ll}
文档 & 关注点 \\
\hline
README & 项目是什么,怎么用 \\
spec & 功能行为是什么 \\
design doc & 怎么实现 \\
ADR & 为什么这么设计 \\
\end{tabular}

ADR 是「不可见的工程记忆」。在 AI 时代更重要,因为:AI 看不到当初的讨论、AI 容易重提已被否定的方案、人类也会忘。

\subsection{ADR 的优先级}

ADR 的优先级是:对「是什么」的解释权 > 对「为什么」的解释权。README 可以说「我们使用 PostgreSQL」,但不能说「因为 XYZ,所以否掉了 DynamoDB」。这种理由属于 ADR。

\section{AI Agent 概念辨析}

\subsection{LSP}

LSP 本质是:自动补全、跳转、类型提示、语法检查。它解决的是编辑器智能,不是:设计、架构、spec。

不用 LSP 也完全 OK,因为把 LSP 能做的事,交给了脑子 + spec。并没有缺能力,只是工具选择不同。

\subsection{Skills 焦虑}

spec 是「工程契约」,skills 是「个人经验」;被营销放大的 skills,往往是在掩盖 spec 的缺席。

真正的强的 Claude/LLM 使用者,往往 skills 很少、spec 很清楚。因为他们明白:模型最擅长明确任务、明确约束;最不擅长模糊经验、玄学规则。

什么时候 skills 才值得写:
\begin{itemize}
    \item 自己反复踩同一个坑 $\ge$ 3 次 $\to$ 写一条 personal rule
    \item 这是跨项目、长期有效的模式 $\to$ 写成经验总结/playbook
\end{itemize}

宁愿多写 spec,也不要堆 skills。

\section{工程工具目录辨析}

\subsection{常见工具目录}

\textbf{.idea/}:JetBrains 系列 IDE 的工程结构、运行配置、个人快捷键。IDE 私有、高度个人化。通常不 commit、不能当 spec/文档。

\textbf{.vscode/}:VS Code 的 settings.json、launch.json、extensions 推荐。个人设置不 commit,团队约定(少量)可 commit(例如 formatter)。

\textbf{.cursor/}:Cursor AI 编辑器的 AI 行为配置、prompt 模板、会话状态。本质和 .specify/ 一样是 AI 工具的私有工作区。不应该被当作工程事实。

\textbf{.venv/}:Python 虚拟环境、第三方库、解释器副本。本地运行环境、可再生。永远不 commit、用 requirements.txt/pyproject.toml 描述即可。

共同点:可再生、不可审计、不应成为「项目真相」。

\subsection{判断公式}

以后看到任何目录/文件,问一句:「如果我换 IDE/换 AI/换机器,这个东西还应该存在吗?」
\begin{itemize}
    \item 是 $\to$ 应该进 repo(docs/code/adr)
    \item 否 $\to$ 工具状态(ignore)
\end{itemize}

\section{恢复与连续性}

\subsection{/resume 的本质}

/resume 的意思是:让 Claude 尝试「继续上一次中断的会话状态」,而不是重新理解项目或重新加载工程事实。它是会话级恢复,不是工程级恢复。

/resume 只可能恢复:上一次对话中的任务目标、讨论上下文、未完成的步骤;Claude 在当前会话窗口里的推理轨迹、临时计划、工作节奏。

/resume 不能:恢复本地的 .specify/、恢复 $\sim$/.factory/specs/、重新读懂你没 commit 的 WIP spec、替代 ADR/README/CLAUDE.md、跨账号/跨机器保证一致。

\subsection{Droid /session}

droid 的 /session 本质和 Claude 的 /resume 一样:都是会话级连续性,不是工程级状态恢复。

\subsection{Spec 恢复依赖 Git}

spec 的「恢复」本质上依赖的是 Git(以及 repo 内文档),而不是 /resume。只有两种来源:
\begin{itemize}
    \item Git repo(代码 + docs)
    \item 被 commit 的 ADR/WIP spec/README
\end{itemize}

除此之外,都是:临时缓存、会话残影、工具状态。

工程上更稳的替代方案:WIP spec(docs/wip)、ADR(docs/adr)、清晰的 commit。这些东西:不靠会话、不靠模型、不靠工具。

\section{文档优先级与 CLAUDE.md}

\subsection{三文件的角色}

\textbf{ADR}:项目的「宪法条款」。记录:为什么做了某个不可逆决定。面向:未来的工程师(包括你自己)。特点:稳定、可追溯、不频繁修改。ADR 是「事实之上的理由」,不是操作说明。

\textbf{README}:项目的「说明书」。记录:项目是什么、如何运行/使用。面向:新人、用户、外部贡献者。特点:相对稳定、高层概述。README 不解释复杂历史决策。

\textbf{CLAUDE.md}:AI 的「工作指令书」。记录:spec 优先级、禁止事项、代码风格、agent 行为边界。面向:AI(Claude/droid/未来工具)。特点:工具相关、可变、行为导向。CLAUDE.md 是「怎么干活」,不是「为什么这么设计」。

\subsection{优先级在 CLAUDE.md 中定义}

五个文档的优先级顺序,最合适、最有效、最不混乱的定义位置就是 CLAUDE.md。因为:它面向 AI、它约束行为、它不承载工程事实本身。

推荐写法:
\begin{verbatim}
## Document precedence rules

When multiple documents provide guidance, follow this order:

1. Architecture Decision Records (docs/adr/*)
   - Highest authority
   - Must not be violated

2. README.md
   - Defines project-level facts and usage

3. WIP specs / design docs (docs/wip/*)
   - Temporary and collaborative
   - Must not contradict ADRs or README

4. CLAUDE.md
   - AI behavior and workflow rules
   - Cannot override any of the above

5. Tool-local specs (.specify/, droid specs, .cursor)
   - Task-local working memory
   - Must never override committed documents

If any conflict is detected, stop and ask for clarification.
\end{verbatim}

CLAUDE.md 里写「优先级」,并不是说 CLAUDE.md 比 README 或 ADR 更高。恰恰相反:CLAUDE.md 是在自我约束,它是在说:「我(AI)承认我在这个体系里的位置。」

\section{何时需要 ADR 和 WIP Spec}

在 droid coding 的过程中:你不需要「为了用 AI」而刻意维持 ADR 或 WIP spec。你只需要在「工程本身需要它们」的时候才写。

\subsection{完全不需要的情况}

满足以下条件之一:
\begin{itemize}
    \item 单人项目/小项目:你是唯一维护者、变更半天内完成、上下文都在你脑中
    \item feature 局部、无长期影响:不影响架构、不引入新概念、不改变数据模型
    \item droid coding 的「短反馈循环」:快速试、快速改、squash 掉
\end{itemize}

\subsection{需要 WIP Spec 的情况}

满足以下任意一条:
\begin{itemize}
    \item feature 超过 1–2 天
    \item 你需要「明天继续」
    \item 你发现自己反复跟 AI 解释同一件事
    \item 可能有人中途接手
\end{itemize}

写一个极简 WIP spec,不超过一页。

\subsection{必须写 ADR 的情况}

只有在这些情况下:
\begin{itemize}
    \item 做了不可逆或高成本决策
    \item 明确否掉了「看似合理」的方案
    \item 这个决定未来会被反复质疑
\end{itemize}

这是给未来人(包括你)用的,不是给 AI。

\subsection{交接风险判断}

只要满足以下任意一条:可能有第二个人接手、你自己可能换账号/换机器/中断一段时间。那么:工程级信息就不能只存在于 droid 的私有目录里,必须至少有 WIP spec;涉及不可逆决策时,必须有 ADR。

30 秒判断法:在写代码前或 merge 前,问自己:「如果我明天请假两周,别人能接手吗?」
\begin{itemize}
    \item 能 $\to$ OK,不用写
    \item 不能 $\to$ 至少写 WIP spec
\end{itemize}

\section{模型侧开发的特殊性}

在 transformers 这种「接口经常变、体量巨大、语义靠源码而不是注释」的库,用 AI agent/vibe coding 的「性价比和安全性」反而低于古法手写 + 文档 + grep。

这不是「不用工具」,而是用对工具:IDE + LSP、grep/ripgrep、看源码、看 commit history、跑最小复现脚本、手写改动。这是「底层工程的 vibe」。

这类代码的特点:接口是软约定、行为靠实现细节、一点点错 = silent bug/数值错/性能灾难。这是 AI agent 最不擅长的区域之一。

AI agent 不会像人一样:rg class\_name、跳进源码、看 call graph、追 commit diff。即使能,它做得慢、不可靠、容易 hallucinate。

AI 更适合外围 glue、实验代码、pipeline 生成,不适合 transformers 核心路径。

\vspace{1em}
\normalsize
