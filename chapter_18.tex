% chapter_18.tex
% Lines 155214 to 164349 of original content
\small

\section{Sub-Agent 与工程流程}

关于 sub-agent 的详细定义、与 multi-agent 的对比、以及 Claude Code 的 sub-agent 机制,请参见"Multi-Agent 与 Sub-Agent 详解"章节。

在本章中,我们重点关注 sub-agent 在工程实践中的定位。

\subsection{droid 与 Claude Code 的 sub-agent 机制}

\subparagraph{droid 对 sub-agent 的支持}

droid 本身不显式暴露 sub-agent 概念,但用户的使用方式本质上已经在"手动模拟 sub-agent"。在单上下文工作流中,用户通过明确指令("现在只做 X"、"不要改别的")实现受控的子任务执行,用完即丢。这在认知模型上就是 sub-agent,只是没有被工具层包装。droid 选择的是"人当 orchestrator,而不是系统当 orchestrator",这对复杂工程反而更稳。

在 droid 单上下文、spec 驱动、人控节奏的工作方式中,sub-agent 是工具层优化,不是方法论升级,也不是必须品。

\textbf{核心原则:} sub-agent 的意义是减少上下文切换和重复解释,而不是替代判断或 spec。不用它不代表落后,理解它反而更不容易被营销带偏。

\subsection{Git 工作流:feat branch 与 squash merge}

\subsubsection{标准工程流}

手动 branch 一个 feat → implement → 测试没问题 → squash merge。这套流程本质上是"纯工程习惯驱动的 spec-coding",完全正确,不依赖任何 skills/agent 魔法,是目前最稳、最抗 AI 演进变化的方式。这是标准工程流 + LLM 辅助,而不是"AI 驱动工程"。

\textbf{工程意义:}
\begin{itemize}
\item 变更是有边界的
\item 失败是可回滚的
\item 主分支永远是干净的
\end{itemize}

\subsubsection{为什么这套流程在不依赖 agent 时更重要}

当不用自动化 spec/agent 时:
\begin{itemize}
\item git 就是状态管理系统
\item branch 就是 agent 隔离
\item squash 就是 context 压缩
\end{itemize}

这是 Git 级别的"sub-agent 隔离",比很多 multi-agent demo 更可靠。

\subsubsection{squash merge 的工程价值}

squash merge 把试错 prompt、调整、LLM 生成垃圾全部压扁,最终主分支只留下"spec → 实现成功"这一条事实。这非常符合 LLM 不可审计中间思考的现实。

对比 AI agent 流程(agent 自己开分支、自己 commit、人事后 review),feat branch + 人控 merge 在工程上反而更安全。

\textbf{核心原则:} "所有不确定性,都留在分支里;所有确定性,才进 main。"这条原则比任何 skills 都值钱。

spec-coding 不需要花哨工具;branch 是隔离,test 是验证,squash 是记忆压缩。

\subsection{Spec 的生命周期管理}

\subsubsection{spec 在 feat branch 与 main 中的差异}

在所述工作流中,spec 通常只存在于 feat branch 的工作上下文中,不一定、也不必须被 commit 到 main。但这需要精确前提。

\textbf{临时 spec 的本质:} 一次性工程决策说明,而不是 API 合同、架构规范、长期维护文档。这类 spec 用于:
\begin{itemize}
\item 思考
\item 对齐
\item 驱动实现
\end{itemize}

不要求持久化,squash 后自然消失。squash merge 的结果是 main 只保留"做成了什么",而不保留"中间怎么想的",这是非常成熟的工程选择。

\subsubsection{三种 spec 形态}

\textbf{① 工作态 spec(不进 git):}
\begin{itemize}
\item 形态:本地笔记、Chat/LLM 对话、临时 markdown(不 add)、issue comment/PR description
\item 特点:不要求持久化,squash 后自然消失
\end{itemize}

\textbf{② 协作态 spec(进 git,但有生命周期):}
\begin{itemize}
\item 形态:\texttt{docs/wip/feature-x.md}、\texttt{specs/feature-x.md}、\texttt{design/feature-x.md}
\item 做法:在 feat branch commit、PR 中 review、merge 时,要么一起 squash 进 main,要么在 merge 前删除。是否留下是显式决策。
\end{itemize}

\textbf{③ 契约态 spec(必须长期存在):}
\begin{itemize}
\item 形态:\texttt{docs/adr/0007-auth-flow.md}、\texttt{docs/api/*.md}、README 更新
\item 特点:必须进 main,长期维护,是团队"记忆"
\end{itemize}

\subsubsection{spec 去留判断标准}

判断句:"如果半年后我或别人看到 main,只看代码和测试,还需要知道这份 spec 吗?"

✅ 需要 → commit;❌ 不需要 → 不 commit。

\textbf{spec 必须进 main 的情况:}
\begin{itemize}
\item 跨 feature、影响未来决策
\item 对他人是约束
\item 需要被 review/回溯
\end{itemize}
例如:API 设计、数据模型约定、行为边界(edge cases)、业务规则。这时它不再是"临时 spec",而是设计文档/contract,应该 commit 到 main(如 \texttt{docs/}、\texttt{adr/})。

\textbf{spec 不需要进 main 的情况:}
\begin{itemize}
\item 只为这一次实现服务
\item 已经通过代码和测试体现
\item squash 后不再需要
\end{itemize}
留在 branch/对话/note 就好。

\subsubsection{未完成 feature 的 spec 处理}

对于未完成的 feature,spec \textbf{必须}存在于:
\begin{itemize}
\item PR 描述
\item issue
\item wip doc
\end{itemize}

而不是只存在于已 squash 掉的 branch,否则就是把项目状态藏进了历史里(反模式)。

\textbf{"两问法"决定 spec 去向:}
\begin{enumerate}
\item 这个 spec 对"feature 完成后"还有用吗?
\item 如果我明天请假,别人需要它吗?
\end{enumerate}
两个都否 → 不 commit/删除;任一为是 → commit 到 main(或 issue/PR)。

\subsubsection{spec 与 LLM-assisted coding 的关系}

LLM 的"中间 reasoning"不可审计,spec 里往往包含假设、试错、被推翻的方案。保留这些反而会污染长期认知。squash merge + 丢弃临时 spec,是在主动做"认知垃圾回收"。

临时 spec 是工作记忆,main 只保留长期记忆;squash merge 的意义不仅是压缩 commit,也是压缩思考噪音。

\subsection{Architecture Decision Record (ADR)}

\subsubsection{ADR 定义}

ADR = Architecture Decision Record,用来记录"为什么我们在架构上做了这个决定"的文档。它解决的不是"怎么写代码",而是:
\begin{itemize}
\item 我们当时在多个方案中,为什么选了这一种
\item 哪些选择是被明确否掉的
\item 这个决定对未来有什么约束
\end{itemize}

\subsubsection{ADR 与其他文档的区别}

\begin{tabular}{ll}
\textbf{文档} & \textbf{关注点} \\
README & 项目是什么,怎么用 \\
spec & 功能行为是什么 \\
design doc & 怎么实现 \\
ADR & 为什么这么设计 \\
\end{tabular}

ADR 是"不可见的工程记忆"。

\subsubsection{ADR 的核心结构}

典型 ADR 回答四件事:
\begin{enumerate}
\item \textbf{Context(背景):} 发生了什么问题?为什么要做决定?
\item \textbf{Decision(决定):} 我们选了什么方案?
\item \textbf{Alternatives(备选):} 还考虑过哪些方案?为什么没选?
\item \textbf{Consequences(后果):} 这个决定带来的好处和代价是什么?
\end{enumerate}

\textbf{示例:} \texttt{docs/adr/0005-use-postgres-for-events.md}
\begin{verbatim}
# ADR 0005: Use PostgreSQL for Event Storage

## Context
We need durable event storage with transactional guarantees.

## Decision
We will use PostgreSQL instead of DynamoDB.

## Alternatives
- DynamoDB: rejected due to lack of multi-row transactions.
- Kafka: rejected due to operational overhead.

## Consequences
- Strong consistency
- Higher operational cost
\end{verbatim}

短、明确、可长期保留。

\subsubsection{ADR 原则与 AI 时代的意义}

\textbf{工程实践原则:}
\begin{itemize}
\item 一条 ADR = 一个决定
\item 决定一旦生效,不随代码随意修改
\item 如果反悔 → 写新的 ADR(supersedes)
\end{itemize}

在 AI 时代,ADR 更重要,因为:
\begin{itemize}
\item AI 看不到当初的讨论
\item AI 容易重提已被否定的方案
\item 人类也会忘
\end{itemize}

ADR 的作用是阻止未来的人(包括 AI)重复走歪路。ADR 不是为了写给别人看的,而是为了防止未来的你、同事、以及 AI 重复犯已经付过代价的错误。

在 spec/agent/git 体系中:
\begin{itemize}
\item \texttt{.specify/}:AI 工作记忆
\item \texttt{docs/wip/}:协作 spec
\item \texttt{docs/adr/}:最终解释权
\end{itemize}

ADR 是 AI 不能越权的"宪法条款"。

\subsection{工程目录边界}

\subsubsection{开发工具目录的本质}

\textbf{.idea/}(JetBrains IDE):工程结构、运行配置、个人快捷键。IDE 私有、高度个人化,通常不 commit,不能当 spec/文档。

\textbf{.vscode/}(VS Code):settings.json、launch.json、extensions 推荐。个人设置不 commit,团队约定(少量)可 commit(如 formatter),可选但要克制。

\textbf{.cursor/}(Cursor AI 编辑器):AI 行为配置、prompt 模板、会话状态。本质和 \texttt{.specify/} 一样,是 AI 工具的私有工作区,不应该被当作工程事实,不适合作为长期记忆。

\textbf{.venv/}(Python 虚拟环境):第三方库、解释器副本。本地运行环境、可再生,永远不 commit,用 \texttt{requirements.txt}/\texttt{pyproject.toml} 描述即可。

\subsubsection{共同特征}

这些目录的共同点:可再生、不可审计、不应成为"项目真相"。

\textbf{工程判断公式:} "如果我换 IDE/换 AI/换机器,这个东西还应该存在吗?"
✅ 是 → 应该进 repo(docs/code/adr);❌ 否 → 工具状态(ignore)。

代码和文档是项目的身体,\texttt{.idea}/\texttt{.cursor}/\texttt{.venv} 只是工具的影子;影子会变,身体必须稳定。

\subsection{Droid Coding 中的文档策略}

\subsubsection{不需要为了 AI 而刻意维持文档}

在 droid coding 过程中,不需要"为了用 AI"而刻意维持 ADR 或 WIP spec,只需要在"工程本身需要它们"的时候才写。AI 不会要求升级工程复杂度。

\textbf{ADR/WIP spec 的唯一正当理由:} 人类需要、团队需要、未来需要。而不是"AI 最佳实践"、"别人都在用"、"不写就落后"。

\subsubsection{完全不需要 ADR/WIP spec 的情况}

\begin{itemize}
\item \textbf{单人项目/小项目:} 唯一维护者、变更半天内完成、上下文都在脑中。不写是更高效的工程决策。
\item \textbf{feature 局部、无长期影响:} 不影响架构、不引入新概念、不改变数据模型。让代码和测试说话就够了。
\item \textbf{droid coding 的"短反馈循环":} 快速试、快速改、squash 掉。ADR 反而是噪音。
\end{itemize}

\subsubsection{应该写 WIP spec 的情况}

满足以下任意一条:
\begin{itemize}
\item feature 超过 1-2 天
\item 需要停下来明天继续
\item 反复跟 AI 解释同一件事
\item 有人可能中途接手
\end{itemize}

写一个极简 WIP spec,不超过一页:
\begin{verbatim}
# Feature X – WIP
- Goal: ...
- Non-goals: ...
- Open questions:
  - ...
\end{verbatim}

\subsubsection{必须写 ADR 的情况}

只在这些情况下写 ADR:
\begin{itemize}
\item 做了不可逆或高成本决策
\item 明确否掉了一些看似合理的方案
\item 这个决定未来会被反复质疑
\end{itemize}

这是给未来人(包括自己)用的,不是给 AI。

\subsubsection{防焦虑原则}

任何"为了 AI 而引入的工程流程",如果不能明显帮助人类,就应该被拒绝。

ADR 和 WIP spec 是"工程需要时才打开的工具",不是 droid coding 的前置条件。完全可以继续写代码,而不欠任何"文档债"。
