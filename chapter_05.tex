% chapter_05.tex
\small

\chapter{Claude Code Plan Mode 与 Spec-Coding Agents}

\section{Plan Mode 的定位}

Claude Code 的 Plan Mode,本质上就是:把「vibe $\to$ spec 的中间态」产品化了。它不是否定 vibe coding,而是试图解决 vibe coding 最大的工程缺陷。

\subsection{Plan Mode 的真实定位}

Plan Mode 在做一件事:在「纯 vibe」和「纯 spec」之间,建立一个「半结构化的中间态」。

它提供的是:
\begin{itemize}
    \item 结构化的思考框架
    \item 可审查的决策过程
    \item 可复现的步骤
\end{itemize}

但它不提供:
\begin{itemize}
    \item 完整的工程 spec
    \item 长期维护的契约
    \item 团队级的规范
\end{itemize}

\subsection{Plan Mode 在 vibe / spec 光谱中的位置}

如果画一条轴:纯 Vibe $-$\,-$-$ Plan Mode $-$\,-$-$ 纯 Spec

Plan Mode 处在中间偏右,其特点是:
\begin{itemize}
    \item 仍然允许需求模糊
    \item 仍然用自然语言
    \item 仍然是人机对话驱动
    \item 但不允许「直接拍脑袋写代码」
\end{itemize}

它是一个「vibe 的刹车系统」。

\subsection{为什么 Plan Mode 会出现}

\textbf{Vibe coding 在 scale 时必然崩}。vibe coding 的特点:demo 很爽,项目一大就失控。问题不在模型,而在:隐性假设、未对齐的理解、需求漂移。

Plan Mode 的目标是解决:「你以为模型懂了,但它懂的是另一个东西。」

反过来,\textbf{Spec 太重,不适合探索期}。让用户一开始就写 RFC、OpenAPI、状态机、不变量,用户直接跑了。

Plan Mode 是一种折中:不要求你先会工程,但要求你在写代码前先点头确认理解。

\subsection{Plan Mode vs Spec Coding}

\begin{tabular}{lll}
维度 & Plan Mode & Spec Coding \\
\hline
表达形式 & 自然语言为主 & 形式化 / 半形式化 \\
可执行性 & ❌ 不能直接执行 & ✅ 可被验证 \\
稳定性 & 中 & 高 \\
是否可版本控制 & ⚠️ 勉强 & ✅ \\
是否抗模型漂移 & ❌ & ✅ \\
\end{tabular}

Plan $\neq$ Spec,Plan 是「spec 的草稿」。

\subsection{成熟使用方式}

真正厉害的团队是这么用的:
\begin{itemize}
    \item Vibe:快速试方向
    \item Plan Mode:冻结一次「共识理解」
    \item Spec:把关键路径形式化
    \item Code:让模型在约束里跑
\end{itemize}

Plan Mode 是 vibe $\to$ spec 的翻译器。

\section{Plan Mode vs Spec-Coding Agent}

Plan Mode 的 plan,本质上是「一次性对齐产物」;Spec-coding agent(比如 Factory 的 Droid)的 spec,是「可演化的一等工程资产」。这不是能力强弱问题,而是工程哲学完全不同。

\subsection{Claude Code Plan Mode:为什么「不能自己改 plan」}

在 Claude Code 里,Plan Mode 生成的 plan 是给人看、给人确认的,不是 agent 内部的可编辑状态。你想「改 plan」实际上是重新 prompt,重新生成一份新的 plan。它更像一次性的「共识快照」,而不是持续维护的文档。

Anthropic 这么做是刻意的:
\begin{itemize}
    \item 避免用户以为这是「正式 spec」
    \item 避免 plan 演化但代码不一致
    \item 避免用户陷入「维护自然语言 spec 的幻觉」
\end{itemize}

\subsection{Spec-Coding Agent(Droid)的根本不同}

Droid 的 spec 是:
\begin{itemize}
    \item 可编辑
    \item 可 diff
    \item 可版本控制
    \item 可被 agent 反复读取、校验、修改
\end{itemize}

Spec 是 agent 的「世界模型」,而不是一次对话产物。

\textbf{Spec 是一等公民}:在 Droid 里,Code $\neq$ 真理,Spec $>$ Code,agent 的职责是保持 code 与 spec 一致,而不是「猜你想要什么」。

\textbf{Agent 可以自己发现 spec 问题}:Droid 可以写代码时发现冲突,回头修改 spec,再重新生成代码,形成一个闭环。这一步才是真正的 spec coding。

\subsection{本质分水岭:有没有「可变的内部状态」}

\begin{tabular}{lll}
维度 & Plan Mode & Spec-Coding Agent \\
\hline
Plan / Spec 地位 & 输出给人看的 & Agent 的内部状态 \\
是否可被 agent 修改 & ❌ & ✅ \\
是否参与推理 & ❌(一次性) & ✅(持续) \\
是否可长期维护 & ❌ & ✅ \\
适合项目规模 & 小 / 中 & 中 / 大 \\
\end{tabular}

能不能「自己改 spec」,不是小功能,而是智能体级别的差异。

\subsection{Plan Mode vs Droid 的产品定位}

Spec-coding agent:学习成本高、心智负担大、适合专业团队。

Plan Mode:普通开发者也能用、不改变工作流、风险可控。

Plan Mode 是大众产品,Droid 是工程系统。

\subsection{实用判断标准}

「这个项目三个月后,我敢不敢让一个新模型只看 spec 就接手?」
\begin{itemize}
    \item 不敢 $\to$ Plan Mode 足够
    \item 敢 $\to$ 你需要 Droid 这一类 spec-coding agent
\end{itemize}

一句话终极总结:Plan Mode 是「让模型先把话说清楚」,Spec-coding agent 是「让模型对说过的话负责」。

\section{Spec-Coding Agents 生态}

截至 2026 年初,真正意义上「把 spec 当一等工程资产、并允许 agent 自我修改 spec」的产品,Droid 依然是极少数中的极少数。

\subsection{分层地图}

\textbf{第一层:真 · Spec-Coding Agent(极少)}

spec 是一等公民,spec 可编辑、可 diff,agent 可以修改 spec,code 必须服从 spec。

\textbf{代表性产品}:

1️⃣ Droid(Factory.ai):Spec 是中心真理,Agent 维护 spec $\leftrightarrow$ code 一致性,可以:发现 spec 冲突、回滚 spec、重写实现。这是「范式级」的 spec coding。

2️⃣ 一些内部 / 企业定制系统(未公开):大厂内部(金融 / 防务 / 云服务),基于形式化 spec、DSL、状态机。但不对外,没有通用产品形态。

公开、可用、通用的,基本只有 Droid。

\textbf{第二层:半 Spec Agent(看起来很像,但差一口气)}

有 spec 结构,但 spec 不是真正「可演化真理」,agent 不能对 spec 负责。

典型代表:SWE-agent / AutoCodeRover / OpenDevin 有 task description、constraints,但 spec 是 prompt 不是长期资产,agent 不会维护它。这是「任务驱动 agent」,不是 spec coding。

Devin(Cognition):会规划、会分解任务,但 Plan $\neq$ Spec,没有可维护的工程不变量。Devin 是 autonomous worker,不是 spec executor。

\textbf{第三层:Plan / Blueprint 工具(Claude Code 在这里)}

显式 plan、人确认,但 agent 不维护 plan、plan 会过期。

代表:Claude Code Plan Mode、Cursor 的 planning step、ChatGPT 的「let's plan first」。

这是 vibe $\to$ spec 的过渡层。

\textbf{第四层:「伪 spec」(营销用语,务必小心)}

这些会用词如:「AI understands your requirements」、「Declarative development」、「No-code spec」,但实际上:spec = prompt,不可验证、不可回滚、不参与 agent 推理。

和 spec coding 没有工程意义上的关系。

\subsection{为什么「真正的 spec-coding agent」这么少?}

原因有 4 个:

\textbf{1️⃣ Spec 本身就难}:Spec 是最难的工程产物之一。人都写不好,更别说让 agent 维护。

\textbf{2️⃣ 心智负担太高}:Spec-coding agent 要求用户接受「代码不是第一真理」,学会约束思维、不变量思维。这直接淘汰 90\% 用户。

\textbf{3️⃣ Demo 不好看}:spec 收敛模型,不会「神奇写一切」,投资人和用户都不兴奋。

\textbf{4️⃣ 责任与法律风险}:一旦 agent 修改 spec、承诺一致性,那就意味着更高的责任暗示、更复杂的合规问题。

\subsection{一个非常残酷但真实的判断}

Spec-coding agent 不是「下一个 Copilot」,而是「下一个 CI/CD 级别的基础设施」。

所以:不会爆发、不会全民使用,但一旦成功,会深度锁定工程组织。

\subsection{一句话终极总结}

除了 Droid,目前几乎没有真正把「spec 作为可演化工程真理」的通用 spec-coding agent;绝大多数所谓竞品,仍停留在 plan 或 task 层。

\vspace{1em}
\normalsize
