\chapter{Spec Coding 实战与工程深化}

\small

\section{LSP-first 开发范式}

\subsection{LSP 的核心价值}

LSP(Language Server Protocol)在 AI 辅助开发中的价值不仅是"代码补全",而是建立了一个稳定的工具层。LSP-first 开发意味着:

\begin{itemize}
  \item \textbf{语义稳定性}:通过 LSP 获取准确的定义、引用、类型信息
  \item \textbf{工具解耦}:不依赖特定 IDE 的黑盒功能
  \item \textbf{可审计性}:每个操作都有明确的语义来源
\end{itemize}

\subsection{LSP-first Prompt 模板}

\begin{verbatim}
使用 LSP 工具查询以下信息:
1. 获取当前光标位置的定义 (goToDefinition)
2. 查找所有引用点 (findReferences)
3. 显示文档和类型信息 (hover)
基于 LSP 返回的准确语义,分析代码结构并给出建议。
\end{verbatim}

\section{Claude Code 文档体系深度解析}

\subsection{PRD / plan / CLAUDE.md 的层级关系}

\textbf{PRD.md}(产品需求文档):
\begin{itemize}
  \item 定义"做什么"和"为什么做"
  \item 作为参考上下文加载,非 system prompt
  \item 描述产品目标、用户需求、功能范围
\end{itemize}

\textbf{plan.md}(实施计划):
\begin{itemize}
  \item 定义"怎么做"的技术方案
  \item 同样作为参考材料
  \item 包含架构设计、技术选型、实施步骤
\end{itemize}

\textbf{CLAUDE.md}(项目约束规则):
\begin{itemize}
  \item 优先级最高,接近"会话级 system context"
  \item 定义不可违反的行为约束
  \item 包含编码规范、安全要求、项目特定规则
\end{itemize}

\subsection{文档遗忘问题与应对策略}

Claude Code 的三文件体系存在以下局限:

\begin{enumerate}
  \item \textbf{跨会话遗忘}:关闭后重新打开,需要重新加载
  \item \textbf{上下文溢出}:新对话会挤压旧信息
  \item \textbf{实现 drift}:agent 以当前代码为事实源,而非文档
\end{enumerate}

\textbf{应对策略}:
\begin{enumerate}
  \item \textbf{锚点最小化}:只保留 5-10 条必须不变的硬约束
  \item \textbf{提升到 CLAUDE.md}:绝对不能违反的行为规则
  \item \textbf{Re-anchor Prompt}:关键修改前强制重新阅读文档
  \item \textbf{阶段化 plan}:只保留当前阶段的计划
\end{enumerate}

\section{Drift 与 Hallucination 机制}

\subsection{Drift 的本质}

Drift 是 LLM 在长期交互中逐渐偏离初始意图的现象:

\begin{itemize}
  \item \textbf{语义漂移}:对同一概念的理解逐渐变化
  \item \textbf{约束松弛}:初期遵守的规则逐渐被忽略
  \item \textbf{上下文污染}:早期决策被后期信息覆盖
\end{itemize}

\subsection{Drift 产生的机制}

\begin{enumerate}
  \item \textbf{上下文窗口有限}:新信息挤压旧信息
  \item \textbf{概率性生成}:每次生成都是重新采样,存在随机性
  \item \textbf{缺乏显式锚点}:没有强制性的真值源
\end{enumerate}

\subsection{防止 Drift 的方法}

\begin{enumerate}
  \item \textbf{显式锚点}:用 CLAUDE.md 定义不可违反的约束
  \item \textbf{定期 Re-anchor}:在关键节点重新加载核心文档
  \item \textbf{版本化 spec}:让决策有明确的"真相时刻"
  \item \textbf{分离对话与执行}:决策阶段和执行阶段分开
\end{enumerate}

\section{Droid 系统架构}

\subsection{Droid 与 Claude Code 的根本差异}

\begin{table}[h]
\centering
\begin{tabular}{lll}
\hline
\textbf{文件} & \textbf{Claude Code} & \textbf{Droid} \\
\hline
PRD.md & 参考材料 & 系统真理(Authoritative State) \\
plan.md & 设计说明 & 可执行结构(Execution Graph) \\
CLAUDE.md & 说明文件 & Agent 宪法(Policy Layer) \\
\hline
\end{tabular}
\end{table}

\subsection{Droid 的核心假设}

Droid 的设计假设是"agent 默认不可信":

\begin{itemize}
  \item spec 拥有系统级否决权
  \item plan 是执行约束,而非建议
  \item agent 行为规则是系统的一部分,不可随意覆盖
\end{itemize}

\subsection{Drift 防护机制}

Droid 通过以下机制防止 drift:

\begin{enumerate}
  \item \textbf{System-level enforcement}:spec 在系统层面强制执行
  \item \textbf{Explicit state tracking}:维护显式的 spec state
  \item \textbf{Reference obligation}:任何实现必须引用 spec 条目
  \item \textbf{Conflict detection}:显式检测和处理冲突
\end{enumerate}

\section{Spec-kit 工具链}

\subsection{spec-kit 的本质}

spec-kit 不是"让 Claude 更聪明",而是把靠自觉维护的 PRD/plan/CLAUDE.md 流程,变成:

\begin{itemize}
  \item \textbf{可执行}:spec 可被机器解析和验证
  \item \textbf{可审计}:每个改动都能追溯到 spec
  \item \textbf{可持续}:建立长期的工程记忆
\end{itemize}

\subsection{spec-kit 与手写文档的对比}

\textbf{spec-kit + Claude Code}:
\begin{itemize}
  \item 强制 artifact 链条(不能跳步)
  \item 明确"哪一步可以问不确定问题"
  \item spec/plan/tasks 都是机器可解析的
\end{itemize}

\textbf{手写文档}:
\begin{itemize}
  \item 完全可控
  \item 高度贴合个人习惯
  \item 容易出现 drift、不一致
\end{itemize}

\subsection{spec-kit 实战流程}

\begin{enumerate}
  \item \texttt{specify init . --ai claude}:初始化项目
  \item \texttt{/speckit.constitution}:写项目宪法
  \item \texttt{/speckit.specify}:写功能 spec
  \item \texttt{/speckit.plan}:生成技术方案
  \item \texttt{/speckit.tasks}:生成任务拆分
  \item \texttt{/speckit.implement}:让 Claude 写代码
\end{enumerate}

\section{Git 工作流与 Spec}

\subsection{Feature = Git 分支约定}

"Feature = Git 分支"是被广泛采用的工程约定:

\begin{itemize}
  \item 最简单的隔离单位
  \item 天然的回滚边界
  \item 与 PR/Review 完美对齐
  \item 与 spec-driven 极度契合
\end{itemize}

\subsection{Pull Request 的本质}

PR 不是 Git 概念,而是协作平台的"变更提议机制":

\begin{itemize}
  \item 提供意图边界:"这次改动想干嘛"
  \item 提供讨论空间:对话不污染 commit 历史
  \item 提供安全闸门:review 后才 merge
\end{itemize}

\subsection{Merge 策略对比}

\begin{table}[h]
\centering
\begin{tabular}{lll}
\hline
\textbf{策略} & \textbf{feature 内 commit} & \textbf{main 上 commit} \\
\hline
普通 merge & 原样保留 & 原样保留 \\
squash merge & 保留在 feat & 不进入 main \\
rebase + FF & 改写后保留 & 线性进入 main \\
\hline
\end{tabular}
\end{table}

\textbf{推荐}:在 spec-driven 工作流中,squash merge 是最常见、最稳妥的选择。

\subsection{Squash 的本质}

Squash = 把多个 commit "压扁"为一个 commit。其目的是:

\begin{itemize}
  \item main 只保留"做了什么"
  \item 不保留"中间怎么折腾的"
  \item 历史被整理过,只留下对外可读的结果
\end{itemize}

\section{spec / plan / tasks 与 Git 分支的对应关系}

\subsection{核心对应表}

\begin{table}[h]
\centering
\begin{tabular}{lllll}
\hline
\textbf{层级} & \textbf{文件} & \textbf{位置} & \textbf{语义归属} & \textbf{Git 分支} \\
\hline
spec & spec.md & .specify/specs/xxx/ & Feature & feat/\emph{} 分支 \\
plan & plan.md & .specify/specs/xxx/ & Feature & feat/\emph{} 分支 \\
tasks & tasks.md & .specify/specs/xxx/ & Feature & feat/\emph{} 分支 \\
code & src/... & 项目源码目录 & Feature & feat/\emph{} 分支 \\
constitution & constitution.md & .specify/memory/ & 项目 & 所有分支共享 \\
\hline
\end{tabular}
\end{table}

\subsection{时间轴视角}

\begin{verbatim}
(main)
│
├── git checkout -b feat/todo-cli
│
(feat/todo-cli)
│
├── spec.md      ← 决策:要做什么
├── plan.md      ← 决策:怎么做
├── tasks.md     ← 决策:分几步做
│
├── /speckit.implement
│
├── code changes ← 执行结果
│
├── PR review    ← 验证:是否按 spec
│
└── merge → main
\end{verbatim}

\subsection{implement 和 merge 的关系}

\textbf{关键事实}:
\begin{itemize}
  \item \texttt{/speckit.implement} 不会自动 commit
  \item implement 和 merge 不是原子操作
  \item merge 永远是人类的权力边界
\end{itemize}

\textbf{推荐节奏}:
\begin{enumerate}
  \item spec/plan/tasks 完成后 commit
  \item implement 执行
  \item review(git diff / PR)
  \item 满意后 commit implementation
  \item merge(squash 推荐)
\end{enumerate}

\section{Spec 的三种形态}

\subsection{形态分类}

\textbf{工作态 spec}(不进 git):
\begin{itemize}
  \item 本地笔记、对话、临时 markdown
  \item 用于思考、对齐、驱动实现
  \item 不要求持久化,squash 后自然消失
\end{itemize}

\textbf{协作态 spec}(进 git,有生命周期):
\begin{itemize}
  \item docs/wip/、specs/、design/ 目录
  \item PR 中 review,merge 时决定是否留下
\end{itemize}

\textbf{契约态 spec}(必须长期存在):
\begin{itemize}
  \item docs/adr/、docs/api/、README
  \item 必须进 main,长期维护,是团队记忆
\end{itemize}

\subsection{判断准则}

"两问法"决定 spec 去向:
\begin{enumerate}
  \item 这个 spec 对"feature 完成后"还有用吗?
  \item 如果我明天请假,别人需要它吗?
\end{enumerate}

\begin{itemize}
  \item 两个都否 → 不 commit / 删除
  \item 任一是 → commit 到 main(或 issue/PR)
\end{itemize}

\section{Rebase 的本质}

\subsection{机制说明}

Git rebase = 把"我这一串 commit"从"旧的 base"重新接到"新的 base"上。

\begin{verbatim}
原始状态:
  main:  A ── B ── C
           \
  feat:           D ── E

rebase 后:
  main:  A ── B ── C
           \
  feat:               D' ── E'
\end{verbatim}

\subsection{与 Stash 的关系}

stash 是"可选的前置操作",不是 rebase 的一部分:
\begin{itemize}
  \item 工作区干净 → 直接 rebase
  \item 有未提交修改 → Git 会拒绝
  \item 所以:git stash → git rebase → git stash pop
\end{itemize}

\subsection{适用时机}

\begin{itemize}
  \item[\textbf{✅}] 只在自己的 feature 分支上
  \item[\textbf{✅}] 提交还没被别人依赖时
  \item[\textbf{❌}] 不要在 main/已共享分支上 rebase
\end{itemize}

\section{.specify/ 目录定位}

\subsection{本质澄清}

.specify/ 的本质是:LLM 的工作目录/私有工具状态/临时输入输出。

它更像 .idea/、.vscode/、.cursor/:
\begin{itemize}
  \item ❌ 不稳定
  \item ❌ 工具相关
  \item ❌ 不保证长期语义
  \item ✅ 可随工具版本变化
\end{itemize}

\subsection{正确位置}

\begin{itemize}
  \item 工具私有状态 → .specify/、.claude/
  \item 临时工作 spec → 不进 git、对话、note
  \item 协作态 spec → docs/wip/、specs/(PR 时决定去留)
  \item 契约态 spec → docs/adr/、docs/api/、必须进 main
\end{itemize}

\section{Claude Skills 的本质}

\subsection{Skills 的本质}

所谓的"Claude skills"本质是:
\begin{itemize}
  \item 一组固定的 prompt 模板
  \item 一些 if/when/condition
  \item 一些行为约束
  \item 一点"人类写的经验总结"
\end{itemize}

\subsection{Skills 的不可靠性}

\begin{enumerate}
  \item \textbf{没有稳定接口}:模型一升级,行为就可能变化
  \item \textbf{经验叠加}:是"我觉得这样好",不是明确契约
  \item \textbf{不可迁移}:博主的项目背景与你的完全不同
  \item \textbf{营销载体}:制造信息差、暗示稀缺、承诺捷径
\end{enumerate}

\subsection{Spec-driven 与 Skills 的关系}

在 spec-driven 工作流中:
\begin{itemize}
  \item spec / plan / tasks 已经记录了决策
  \item 不需要靠"skills 猜"
  \item 模型最擅长明确任务、明确约束
  \item 最不擅长模糊经验、玄学规则
\end{itemize}

\section{工程实践总结}

\subsection{Spec Coding 的核心原则}

\begin{enumerate}
  \item 不迷信 skills、不迷信 agent
  \item spec 只服务确定性
  \item git 是最终事实源
  \item 所有不确定性留在分支,所有确定性进 main
\end{enumerate}

\subsection{终极心法}

临时 spec 是工作记忆,main 只保留长期记忆;squash merge 的意义,不仅是压缩 commit,也是压缩思考噪音。

\textbf{安全原则}:"所有不确定性,都留在分支里;所有确定性,才进 main。"
