% chapter_03.tex - Spec Coding 实战深化
\small

\chapter{Spec Coding 实战深化}

\section{LSP-first 开发范式}

\subsection{LSP 是什么}

LSP(Language Server Protocol)是一套标准协议,用来把"语言理解能力"从编辑器中解耦出来。编辑器(VS Code/Vim/IntelliJ/Terminal)通过 LSP 协议与语言服务器通信,后者了解语法、类型、项目结构。

\subsection{有无 LSP 的本质区别}

\paragraph{没有 LSP(纯文本 AI)}

AI 看到的是一堆文件内容 + 模糊上下文。结果通常是:猜 API、猜类型、容易 hallucinate、改错文件/行号。

\paragraph{有 LSP(工程级 AI)}

AI 可以查询真实符号表、知道函数定义在哪、知道类型/引用/依赖、精确跳转/重构。这是"像工程师一样改代码"的前提。

\subsection{LSP 是 Coding Agent 的分水岭}

没有 LSP 的 AI = 聊天机器人,有 LSP 的 AI = 工程协作者。尤其对 spec-coding、refactor、大项目、多文件一致性、长期维护至关重要。

\subsection{各工具对 LSP 的支持}

\paragraph{OpenCode —— 显式、原生、卖点级支持}

LSP enabled,Automatically loads the right LSPs for the LLM。OpenCode 会自动启动对应语言的 LSP,把 LSP 查询结果作为 agent 的输入。LSP 是一等公民,不是"顺带支持"。这是 OpenCode 非常重要的定位优势。

\paragraph{Claude Code —— 隐式支持(但不可控)}

内部确实使用了 LSP/AST/repo analysis,但对用户完全不可见、不可配置,无法选择 LSP、强制读取、验证是否使用。结论:Claude Code "有 LSP 能力,但你无法把它当成系统依赖"。

\paragraph{Droid —— 必然支持(但抽象在 agent 层)}

从设计上必须有 LSP/AST/graph,否则无法维护 spec 一致性。但通常不会跟你说"这是 LSP",而是说"结构化代码理解"、"semantic model"。LSP 在 Droid 里是底层基础设施,不是功能点。

\subsection{LSP 与 spec-coding 的关系}

LSP $\neq$ Memory,但 LSP 是"spec 能落地"的前提条件。Memory/spec 决定"应该做什么",LSP 决定"怎么安全地做"。没有 LSP,spec-coding 只能停留在纸面。

\subsection{为什么 OpenCode 把显式 LSP 当卖点}

OpenCode 把「显式 LSP」当卖点,不是因为它比 Claude Code/Droid 更"先进",而是因为它解决的是「开源、可控、可验证」这个层级的问题。

\textbf{一句话对比:}
\begin{itemize}
  \item Claude Code / Droid:"我有,但你别管"
  \item OpenCode:"我有,而且你可以看、可以控、可以换"
\end{itemize}

\subsubsection{为什么 Claude Code/Droid "不需要说 LSP"}

对闭源/强系统来说,LSP 是"理所当然的内脏器官"。在这些工具中,LSP $\approx$ AST/symbol graph/semantic index,是 agent 的基础设施,用户无权、也无必要干预。所以它们的叙事是"我理解你的代码",而不是"我用 LSP 理解你的代码"。LSP 在这里不是产品卖点,而是工程前提。

显式暴露 LSP 反而会"降低抽象层级"。如果你还在选 LSP、看诊断,那说明 agent 没把事情做好。所以它们刻意不提 LSP。

\subsubsection{OpenCode 为什么要"高调说 LSP"}

\textbf{1. OpenCode 是「开源 Agent 框架」,不是「封装好的 Agent」}

用户是开发者、架构师、OSS 社区、企业 IT/安全团队。这些人关心:AI 到底"看到了什么"?是否真的理解代码?能不能在我环境里复现?显式 LSP = 可验证的工程承诺。

\textbf{2. 必须对抗一个默认怀疑}

在开源世界:"你说你理解代码?你是不是只是 grep + prompt?"OpenCode 用 "LSP enabled" 明确告诉你:我不是纯文本 agent、我走的是标准语言协议、我能跑在你的本地环境。这是信任构建,而不是炫技。

\textbf{3. 需要和「聊天式 AI」明确切割}

最大的竞争对手不是 Claude Code/Droid,而是"CLI 聊天机器人"、"包了一层 UI 的 ChatGPT"。所以它必须强调"我是工程 agent,不是 chat bot",而 LSP 是最好、最直观的工程信号。

\textbf{4. OpenCode 的"可替换性焦虑"}

OpenCode 的定位是 Model-agnostic Agent Runtime,意味着模型随时可能换,能力不能依赖模型"聪明"。那什么是不随模型变化的确定性能力?LSP。LSP 是本地的、确定的、可测试的,和模型能力解耦。所以 OpenCode 必须把 LSP 抬到台前,作为"系统确定性"的锚点。

\subsection{LSP-first 工作流程}

在大多数 AI coding agent 里,你"不会直接使用 LSP",而是通过"正确的工作方式"让 agent 去用 LSP。

\subsubsection{核心原则}

LSP 不是一个你去点的功能,而是一套你要"逼着 agent 依赖"的事实来源。你要做的不是"操作 LSP",而是:
\begin{itemize}
  \item \textbf{让 agent 必须通过 LSP 才能回答}
  \item \textbf{不允许它靠猜}
\end{itemize}

\subsubsection{四种 LSP-first 行为}

\textbf{1. 要求基于符号而不是文本行动}

错误示例(不触发 LSP):"把这个函数改一下,让它更清晰"

正确示例(强烈触发 LSP):"重构 UserService.createUser,要求:不改变 public signature、更新所有调用点、如果有类型冲突,指出具体文件和行号"

👉 这类要求没有 LSP 几乎做不到。

\textbf{2. 要求 agent 定位定义 / 引用}

"在当前 repo 中:找到 PaymentProcessor 的定义、列出所有直接调用它的地方、判断是否存在循环依赖"

👉 这是标准 LSP query(go to definition / find references)。

\textbf{3. 要求类型级别的保证}

"确认这个改动:不引入新的 TypeScript 类型错误、不破坏 Rust 的 borrow 规则、不改变 Python public API 的 typing"

👉 没有 LSP,这只能靠"感觉";有 LSP,这是确定性检查。

\textbf{4. 让 agent 先分析再改代码}

"在修改任何代码前:用语言服务器分析相关模块、总结当前结构和依赖、再给出修改方案"

这一步会把 agent 强制拉到 LSP → plan → edit 的路径。

\subsubsection{使用 LSP 的关键}

不在于「操作它」,而在于「提出只有 LSP 才能完成的任务」。

\begin{itemize}
  \item 永远用符号名,不用「这个函数」
  \item 要求更新所有引用
  \item 要求类型不变/API 不变
  \item 先分析,再实现
  \item 允许 agent 说「不确定,需要确认」
\end{itemize}

\subsection{LSP-first Prompt 模板}

\begin{verbatim}
使用 LSP 工具查询以下信息:
1. 获取当前光标位置的定义 (goToDefinition)
2. 查找所有引用点 (findReferences)
3. 显示文档和类型信息 (hover)
基于 LSP 返回的准确语义,分析代码结构并给出建议。
\end{verbatim}

\subsection{终极总结}

LSP 是让 AI 从"猜代码"升级为"理解代码"的基础设施;OpenCode 显式拥抱 LSP,Claude Code 隐式使用,Droid 将其视为理所当然。LSP 在 Claude Code/Droid 中是"默认前提",在 OpenCode 中是"公开承诺";前者追求抽象完成度,后者追求可控与可验证。

\section{Spec 的存储与连续性}

\subsection{Spec 的三种载体层}

\begin{enumerate}
  \item \textbf{Repo-level spec}:Git 仓库中的 Markdown/YAML 文件(最稳定)
  \item \textbf{Project-level spec}:Droid 内部 spec state(系统态)
  \item \textbf{Session/View state}:当前执行上下文(不可追溯)
\end{enumerate}

\subsection{判断 Spec 是否在仓库}

\textbf{三问判定法}:
\begin{enumerate}
  \item Spec 能否被另一个账号完整恢复?
  \item Spec 是否存在可版本化、可 diff 的实体?
  \item Agent 引用 spec 时是"引用文件"还是"模糊转述"?
\end{enumerate}

\subsection{防止 Spec 失效的条件}

把 spec 放进 repo 只是"存储正确",必须同时满足:
\begin{enumerate}
  \item 明确宣告 spec = 唯一真理源
  \item 强制"引用义务":任何实现必须引用 spec 条目
  \item spec 可被 diff/review
  \item 冲突时显式处理,不自动调和
\end{enumerate}

\section{spec / plan / tasks 与 Git 分支的对应关系}

\subsection{核心对应表}

\begin{table}[h]
\centering
\begin{tabular}{lllll}
\hline
\textbf{层级} & \textbf{文件} & \textbf{位置} & \textbf{语义归属} & \textbf{Git 分支} \\
\hline
spec & spec.md & .specify/specs/xxx/ & Feature & feat/\emph{} 分支 \\
plan & plan.md & .specify/specs/xxx/ & Feature & feat/\emph{} 分支 \\
tasks & tasks.md & .specify/specs/xxx/ & Feature & feat/\emph{} 分支 \\
code & src/... & 项目源码目录 & Feature & feat/\emph{} 分支 \\
constitution & constitution.md & .specify/memory/ & 项目 & 所有分支共享 \\
\hline
\end{tabular}
\end{table}

\subsection{时间轴视角}

\begin{verbatim}
(main)
│
├── git checkout -b feat/todo-cli
│
(feat/todo-cli)
│
├── spec.md      ← 决策:要做什么
├── plan.md      ← 决策:怎么做
├── tasks.md     ← 决策:分几步做
│
├── /speckit.implement
│
├── code changes ← 执行结果
│
├── PR review    ← 验证:是否按 spec
│
└── merge → main
\end{verbatim}

\section{.specify/ 目录定位}

\subsection{本质澄清}

.specify/ 的本质是:LLM 的工作目录/私有工具状态/临时输入输出。

它更像 .idea/、.vscode/、.cursor/:
\begin{itemize}
  \item ❌ 不稳定
  \item ❌ 工具相关
  \item ❌ 不保证长期语义
  \item ✅ 可随工具版本变化
\end{itemize}

\subsection{不应放的内容}

形态 2(协作态 spec)和形态 3(契约态 spec),都不应该放在 .specify/ 里,而应该放在项目可见、可 review 的目录(./docs、./design、./adr 等)。

\textbf{原因}:
\begin{enumerate}
  \item .specify/ 是 specify 工具的实现细节,工程 spec 必须工具无关
  \item 新工程师会看 README/docs/adr/,不会默认翻 .specify/
  \item review/diff/blame 语义错误
\end{enumerate}

\subsection{正确位置}

\begin{itemize}
  \item 工具私有状态 → .specify/、.claude/
  \item 临时工作 spec → 不进 git、对话、note
  \item 协作态 spec → docs/wip/、specs/(PR 时决定去留)
  \item 契约态 spec → docs/adr/、docs/api/、必须进 main
\end{itemize}

\section{恢复与连续性}

\subsection{/resume 的本质}

/resume 的意思是:让 Claude 尝试「继续上一次中断的会话状态」,而不是重新理解项目或重新加载工程事实。它是会话级恢复,不是工程级恢复。

/resume 只可能恢复:上一次对话中的任务目标、讨论上下文、未完成的步骤;Claude 在当前会话窗口里的推理轨迹、临时计划、工作节奏。

/resume 不能:恢复本地的 .specify/、恢复 $\sim$/.factory/specs/、重新读懂你没 commit 的 WIP spec、替代 ADR/README/CLAUDE.md、跨账号/跨机器保证一致。

\subsection{Droid /session}

droid 的 /session 本质和 Claude 的 /resume 一样:都是会话级连续性,不是工程级状态恢复。

\subsection{Spec 恢复依赖 Git}

spec 的「恢复」本质上依赖的是 Git(以及 repo 内文档),而不是 /resume。只有两种来源:
\begin{itemize}
    \item Git repo(代码 + docs)
    \item 被 commit 的 ADR/WIP spec/README
\end{itemize}

除此之外,都是:临时缓存、会话残影、工具状态。

工程上更稳的替代方案:WIP spec(docs/wip)、ADR(docs/adr)、清晰的 commit。这些东西:不靠会话、不靠模型、不靠工具。


\section{工程实践总结}

\subsection{Spec Coding 的核心原则}

\begin{enumerate}
  \item 不迷信 skills、不迷信 agent
  \item spec 只服务确定性
  \item git 是最终事实源
  \item 所有不确定性留在分支,所有确定性进 main
\end{enumerate}

\subsection{终极心法}

临时 spec 是工作记忆,main 只保留长期记忆;squash merge 的意义,不仅是压缩 commit,也是压缩思考噪音。

\textbf{安全原则}:"所有不确定性,都留在分支里;所有确定性,才进 main。"

\subsection{何时需要 ADR 和 WIP Spec}

在 droid coding 的过程中:你不需要「为了用 AI」而刻意维持 ADR 或 WIP spec。你只需要在「工程本身需要它们」的时候才写。

\subsubsection{完全不需要的情况}

满足以下条件之一:
\begin{itemize}
    \item 单人项目/小项目:你是唯一维护者、变更半天内完成、上下文都在你脑中
    \item feature 局部、无长期影响:不影响架构、不引入新概念、不改变数据模型
    \item droid coding 的「短反馈循环」:快速试、快速改、squash 掉
\end{itemize}

\subsubsection{需要 WIP Spec 的情况}

满足以下任意一条:
\begin{itemize}
    \item feature 超过 1–2 天
    \item 你需要「明天继续」
    \item 你发现自己反复跟 AI 解释同一件事
    \item 可能有人中途接手
\end{itemize}

写一个极简 WIP spec,不超过一页。

\subsubsection{必须写 ADR 的情况}

只有在这些情况下:
\begin{itemize}
    \item 做了不可逆或高成本决策
    \item 明确否掉了「看似合理」的方案
    \item 这个决定未来会被反复质疑
\end{itemize}

这是给未来人(包括你)用的,不是给 AI。

\subsubsection{交接风险判断}

只要满足以下任意一条:可能有第二个人接手、你自己可能换账号/换机器/中断一段时间。那么:工程级信息就不能只存在于 droid 的私有目录里,必须至少有 WIP spec;涉及不可逆决策时,必须有 ADR。

30 秒判断法:在写代码前或 merge 前,问自己:「如果我明天请假两周,别人能接手吗?」
\begin{itemize}
    \item 能 $\to$ OK,不用写
    \item 不能 $\to$ 至少写 WIP spec
\end{itemize}

\vspace{1em}
\normalsize
