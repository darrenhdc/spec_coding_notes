% chapter_14.tex
\small
\begin{verbatim}
.specify/memory/constitution.md
\end{verbatim}

这是 Claude 长期遵守的规则文件。

\section{Spec-Driven Demo: 最小但完整的 Todo CLI}

\subsection{前置条件}

\begin{itemize}
    \item 已安装 specify-cli
    \item 已安装 Claude Code(或支持 slash command 的 Claude)
    \item 当前目录是未初始化的项目
\end{itemize}

\subsection{Step 0: 初始化 spec-kit}

\begin{verbatim}
mkdir todo-demo
cd todo-demo
specify init . --ai claude
\end{verbatim}

生成的目录结构:

\begin{verbatim}
todo-demo/
├── .specify/
│   ├── memory/
│   │   └── constitution.md
│   ├── scripts/
│   └── templates/
├── CLAUDE.md
└── README.md
\end{verbatim}

\subsection{Step 1: 启动 Claude Code}

在项目 root 目录启动 Claude Code,确认能看到 /speckit.* 命令。

\subsection{Step 2: 写项目宪法}

\begin{verbatim}
/speckit.constitution
Create simple, pragmatic principles:
- Prefer clarity over abstraction
- Keep dependencies minimal
- Write code that is easy to read and modify
- Use basic Python standard library only
- Favor small functions and explicit logic
\end{verbatim}

生成 \texttt{.specify/memory/constitution.md}。

\subsection{Step 3: 写功能 spec(最关键)}

\begin{verbatim}
/speckit.specify
Build a simple command-line Todo application.
Users can:
- Add a todo item with text
- List all todo items
- Mark a todo item as completed

Todos are stored locally on disk.
This is a single-user tool and does not require authentication.
The focus is correctness and simplicity, not performance.
\end{verbatim}

生成 \texttt{.specify/specs/001-todo-cli/spec.md}。此时还没有技术细节。

\subsection{Step 4: 生成技术方案}

\begin{verbatim}
/speckit.plan
Implement this as a Python CLI tool.
Use argparse for command parsing.
Persist todos in a local JSON file.
The application should be runnable via `python todo.py`.
\end{verbatim}

生成 plan.md(可能还有 data-model.md / research.md)。

\subsection{Step 5: 生成任务拆分}

\begin{verbatim}
/speckit.tasks
\end{verbatim}

生成 tasks.md,包含可执行步骤列表。

\subsection{Step 6: 执行实现}

\begin{verbatim}
/speckit.implement
\end{verbatim}

Claude 按 tasks.md 顺序执行,创建 todo.py 并写入代码。

最终目录结构:

\begin{verbatim}
todo-demo/
├── todo.py
├── todos.json
└── .specify/
\end{verbatim}

手动测试:

\begin{verbatim}
python todo.py add "Buy milk"
python todo.py list
python todo.py done 1
\end{verbatim}

\subsection{核心体感}

\begin{itemize}
    \item Claude 没有跳步——每一步都有文件产物
    \item 可随时打断/修改——改 spec → 重新 plan → 重新 tasks
    \item 这些不是 Claude 的"记忆"——全在 repo 里,换 agent 也能继续
\end{itemize}
