% chapter_14.tex
\small
\begin{verbatim}
.specify/memory/constitution.md
\end{verbatim}

这是 Claude 长期遵守的规则文件。

\section{Spec-Driven Demo: 最小但完整的 Todo CLI}

\subsection{前置条件}

\begin{itemize}
    \item 已安装 specify-cli
    \item 已安装 Claude Code(或支持 slash command 的 Claude)
    \item 当前目录是未初始化的项目
\end{itemize}

\subsection{Step 0: 初始化 spec-kit}

\begin{verbatim}
mkdir todo-demo
cd todo-demo
specify init . --ai claude
\end{verbatim}

生成的目录结构:

\begin{verbatim}
todo-demo/
├── .specify/
│   ├── memory/
│   │   └── constitution.md
│   ├── scripts/
│   └── templates/
├── CLAUDE.md
└── README.md
\end{verbatim}

\subsection{Step 1: 启动 Claude Code}

在项目 root 目录启动 Claude Code,确认能看到 /speckit.* 命令。

\subsection{Step 2: 写项目宪法}

\begin{verbatim}
/speckit.constitution
Create simple, pragmatic principles:
- Prefer clarity over abstraction
- Keep dependencies minimal
- Write code that is easy to read and modify
- Use basic Python standard library only
- Favor small functions and explicit logic
\end{verbatim}

生成 \texttt{.specify/memory/constitution.md}。

\subsection{Step 3: 写功能 spec(最关键)}

\begin{verbatim}
/speckit.specify
Build a simple command-line Todo application.
Users can:
- Add a todo item with text
- List all todo items
- Mark a todo item as completed

Todos are stored locally on disk.
This is a single-user tool and does not require authentication.
The focus is correctness and simplicity, not performance.
\end{verbatim}

生成 \texttt{.specify/specs/001-todo-cli/spec.md}。此时还没有技术细节。

\subsection{Step 4: 生成技术方案}

\begin{verbatim}
/speckit.plan
Implement this as a Python CLI tool.
Use argparse for command parsing.
Persist todos in a local JSON file.
The application should be runnable via `python todo.py`.
\end{verbatim}

生成 plan.md(可能还有 data-model.md / research.md)。

\subsection{Step 5: 生成任务拆分}

\begin{verbatim}
/speckit.tasks
\end{verbatim}

生成 tasks.md,包含可执行步骤列表。

\subsection{Step 6: 执行实现}

\begin{verbatim}
/speckit.implement
\end{verbatim}

Claude 按 tasks.md 顺序执行,创建 todo.py 并写入代码。

最终目录结构:

\begin{verbatim}
todo-demo/
├── todo.py
├── todos.json
└── .specify/
\end{verbatim}

手动测试:

\begin{verbatim}
python todo.py add "Buy milk"
python todo.py list
python todo.py done 1
\end{verbatim}

\subsection{核心体感}

\begin{itemize}
    \item Claude 没有跳步——每一步都有文件产物
    \item 可随时打断/修改——改 spec → 重新 plan → 重新 tasks
    \item 这些不是 Claude 的"记忆"——全在 repo 里,换 agent 也能继续
\end{itemize}

\section{交互范式转移:Vibe Coding vs Spec-Driven}

\subsection{旧范式:Vibe Coding}

典型流程:

\begin{verbatim}
你:我们来做个 X 吧
Claude:可以,方案 A / B / C
你:选 B
Claude:开始写代码
你:等等,这里不太对
Claude:改
你:顺便加个功能
Claude:好
\end{verbatim}

特点:对话即执行,决策和实现混在一起,每句话都可能改变代码走向。

常见问题:上下文漂移,Claude 推翻自己,改到后面不敢动了。

\subsection{新范式:Spec-Driven + Claude}

\begin{verbatim}
你 ⇄ Claude:讨论 spec(what / why)
你 ⇄ Claude:讨论 plan(how)
你:确认 tasks
你:/speckit.implement
Claude:按计划执行
\end{verbatim}

关键变化:对话阶段 $\neq$ 执行阶段。对话时不写代码、不跑命令、不动 repo,只做产品决策、工程决策和约束声明。

\subsection{角色变化}

从「即时批准者」到「规范制定者」:

\begin{itemize}
    \item 以前:approve 某一行代码、某个方案
    \item 现在:定义"允许出现的方案空间"、"禁止的行为"、"成功的标准"
\end{itemize}

这是工程负责人 / tech lead 的角色。

\subsection{/speckit.implement 的本质}

相当于一次"批量、受控、可回滚的自动 commit":

\begin{itemize}
    \item 之前:Claude 只能"思考 + 写文档"
    \item 之后:Claude 被允许执行
    \item 执行范围被 tasks.md 严格限制,执行顺序已确定,不再临时加戏
\end{itemize}

\subsection{范式对比}

\begin{center}
\begin{tabular}{lcc}
\hline
维度 & Vibe Coding & Spec-Driven \\
\hline
approve 的对象 & 某个即时建议 & 整体设计 \\
approve 的粒度 & 很细 & 很粗 \\
是否随时写代码 & \checkmark & $\times$ \\
是否允许中途漂移 & \checkmark & $\times$ \\
执行是否可预测 & $\times$ & \checkmark \\
\hline
\end{tabular}
\end{center}

以前在批准"下一步",现在在批准"全部路径"。

\subsection{效率权衡}

\begin{itemize}
    \item 短期看:Spec-Driven 更慢
    \item 长期看:Spec-Driven 更快(少返工、少"改到不敢改"、更容易中断/续上、可换 agent)
\end{itemize}

思考成本前移。

\subsection{适用场景}

不适合 Spec-Driven:快速试探、原型 demo、探索性问题、"我自己都不知道要啥"。

适合 Spec-Driven:已知道目标、想要稳妥交付、项目会活很久、会被多次修改。

\subsection{范式切换标志}

当你发现自己不再纠结"Claude 怎么写代码",而是纠结"我这个 spec 写得准不准",说明已完成范式切换。

\section{Git 与 Spec-Driven 的集成}

\subsection{/speckit.implement 默认不自动 commit}

默认行为:

\begin{itemize}
    \item \checkmark 读取 Git 仓库状态
    \item \checkmark 在当前分支上工作
    \item \checkmark 创建 / 修改文件
    \item $\times$ 不自动 git commit
    \item $\times$ 不自动 git push
\end{itemize}

Git 的控制权始终在你手里。

\subsection{spec-kit 如何使用 Git}

Git 主要用于定位 feature,而不是提交:

\begin{enumerate}
    \item 生成 feature 目录名(如 \texttt{.specify/specs/001-todo-cli/})
    \item 判断"当前在做哪个 feature"(假设一个 feature = 一个 Git 分支)
    \item 不是用来自动提交(commit 是"工程决策",不是 AI 决策)
\end{enumerate}

\subsection{可能看到 commit 的情况}

\begin{enumerate}
    \item 你明确要求 Claude 帮你提交
    \item 使用高度自动化 agent(Droid / 类 Droid)
    \item 教程中说 "You can ask Claude to create a PR"(ask 不是自动)
\end{enumerate}

\subsection{与现有 .git 不冲突}

spec-kit 不修改 .git/、不改 config、不强制 branch 策略,只读 Git 信息并组织 spec 文件。可用已有仓库、monorepo、非标准分支名。

\subsection{推荐 workflow}

\begin{verbatim}
git checkout -b feat/todo-cli

# spec / plan / tasks / implement(Claude 干活)

git diff  # review
git commit -am "feat: add simple todo CLI"
\end{verbatim}

\section{回滚策略}

\subsection{回滚方式}

\begin{enumerate}
    \item \textbf{Git 原生回滚}(最常用):\texttt{git restore .} 或 \texttt{git restore todo.py}
    \item \textbf{Feature 分支即"天然回滚点"}:\texttt{git branch -D feat/todo-cli}
    \item \textbf{spec / plan 本身就是"可回滚状态"}:修改 spec / plan 后重新 implement
    \item \textbf{Checkpoint commit}:手动打 checkpoint,失败时 \texttt{git reset --hard HEAD}
    \item \textbf{git stash}:临时避险
\end{enumerate}

\subsection{设计哲学}

spec-kit 不需要自动 commit,因为:

\begin{itemize}
    \item commit 是"你对世界负责的边界",implement 只是执行
    \item 如果 AI 自动 commit:commit message 谁负责?哪些文件算"完成"?哪些只是中间状态?
\end{itemize}

这些刻意留给人类。

\subsection{范式差异}

\begin{itemize}
    \item Vibe coding:回滚 = 撤代码
    \item Spec-driven:回滚 = 重跑执行
\end{itemize}

只要保留 spec.md、plan.md、tasks.md,就永远可以再来一次。spec 文件比代码更重要。

\section{Spec 文档的管理}

\subsection{必须提交 .specify/}

.specify/ 下的 spec / plan / tasks 文档应该被提交到 Git,不是临时文件或草稿。

\subsection{理由}

\begin{enumerate}
    \item \textbf{可回滚决策}:不提交只能回滚代码,不知道当初为什么这么写
    \item \textbf{可审计性}:Git 历史显示设计在前、实现随后
    \item \textbf{PR review 价值}:reviewer 先看 spec.md、plan.md,再看代码
\end{enumerate}

\subsection{常见错误做法}

\begin{itemize}
    \item[$\times$] .specify/ 加进 .gitignore
    \item[$\times$] 只 commit 代码,不 commit spec
    \item[$\times$] spec 只存在 Claude 对话里
\end{itemize}

\subsection{标准提交节奏}

\begin{verbatim}
git checkout -b feat/todo-cli

# 1️⃣ spec / plan / tasks
git add .specify
git commit -m "docs(todo): define spec and plan"

# 2️⃣ implement
/speckit.implement

# 3️⃣ review & adjust
git diff

# 4️⃣ final commit
git add todo.py
git commit -m "feat(todo): implement CLI"

# 可选:rebase -i main squash docs + feat
\end{verbatim}

\subsection{判断原则}

只要一个文件影响 Claude 的行为,它就必须进 Git。.specify/ 里的文件 100\% 符合这条。

\section{Feature = Git 分支}

\subsection{不是 Git 常识,是工程约定}

"Feature = Git 分支"不是 Git 的内建规则,而是工作流约定。Git 只认识 commit 和 branch,工程才认识 feature。

Git 的真正事实:branch 只是一个指向 commit 的指针。Git 不关心 feature、bug、release、task、ticket。

\subsection{为什么成为事实标准}

因为人类需要:

\begin{enumerate}
    \item 最简单的"隔离单位":不污染 main、可随时丢弃、可并行工作
    \item 天然回滚边界:\texttt{git branch -D feat/todo-cli} = "这个 feature 从未发生过"
    \item 与 PR / Review 完美对齐:一个分支 $\to$ 一个 PR,一个 PR $\to$ 一个 feature
    \item 与 spec-driven 极度契合:spec = 定义,plan = 路线,implement = 执行
\end{enumerate}

\subsection{spec-kit 的假设}

spec-kit 假设(assume)这个约定,不是强制(enforce)。需要"稳定的 feature 边界",Git 分支是最便宜、最成熟的边界。

\subsection{不推荐的用法}

\begin{itemize}
    \item[$\times$] 所有功能都在 main 上做:spec 目录不断叠加、implement 相互干扰、回滚成本爆炸
    \item[$\times$] 一个分支干多个 feature:spec 和代码无法对应、PR review 失焦
    \item[$\checkmark$] 高级:一个 branch + feature flag(适合超大型团队、Trunk-based development)
\end{itemize}

\section{Pull Request (PR)}

\subsection{PR 不是 Git 概念}

PR 是协作平台(GitHub、GitLab、Bitbucket)提供的"变更提议机制"。Git 本身只有 commit、branch、merge,没有 PR。

\subsection{PR 的定义}

Pull Request = "我在一个分支上完成了一组有明确意图的变更,现在我请求你:理解它、评审它、决定是否把它并入主线"。

\subsection{PR 的真正价值}

\begin{itemize}
    \item 一个"意图边界":这次改动想干嘛?
    \item 一个"讨论空间":为什么这么设计?有没有更简单的?(不污染 commit 历史)
    \item 一个"安全闸门":看 diff、跑 CI、看 spec / plan、决定 merge / reject
\end{itemize}

merge 不是 PR 的重点,review 才是。

\subsection{PR 的心智模型}

PR = 代码的"说明书 + 审批单"

\begin{itemize}
    \item commit = 你动了什么
    \item diff = 你具体怎么动的
    \item PR 描述 = 你为什么要动
    \item review = 别人帮你找问题
    \item merge = 批准生效
\end{itemize}

\subsection{PR 生命周期}

\begin{enumerate}
    \item 建分支:\texttt{git checkout -b feat/todo-cli}
    \item 写代码 + commit:\texttt{git commit -m "feat: add todo CLI"}
    \item push 分支:\texttt{git push origin feat/todo-cli}
    \item 打开 PR:"我在 feat/todo-cli 上完成了一件事,想把它合到 main,请你帮我看看"
    \item Review:别人看 diff、提 comment、要你改、CI 跑
    \item Merge 或拒绝:merge $\to$ main 更新,close PR $\to$ 这件事作废
\end{enumerate}

\subsection{PR 在 Spec-Driven 中的位置}

PR = "spec 是否被正确实现"的最终检查点。健康的 PR 应该让 reviewer 能看到:spec.md、plan.md、实现代码。Reviewer 不是猜你想干嘛,而是验证你有没有按 spec 做。

\subsection{Feature + PR 的对齐}

\begin{center}
\begin{tabular}{ll}
\hline
概念 & 对应 \\
\hline
Feature 分支 & Feature 意图 \\
PR & Feature 讨论 \\
PR 评论 & Feature 审批 \\
PR merge & Feature 批准 \\
\hline
\end{tabular}
\end{center}

这是工程上极度优雅的一种对齐。
