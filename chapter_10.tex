\chapter{Spec Coding 与 Vibe Coding 的范式转移}

{\small

\section{Spec Coding 与 Vibe Coding 的本质区别}

Vibe Coding 依赖直觉、经验和模糊判断进行快速迭代。Spec Coding 则通过明确的规范文档来驱动开发,将决策前移。

\textbf{核心差异}:
\begin{itemize}
  \item Vibe Coding:对话即执行,决策和实现混在一起,容易上下文漂移
  \item Spec Coding:对话阶段与执行阶段分离,spec → plan → implement 的清晰流程
\end{itemize}

Spec Coding 的交互范式是:用户主要与 Claude"共写规范与决策",而不是"共写代码";真正写代码的时刻被延迟、被约束、被批量执行。

\section{Droid 与 Spec-Driven Development}

Droid 是一种端到端的自动 agent,它已经内置了"Spec → Plan → Execute"的思想。Droid 不需要 spec-kit,因为它的执行能力已经包含了这些阶段。

但 Droid 的产物是"机器状态",而非"人读的 spec"。如果需要:
\begin{itemize}
  \item audit 和 PR review
  \item 长期知识沉淀
  \item 多 agent 协作
\end{itemize}

则仍然需要显式的 spec / plan 文档作为共享真相源。

\section{Claude Skills 的本质}

Claude Skills 不是"Claude 的内在能力",而是外部注册给 Claude 的"可调用模块"。Skills 的价值在于:
\begin{itemize}
  \item 把个人经验沉淀为可复用模式
  \item 让下一次 vibe 更准确
\end{itemize}

但需要注意:
\begin{itemize}
  \item Skills 高度定制化、非官方、不可验证
  \item 很多"Skills 焦虑"是营销叙事,而非工程必需品
  \item Spec Coding 已经将"零散 skills"系统化了
\end{itemize}

\textbf{判断原则}:这段文字是为了"这次能做对"(spec),还是为了"下次能更快 vibe 对"(skills)?前者用 spec,后者才考虑 skills。

\section{Git 工作流与 Spec 管理}

\subsection{Feature = Git 分支}

"Feature = Git 分支"不是 Git 的常识,而是被广泛采用的工程约定。Git 只认识 commit 和 branch,工程才认识 feature。

spec-kit 假设你使用"一个 feature = 一个 Git 分支"的模型,因为这提供了:
\begin{itemize}
  \item 最简单的隔离单位
  \item 天然的回滚边界
  \item 与 PR / Review 的完美对齐
\end{itemize}

\subsection{Spec 文件的物理位置与语义归属}

spec 文件:
\begin{itemize}
  \item \textbf{物理位置}:永远在项目根目录的 .specify/ 下
  \item \textbf{语义归属}:属于某一个 feature(通常是 feat 分支)
  \item \textbf{生命周期}:通过 Git 分支管理
\end{itemize}

spec / plan / tasks 都应该被提交到 Git,因为它们是 feature 的"设计合同"。

\subsection{PR(Pull Request)的本质}

PR 不是 Git 命令,而是协作平台提供的"变更提议机制"。PR 的真正价值不在 merge,而在 review:

\begin{itemize}
  \item 一个"意图边界"(这次改动想干嘛)
  \item 一个"讨论空间"(问为什么、提建议)
  \item 一个"安全闸门"(看 diff、跑 CI、决定 merge)
\end{itemize}

在 spec-driven 工作流里,PR 是"spec 是否被正确实现"的最终检查点。

\subsection{Merge 策略}

\textbf{三种 merge 策略}:

\begin{enumerate}
  \item \textbf{普通 merge}:保留完整分支历史("毛毛虫"形状)
  \item \textbf{Squash merge}:把多个 commit 压扁为一个(推荐,最常见)
  \item \textbf{Rebase + merge}:改写历史后线性合并
\end{enumerate}

在现代 PR-based、spec-driven 工作流里,squash merge 是最常见的选择。因为:
\begin{itemize}
  \item main 历史清晰可读
  \item spec / plan 已经记录了过程
  \item main 更像"产品演进日志"而非"开发录像"
\end{itemize}

\subsection{Rebase 的本质}

Rebase 的体感像"同步代码",但本质是"重写你这段提交的历史位置"。正确理解:

\begin{quote}
我在 feature 分支上,通过 rebase 把自己的提交,重新基于最新的 main 应用一遍,让历史看起来像是从最新 main 开始开发的。
\end{quote}

rebase 只在你自己的 feature 分支上是安全的;不要在 main / 已共享分支上 rebase。

\subsection{Spec 与 Commit 的关系}

\texttt{/speckit.implement} 执行完成后:
\begin{itemize}
  \item 不会自动 commit
  \item 不会自动 merge
  \item 代码只存在于 feat 分支的工作区
\end{itemize}

commit 是"责任边界",必须由人类决定。推荐节奏:
\begin{enumerate}
  \item spec / plan / tasks → commit
  \item implement → review → commit
  \item open PR → merge
\end{enumerate}

\subsection{Spec 的生命周期}

Spec 有三种形态:

\begin{enumerate}
  \item \textbf{工作态 spec}(不进 git):本地笔记、对话、临时 markdown
  \item \textbf{协作态 spec}(进 git,有生命周期):docs/wip/、specs/
  \item \textbf{契约态 spec}(必须长期存在):docs/adr/、API 文档、README
\end{enumerate}

判断标准:"如果半年后我只看代码和测试,还需要知道这份 spec 吗?"

\section{项目文档层级体系}

完整的文档优先级:

\begin{enumerate}
  \item \textbf{ADR(docs/adr/*)}:不可违反的架构决策
  \item \textbf{README.md}:项目事实(是什么、怎么用)
  \item \textbf{WIP spec / design docs}:临时协作文档
  \item \textbf{CLAUDE.md}:AI 行为约束
  \item \textbf{Tool-local specs}(.specify/、droid specs):任务级工作记忆
\end{enumerate}

\subsection{ADR(Architecture Decision Record)}

ADR 记录"为什么我们在架构上做了这个决定":
\begin{itemize}
  \item Context:发生了什么问题?
  \item Decision:我们选了什么方案?
  \item Alternatives:还考虑过哪些方案?为什么没选?
  \item Consequences:这个决定的好处和代价是什么?
\end{itemize}

ADR 是给未来的你、同事、以及 AI 看的,防止重复走歪路。

\subsection{Claude.md 的职责}

CLAUDE.md 是 AI 的"工作指令书",记录:
\begin{itemize}
  \item spec 优先级
  \item 禁止事项
  \item 代码风格
  \item agent 行为边界
\end{itemize}

文档优先级规则应该定义在 CLAUDE.md 里,因为:
\begin{itemize}
  \item 它面向 AI
  \item 它约束行为
  \item 它不承载工程事实本身
\end{itemize}

CLAUDE.md 可以引用 ADR,但不应该替代 ADR。

\subsection{工具目录的定位}

以下目录都是"开发工具或运行环境的私有状态":
\begin{itemize}
  \item .idea/:JetBrains IDE 状态
  \item .vscode/:VS Code 配置
  \item .cursor/:Cursor AI 工具状态
  \item .venv/:Python 虚拟环境
  \item .specify/:spec-kit 工作记忆
  \item \texttt{\textasciitilde{}/.factory/specs/}:Droid 工作记忆
\end{itemize}

它们的共同点:可再生、不可审计、不应成为"项目真相"。

判断原则:"如果我换 IDE / 换 AI / 换机器,这个东西还应该存在吗?"

\section{Spec-Driven 工作流最佳实践}

\subsection{正确的开发流程}

\begin{enumerate}
  \item git checkout -b feat/xxx
  \item spec / plan / tasks(可 commit)
  \item implement
  \item review(git diff / PR)
  \item commit implementation
  \item squash merge
\end{enumerate}

这套流程:
\begin{itemize}
  \item 可控、可复现、可扩展
  \item 抗模型变化
  \item 所有不确定性留在分支里
  \item 所有确定性才进 main
\end{itemize}

\subsection{Spec 冲突的处理}

当 .specify/ / droid specs 与协作态 / 契约态 spec 冲突时:

\textbf{优先级规则}:契约态 > 协作态 > 工具态

这条规则必须显式告诉 AI。当 AI 发现冲突时,应该:
\begin{itemize}
  \item 明确指出冲突文件和冲突点
  \item 提供建议方案
  \item 等待人类确认
\end{itemize}

不要让 AI "协调 spec 冲突",要让 AI "暴露 spec 冲突"。

\subsection{/resume 的本质}

/resume 恢复的是"对话状态",不是"项目状态"。它不能:
\begin{itemize}
  \item 恢复 .specify/
  \item 恢复 droid specs
  \item 重新读懂未 commit 的 WIP spec
  \item 替代 ADR / README / CLAUDE.md
\end{itemize}

适合用 /resume 的场景:断线、页面刷新、模型超时。不适合:隔了一天、换了机器、项目状态有变化。

\section{核心理念总结}

\begin{quote}
  spec 是工程契约,skills 是个人经验;工程方法会成为未来,个人技巧只会周期性被营销。
\end{quote}

\begin{quote}
  .specify/ 是给工具用的,./docs 是给工程师用的;只要是"未来的人要看的 spec",就绝不应该藏在 .specify/ 里。
\end{quote}

\begin{quote}
  临时 spec 是工作记忆,main 只保留长期记忆;squash merge 的意义,不仅是压缩 commit,也是压缩思考噪音。
\end{quote}

} % end small
