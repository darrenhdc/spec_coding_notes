% chapter_04.tex - Spec Coding 工程实践与工具对比
\small

\chapter{Spec Coding 工程实践与工具对比}

\section{Skills 文档与 Spec 的关系}

传统的「skills」本质上只是「为了让自己或别人搞清楚这次要怎么做,而写的临时解释」。在使用 droid 做 spec-coding 的前提下,绝大多数为单个 feature 服务的 skills 文档都可以不写。

\begin{description}
  \item[Skills] vibe 的长期结晶。跨 feature、长期有效的经验总结,如「我们项目里如何做权限控制」、Onboarding 给新人的指南、一次重要的技术复盘(post-mortem)。
  \item[Spec] 一次性工程决策说明,服务于实现,而不是长期规范。
\end{description}

判断标准:「如果半年后我或别人看到 main,只看代码和测试,还需要知道这份 spec 吗?」需要则 commit,不需要则不 commit。

\section{上下文管理与记忆}

\subsection{Context 与 Memory 的区别}

\begin{description}
  \item[Context (上下文)] 当前模型能「看到」的所有 token,包括对话历史、系统提示、/compact 摘要。特点:不可查看、不可版本化、不可靠、适合短期推理。Context 是「工作记忆」,不是资产。
  \item[Memory (记忆)] 你希望模型长期、不漂移记住的东西。特点:必须显式存在(文件/DB)、可编辑、可 diff、可回滚。Memory 是「工程真理的一部分」。
  \item[Spec (规范)] Memory 的「强约束子集」,用来定义不变量、边界条件、行为契约。Spec = 高强度 Memory。
\end{description}

\subsection{Memory 管理等于上下文管理吗?}

不等价。上下文管理只能做三件事:控制对话长度、合理使用 /compact、频繁做理解验证。你无法真正管理 context。

Memory 管理能做的事:定义结构、持久化、明确优先级、强制 agent 读取。这是你能真正掌控的部分。

\subsection{Claude Code 下的可行 Memory 管理体系}

\begin{verbatim}
/docs
├── SPEC.md            ← 不变量 / 合同(极少改)
├── ARCH.md            ← 架构决策
├── CURRENT_STATE.md   ← 当前阶段状态(高频改,/compact 后必读)
└── DECISIONS.md       ← 防止反复争论
\end{verbatim}

每次新对话 / /compact 后,都明确告诉 Claude:「当对话历史、直觉判断与 docs/ 中的内容冲突时,以 SPEC.md → ARCH.md → CURRENT\_STATE.md 的顺序为准。」


\section{AI 工具对比}

\subsection{Claude Code vs Droid vs OpenCode vs Kiro}

\subsubsection{Claude Code}
\begin{itemize}
  \item \textbf{定位}:增强型开发者工具(copilot++)
  \item \textbf{Plan Mode}:把「vibe → spec 的中间态」产品化了。在动手写代码之前,强制模型显式化「它理解的 spec」。
  \item \textbf{/context}:不可查看。没有 /context、/show-context、/dump-context 之类的命令。
  \item \textbf{LSP 支持}:隐式支持(内部确实使用了 LSP/AST/repo analysis),但对用户完全不可见、不可配置。
  \item \textbf{适用}:单人/小团队、已在使用 Claude Code、想要 spec-driven 但不想换平台。
\end{itemize}

\subsubsection{Droid (Factory.ai)}
\begin{itemize}
  \item \textbf{定位}:真正的 spec-coding agent
  \item \textbf{核心特征}:spec 是一等公民、可编辑、可 diff、agent 可以修改 spec、code 必须服从 spec
  \item \textbf{差异}:Claude Code 的 plan 是「建议」,Droid 的 plan 是「执行约束」
  \item \textbf{记忆}:spec 绑定在项目/Repo/Org(不是账号级)
  \item \textbf{适用}:中大型工程、需要 spec 一致性校验、多 agent 自动并行
\end{itemize}

\subsubsection{OpenCode}
\begin{itemize}
  \item \textbf{定位}:开源、模型无关、隐私优先的 AI 编码 Agent 平台
  \item \textbf{特点}:支持 75+ LLM 提供商、显式 LSP enabled、多入口(Terminal/IDE/Desktop)、隐私优先(不存储你的代码)
  \item \textbf{范式}:Context-light + Agent-heavy + External State
  \item \textbf{适用}:需要可替换模型、企业/安全/本地开发友好
\end{itemize}

\subsubsection{Kiro (AWS)}
\begin{itemize}
  \item \textbf{定位}:企业级 agentic IDE
  \item \textbf{三层结构}:Requirements(需求层)、Technical Design(设计层)、Tasks/Hooks(执行层)
  \item \textbf{定位}:spec 是「人维护的工程文档」,而非「agent 维护的系统真理」
  \item \textbf{意义}:第一个把「spec coding」带进主流 IDE 的产品
\end{itemize}

\subsection{Function Calling 与 MCP Tools}

\textbf{函数调用(function calling)}是:AI 说「我要用某个函数,并给你参数」。

\begin{itemize}
  \item AI 不执行代码
  \item AI 只做决定
  \item 执行的是你(或系统)
\end{itemize}

\textbf{MCP (Model Context Protocol)}是:一套「工具怎么描述给 AI 看」的标准。

\begin{itemize}
  \item 规定这个工具叫什么、有哪些参数、参数是什么类型、返回什么结果
  \item 让「工具」看起来像「函数」
\end{itemize}

\section{Spec 在工程中的位置}

\subsection{.specify/ 目录的角色}

.specify/ 的本质是:LLM 的工作目录/私有工具状态/临时输入输出。它更像:
\begin{itemize}
  \item .idea/、.vscode/、.cursor/
  \item 不稳定、工具相关、不保证长期语义、可随工具版本变化
\end{itemize}

\textbf{典型结构}:
\begin{verbatim}
.specify/
├── README.md            #(可选)工具说明
├── context.md           # 当前任务的上下文摘要
├── goal.md              # 本次任务目标
├── constraints.md       # 任务级约束(非工程契约)
├── plan.md              # 执行计划 / 步骤
├── notes.md             # 中间思考 / 草稿
└── outputs/             # 临时生成物(可丢)
\end{verbatim}

.specify/constitution(.md) 是「specify 工具内部的任务级约束说明」,名字听起来像「宪法」,但在工程现实中,它只是「一次性任务说明」。真正的宪法,永远不住在工具目录里。

\subsection{Spec 是否会被当成 Spec?}

有可能。把文件放进 spec/ 目录 $\neq$ 系统就一定会「把它当成 spec」。spec 是否成立,取决于:它是否被「强制作为真理源使用」。

\textbf{成立条件}:
\begin{enumerate}
  \item 明确宣告「spec = 唯一真理源」
  \item 强制「引用义务」(每个实现必须引用 spec 文件和章节)
  \item spec 可被 diff/review
  \item 冲突时「显式处理」,而不是自动融合
\end{enumerate}

\subsection{CLAUDE.md 最小模板}

\begin{verbatim}
# Claude Project Constitution

This file defines non-negotiable rules for all work in this repository.

## 1. Source of Truth
- Files under /spec are the authoritative source of truth.
- Code, tests, conventions, or prior behavior MUST NOT override spec.
- If code conflicts with spec, spec always wins.

## 2. Obligation to Cite Spec
- Every behavior change MUST reference at least one spec file and section.
- References must use a concrete path (e.g. spec/constraints.md#HC-02).
- Changes without explicit spec citation are invalid.

## 3. Handling Missing or Ambiguous Spec
- If the spec does not clearly define a behavior:
  - STOP and ask for clarification.
  - DO NOT infer, guess, or extend behavior.
- Silence in spec does NOT imply permission.

## 4. Conflict Handling
- If spec and implementation diverge:
  - Report the conflict explicitly.
  - Do NOT auto-reconcile or "fix forward".
  - Wait for spec update before proceeding.

## 5. Scope Control
- Do NOT introduce new behavior, APIs, or dependencies
  unless explicitly required by spec.
- Refactoring is allowed only when it preserves all spec-defined behavior.

## 6. Workflow Requirements
Before making changes:
1. Re-read relevant files under /spec.
2. List the constraints that must not change.
3. Confirm the planned change complies with all constraints.
Only proceed after completing these steps.

## 7. Failure Mode
- When in doubt, choose to: Ask, Stop, Report uncertainty
- Never choose to "be helpful" at the cost of violating spec.

End of rules.
\end{verbatim}

一个有效的 CLAUDE.md 不是教 AI 怎么写代码,而是规定:在任何不确定性下,AI 必须选择「服从 spec,而不是取悦用户」。

\section{工程记忆的连续性}

\subsection{Spec 在仓库 vs 在账号}

\textbf{判断标准}:Spec 是否能被「另一个账号」在不依赖你本人的情况下完整恢复?

\begin{itemize}
  \item \textbf{Spec 在项目/Repo/Org}:不会断层。Git 仓库或 Factory.ai 的项目空间(project-level)、有 diff、有时间线、有审计记录。换 Droid 账号:只要有权限、重新 attach 到同一个 project/repo,agent 会读取完整 spec 历史、继续维护一致性。
  \item \textbf{Spec 只在账号私有 workspace}:会断层。账号私有存储、非 repo 绑定,换账号后读不到历史、spec 状态断层。
\end{itemize}

在真正的 spec-coding agent 中,spec 的连续性由「项目」保证,而不是由「账号」保证;如果换账号就断层,那不是你用错了,而是系统还没到那个范式。

\subsection{如何判断 Spec 是否在仓库}

\textbf{三问判定法}:
\begin{enumerate}
  \item Spec 是否能被「另一个账号」在不依赖你本人的情况下完整恢复?
  \item Spec 是否存在一个「可版本化、可 diff 的实体」?
  \item Agent 引用 spec 时,是「引用文件」,还是「模糊转述」?
\end{enumerate}

终极判断标准:「如果我今天离职、账号被删,这个项目是否还能在 spec 不丢失的情况下继续演进?」
\begin{itemize}
  \item 能 $\rightarrow$ spec 在仓库/项目
  \item 不能 $\rightarrow$ spec 在账号
\end{itemize}

\section{关键术语}

\begin{description}
  \item[Drift] 逐步偏离(context/spec/behavior)。context drift(上下文漂移)、spec drift(规格漂移)、behavior drift(行为漂移)。
  \item[Divergence] 架构/实现逐步分叉、偏离原设计。implementation divergence、design divergence。
  \item[Inconsistency] 结果描述。behavioral inconsistency、state inconsistency。
  \item[Hallucination] 编造不存在的事实。模型在「不知道或不确定」的情况下,仍然自信地生成看起来合理、但实际上是错误或不存在的信息。
\end{description}

LSP-first workflows significantly reduce hallucination, while explicit specs are necessary to prevent drift.

Hallucination 是「凭空编造」,Drift 是「逐步跑偏」;前者靠 LSP 和事实源解决,后者靠 spec 和结构化流程解决。

\vspace{1em}
\normalsize
