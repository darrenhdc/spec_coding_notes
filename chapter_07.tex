% chapter_07.tex - Refactored
{\small
\chapter{Memory 管理与 Spec Coding 范式}

\section{Memory 管理与上下文管理}

\subsection{概念辨析:Context vs Memory vs Spec}

\subsubsection{核心结论}

Memory 管理 $\neq$ 上下文管理。上下文是短期注意力窗口,Memory 是长期可控工程资产。

在 Claude Code 中:上下文不可见、不可控、易漂移;Memory 必须由你显式外置,才能稳定、可审计、可继承。

\subsubsection{三个易混淆概念}

\paragraph{Context(上下文)}

当前模型能"看到"的所有 token,包括对话历史、系统提示、/compact 摘要。特点:不可查看、不可版本化、不可靠,仅适合短期推理。Context 是"工作记忆",不是资产。

\paragraph{Memory(记忆)}

你希望模型长期、不漂移记住的东西。特点:必须显式存在(文件/DB)、可编辑、可 diff、可回滚。Memory 是"工程真理的一部分"。

\paragraph{Spec(规范)}

Memory 的"强约束子集",用来定义不变量、边界条件、行为契约。Spec = 高强度 Memory。

\subsubsection{上下文管理的局限}

你只能做三件事:控制对话长度、合理使用 /compact、频繁做理解验证。你无法真正管理 context。

Memory 管理你能做的:定义结构、持久化、明确优先级、强制 agent 读取。这是你能真正掌控的部分。

\subsection{Claude Code 的可行 Memory 管理体系}

\subsubsection{分层 Memory 结构}

最小可行方案:

\begin{verbatim}
/docs
 ├── SPEC.md            ← 不变量 / 合同(极少改,高门槛)
 ├── ARCH.md            ← 架构决策(记录结构选择)
 ├── CURRENT_STATE.md   ← 当前阶段状态(高频改,/compact 后必读)
 └── DECISIONS.md       ← 为什么这么做(防止反复争论)
\end{verbatim}

\subsubsection{Memory 优先级}

每次新对话或 /compact 后,明确告诉 Claude:当对话历史、直觉判断与 docs/ 中的内容冲突时,以 SPEC.md $\rightarrow$ ARCH.md $\rightarrow$ CURRENT\_STATE.md 的顺序为准。这是人为注入 spec-coding 语义。

\subsubsection{Memory 作为显式输入}

不要说"你应该记得我们之前说过......",要说"请先读取 docs/CURRENT\_STATE.md 和 docs/SPEC.md,再继续"。Memory 必须被当成显式输入。

\subsubsection{/compact 的真实角色}

/compact $\neq$ Memory 管理工具。它的作用:清空短期噪音、释放上下文、强迫你把该留下的东西写进 Memory。

好用法:Memory 固化 $\rightarrow$ /compact $\rightarrow$ 继续。坏用法:全靠对话 $\rightarrow$ /compact $\rightarrow$ 迷失。

\subsection{判断标准}

问自己:如果我现在换一个模型,只给它 docs/ 目录,它能继续干活吗?能则 Memory 管理合格,不能则你在靠 context 硬撑。

Droid / Kiro 看起来"更高级"的原因:它们把你现在手动做的 Memory 管理结构化、自动化,赋权给 agent。

\subsubsection{终极总结}

上下文是模型的短期注意力,Memory 是工程的长期真理;真正的 AI 编程能力,不是"管理 context",而是"设计 Memory"。

\section{Droid 的 /compact 问题}

\subsection{Droid 是否需要 /compact}

在真正的 spec-coding agent(如 Droid)里,基本不需要、也不会暴露给用户一个 /compact 这样的命令。原因不是"没做",而是它解决问题的方式已经不同了。

\subsection{Claude Code 需要 /compact 的前提}

在 Claude Code 中:context = 唯一工作记忆,不可见、不可控,长对话必然爆 token/漂移,因此需要 /compact 把历史"压成摘要"给模型续命。/compact 本质是"在 context-only 世界里的垃圾回收器"。

\subsection{Droid 的不同前提}

Droid 的核心设计:Agent = 持续读取/修改/校验外部 spec,而不是只靠 context 推理。Spec 是持久的、可 diff 的、可回滚的。Agent 的推理入口是 spec + 当前任务,而不是长对话历史。Droid 把"长期记忆"彻底外包给 spec 系统了。

\subsection{为什么 Droid 不需要 /compact}

\paragraph{不靠对话历史当 memory}

在 Droid 中,对话是 UI、是交互手段,不是系统真理。没有"长对话 = 负担"这个问题,也就不需要手动压缩。

\paragraph{内部压缩是自动、结构化、不可见的}

Claude Code 的 /compact 压缩自然语言对话;Droid 内部做的是结构化状态裁剪,如只保留 spec diff、丢弃已失效的中间 reasoning。这些是自动发生、不需要用户操心、不会影响 spec 真理的。这是 agent 层的 state management,不是对话层的 compact。

\paragraph{/compact 会破坏 spec-coding 语义}

如果 Droid 允许用户说 /compact,那等于在暗示"对话历史还是重要真理的一部分",这在 spec-coding 世界里是语义错误。Droid 的哲学:你要清理的不是对话,而是 spec 的演化路径,而这必须有 diff、有审计、有回滚。

\subsection{认知跃迁}

当你还在问"要不要 /compact"时,你还在 context-centric 的世界;当系统不再需要 /compact,说明它已经转向 spec-centric。Claude Code 需要你"手动管理记忆",Droid 把记忆管理升级为工程系统。

\subsection{终极总结}

/compact 是 context-only 工具的补丁;真正的 spec-coding agent(如 Droid)通过外置、可维护的 spec,从根本上消灭了 /compact 存在的必要性。

\section{Spec 的连续性与账号切换}

\subsection{Spec 不会"断层"的条件}

Spec 是否会"断层",不取决于你换不换 Droid 账户,而取决于 spec 的"真理载体"在哪里。Spec 在项目仓库/组织级存储则不会断层,Spec 只存在于账号私有空间则会断层。

\subsection{Droid 中的真实情况}

\paragraph{情况 1:Spec 绑定在项目/Repo/Org(推荐 \& 主流)}

这是 Droid 正常、合理、可规模化的使用方式。Spec 存在于 Git 仓库或 Factory.ai 的项目空间(project-level),有 diff、有时间线、有审计记录。换 Droid 账号后,只要有权限、重新 attach 到同一个 project/repo,Agent 会读取完整 spec 历史、继续维护一致性、不会"失忆"。这是 spec-coding agent 成立的前提。

\paragraph{情况 2:Spec 只存在于账号私有 workspace(早期/轻量用法)}

这种情况在个人试用、Demo、PoC、免费/trial 账户中可能存在。Spec 存在于账号私有存储、非 repo 绑定,换账号后读不到历史、spec 状态断层。新账号的 Droid 只能看到"当前 spec 快照"或空白,spec 演化历史丢失。这不是 spec-coding 的失败,而是"没把 spec 当工程资产"。

\subsection{Spec-coding agent 必须"去账号化"}

如果 spec 跟账号走,人走了 spec 也走了、agent 无法继承、团队协作不成立,那它就退化成高级 Copilot。所以 Droid 必须满足:Spec 的身份 = 项目身份,而不是 = 用户身份。否则法律上不可审计、工程上不可继承、商业上不可规模化。

\subsection{连续性对比}

Claude Code:换账号 = 全断。Droid(正确用法):换账号 $\neq$ 换 spec。这正是 spec-coding 的核心价值之一。

\subsection{确认检查点}

1. Spec 是否存在于 Git/Project 级别?2. 换账号后,是否能重新 attach 同一个 project?3. Spec 是否有可查看的历史 diff?如果都有则不会断层,任意一个没有则存在断层风险。

\subsection{终极总结}

在真正的 spec-coding agent 中,spec 的连续性由"项目"保证,而不是由"账号"保证;如果换账号就断层,那不是你用错了,而是系统还没到那个范式。

\section{OpenCode 项目定位}

\subsection{核心定位}

OpenCode 是一个开源、模型无关、隐私优先的 AI 编码 Agent 平台,用来在终端/IDE/桌面中"长期协作式"地写代码,而不是一次性补全。

\subsection{四个核心特征}

\paragraph{1. Open Source AI Coding Agent(不是 Copilot)}

不是代码补全工具或单次问答聊天机器人,而是 Agent(代理),能持续参与项目、支持多 session、并行工作。关键词是:Agent,而不是 Assistant。

\paragraph{2. 模型无关(Model-agnostic)基础设施}

不绑定 Claude/GPT/Gemini,支持 75+ LLM 提供商,支持云模型和本地模型,甚至可以登录你已有的 Claude Pro/Max。OpenCode 把自己定位成"AI coding agent 的运行时与 UI,而不是模型本身"。

\paragraph{3. 多入口的统一 Agent 体验}

Terminal(CLI)、IDE(扩展)、Desktop App(macOS/Windows/Linux),但核心 agent 是同一个概念。说明它的定位不是 UI,而是 Agent 层。

\paragraph{4. 隐私优先(Privacy-first)}

不存储你的代码、不存储上下文、可以在高隐私环境使用。它把自己放在了企业/安全/本地开发友好的位置。

\subsection{OpenCode 不是什么}

\paragraph{不是 Claude Code 的直接替代}

Claude Code 是强模型、强 context、弱可控性、非开源;OpenCode 是强 agent 框架、强可控性、模型可替换、完全开源。一个是"模型能力驱动",一个是"系统能力驱动"。

\paragraph{不是 Cursor/Copilot 那条线}

Cursor/Copilot 的核心是即时补全、人主导 AI 辅助;OpenCode 的核心是任务驱动、AI 主动执行、多 session 并行。OpenCode 更接近"自动化工程助手"。

\subsection{范式定位}

OpenCode 的真实范式是:Context-light + Agent-heavy + External State。它不依赖超长对话,强调 session/agent,更适合外置 spec、repo-based memory、多 agent 协作。从理念上看,它比 Claude Code 更接近 Droid/Kiro 那一侧。

\subsection{终极总结}

OpenCode 的定位不是"最聪明的 AI",而是"最开放、最可控、最工程化的 AI 编码 Agent 平台"。


\section{spec-kit 与 Claude Code 的组合}

\subsection{spec-kit 是什么}

spec-kit 是一个开源工具包,专门为 Claude Code 这样的"对话驱动 AI"设计的 spec-first 外挂框架。它把 spec 从"对话中的隐形共识"变成"repo 里的一等公民"。

\subsection{核心机制}

\paragraph{Spec 外置化}

spec 以 markdown 文件形式存在项目仓库中,不再是藏在对话里的"默契",而是 commit 历史、可 diff、可回滚的工程资产。

\paragraph{强制读取流程}

在你执行任何代码操作前,spec-kit 会自动检查并提示 Claude 先读取 spec,如果你没明确说"跳过 spec",它会把 spec 内容注入到 Claude 的 context 中。

\paragraph{一致性校验}

在你写完代码后,spec-kit 可以运行一套校验,检查实现是否偏离 spec(类型签名、公开 API、模块边界等)。

\subsection{与 Droid 的对比}

\paragraph{能力覆盖度}

维度:spec-kit + Claude Code vs Droid。spec-coding 流程:完整 vs 原生。spec 是否外置:repo vs 系统级。spec 是否可 diff:是 vs 是。spec 是否驱动实现:是 vs 是。多 agent/并行:手动 vs 原生自动。一致性校验:半自动 vs 内建。人工介入成本:高 vs 低。稳定性/抗漂移:中 vs 高。

\paragraph{本质差异}

spec-kit + Claude Code:spec 是 repo 里的文件,但何时更新、是否冲突、是否遗漏主要靠人 + 提示 discipline。Droid:spec 是 agent 的一等公民,agent 会自动检查 spec 覆盖率、做一致性分析、阻止违规实现。系统在维护 spec,而不是你。

\paragraph{Claude Code 仍是"强模型 + 弱系统"}

即使有 spec-kit,Claude Code 依然非常聪明但非常容易"自作主张"。spec-kit 是一套"硬护栏"但不是强制执行引擎。这意味着你得持续"盯着流程",不能完全放手。

\paragraph{Droid 的优势在规模化和自动化}

当 feature 多了、人多了、agent 多了、spec 演化复杂了,Droid 的优势会指数级放大,而 spec-kit + Claude Code 会开始显得"重人工"。

\subsection{适用场景}

\paragraph{非常适合}

单人/小团队、已经在用 Claude Code、想要 spec-driven、可审计、不想换平台、对流程有自律要求的人。

\paragraph{不太适合}

希望"我只写需求,其它全自动"、多 agent 自动并行、企业级 spec 治理。

\subsection{终极总结}

spec-kit 确实可以把 Claude Code 提升到"可用的 spec-coding agent",但 Droid 的优势不在"有没有 spec",而在"spec 是否由系统原生维护"。

} % end small
