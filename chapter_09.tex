% chapter_09.tex
% Spec Coding 与 Vibe Coding 的范式对话
\small

\section*{Claude Code 与 Droid 的机制对比}

\subsection*{Claude Code 的工作机制}

\subsubsection*{CLAUDE.md —— 会话级 system context}

Claude Code 在会话开始时读取 CLAUDE.md 并注入高优先级上下文,但并非每次请求都重置。它是"会话级 system context"而非"请求级 system prompt":同一会话内持续有效,跨会话会重新加载但不保证100\%一致。

\subsubsection*{PRD.md / plan.md —— 软约束的 context memory}

这些文件被扫描作为参考上下文,但优先级低于 CLAUDE.md 和当前 prompt,且受 context 窗口限制。本质上是显式可读但"软约束"的 context。

\subsubsection*{为什么感觉"一直记得"}

Claude Code 通过会话不频繁重置、repo 结构缓存、模型工程先验,让体验看起来像"系统级记忆",但本质仍是 context + prompt + 模型习惯。

\subsubsection*{工程化总结}

Claude Code = 长 context + 高质量模型 + 软文档约束

\subsection*{Droid 的替代方案}

Droid 不要求写 PRD.md/plan.md/CLAUDE.md,但不是不需要这些概念,而是不需要用文件表达。

\subsubsection*{PRD.md → Spec / Feature Model}

需求被结构化拆解存为 spec state,长期存在、可追踪、不随会话消失。PRD 不再是"文档",而是"数据库里的事实"。

\subsubsection*{plan.md → 可执行的 Planning Graph}

Droid 自动生成 plan 作为执行约束,不能随意跳过。plan 不是建议,是执行路径的硬约束。

\subsubsection*{CLAUDE.md → 系统约束 + Spec 权威}

Droid 不靠"在文件里写规则",而是:
\begin{itemize}
    \item spec 在系统中是"状态",不是"文字"
    \item agent 的任何行为必须能追溯到 spec
    \item 行为与 spec 冲突时,系统默认拒绝而非询问
\end{itemize}

\par\noindent\hrule\par

\subsection*{对话:Droid 换账号后 spec 会断层吗}

\subsubsection*{核心问题}

Spec 是否会"断层",不取决于换不换 Droid 账户,而取决于 spec 的"真理载体"在哪里。

\subsubsection*{判断标准}

\textbf{判定问题 1}:Spec 是否能被"另一个账号"在不依赖你本人的情况下完整恢复?验证方式:用另一个账号/同事账号尝试 attach 同一个 repo/project,不复制任何私有 workspace。

\textbf{判定问题 2}:Spec 是否存在一个"可版本化、可 diff 的实体"?判断标准:spec 能否 git diff、rollback、code review,有无 commit hash 和时间线。

\textbf{判定问题 3}:Agent 引用 spec 时,是"引用文件"还是"模糊转述"?好的信号是"According to spec/auth.md\#NonGoals...",坏的信号是"Based on the product requirements..."(找不到来源)。

\subsubsection*{两种使用方式}

\textbf{情况 1:Spec 绑定在项目/Repo/Org}(推荐 \& 主流)

特征:
\begin{itemize}
    \item Spec 存在于 Git 仓库或 Factory.ai 的项目空间
    \item 有 diff、有时间线、有审计记录
    \item 换账号后只要有权限就能重新 attach 到同一个 project/repo
    \item Agent 读取完整 spec 历史,继续维护一致性,不会"失忆"
\end{itemize}

这是 spec-coding agent 成立的前提。如果做不到这一点,Droid 本身就是伪 spec-coding。

\textbf{情况 2:Spec 只存在于账号私有 workspace}(早期/轻量用法)

特征:
\begin{itemize}
    \item Spec 存在于账号私有存储,非 repo 绑定
    \item 换账号后读不到历史,spec 状态断层
    \item 新账号只能看到"当前 spec 快照"或空白,演化历史丢失
\end{itemize}

这不是 spec-coding 的失败,而是"没把 spec 当工程资产"。

\subsubsection*{去账号化的必然性}

如果 spec 跟账号走,人会走、spec 会走、agent 无法继承、团队协作不成立,退化成高级 Copilot。Droid 必须满足:Spec 的身份 = 项目身份,而非用户身份。否则法律上不可审计、工程上不可继承、商业上不可规模化。

\subsubsection*{连续性对比}

Claude Code:换账号 = 全断

Droid(正确用法):换账号 ≠ 换 spec

这正是 spec-coding 的核心价值之一。

\subsubsection*{验证检查点}

\begin{enumerate}
    \item Spec 是否存在于 Git/Project 级别?
    \item 换账号后,是否能重新 attach 同一个 project?
    \item Spec 是否有可查看的历史 diff?
\end{enumerate}

如果三点都有则不会断层,任意一个没有则存在断层风险。

\par\noindent\hrule\par

\subsection*{对话:如何判断 spec 在仓库还是账号}

\subsubsection*{三问判定法}

\textbf{判定问题 1(最硬)}:Spec 是否能被"另一个账号"在不依赖你本人的情况下完整恢复?这是最权威的判定方式。

\textbf{判定问题 2(工程视角)}:Spec 是否存在一个"可版本化、可 diff 的实体"?spec 有无 commit hash 和时间线。

\textbf{判定问题 3(agent 行为)}:Agent 引用 spec 时,是"引用文件"还是"模糊转述"?无法被定位的 spec,一定不是真理源。

\subsubsection*{正确设置方法}

\textbf{情况 A:能控制 Repo}(最理想)

直接把 spec 作为一等工程资产放进仓库:

\begin{verbatim}
repo/
 ├─ spec/
 │   ├─ prd.md
 │   ├─ constraints.md
 │   └─ decisions.md
 ├─ plan/
 │   └─ phase-01.md
 └─ CLAUDE.md
\end{verbatim}

明确两条规则:spec 只能通过 PR 修改;agent 必须引用 spec 文件路径。这时 spec 的真理载体 = Git。

\textbf{情况 B:工具有 Project/Workspace 概念}

找 Project-level binding 而非 User-level。确认项目绑定 repo 而非"我的 workspace",spec 属于 project settings 而非 user profile。正确状态信号:项目有成员管理和权限控制,spec 所有成员可见、非个人私有。警惕 "My Projects"、"My Specs" 这类账号级表达。

\textbf{情况 C:工具完全不透明}

反向工程保障。

\subsubsection*{强行"去账号化"的三种保险方案}

\textbf{方案 1:Spec 镜像}(最低成本)

不信平台,就把 spec 当 source of truth 放在 repo。平台里的 spec 只当 cache/UI,repo 里的 spec 才是法律意义上的真理。换账号、换工具、换厂商都不影响。

\textbf{方案 2:Spec-as-Code}(最稳)

把 spec 写成 Markdown + 机器可校验结构或 YAML/JSON:

\begin{verbatim}
constraints:
  - id: HC-01
    text: No multi-tenancy
  - id: HC-02
    text: Hard delete only
\end{verbatim}

Agent 只能引用 ID,不能自由发挥。

\textbf{方案 3:Spec 引用强约束}(防 drift)

在 CLAUDE.md/agent rules 中写死:

\begin{verbatim}
- Any behavior must cite a spec file and section
- If no spec exists, ask before proceeding
\end{verbatim}

让 agent 天然拒绝"账号记忆式发挥"。

\subsubsection*{判断标准}

"如果我今天离职、账号被删,这个项目是否还能在 spec 不丢失的情况下继续演进?"能 → spec 在仓库/项目;不能 → spec 在账号。

\subsubsection*{终极总结}

判断 spec 是否在仓库而非账号,唯一可靠的方法不是看 UI,而是看:能否被另一个账号、在没有你参与的情况下,通过项目本身完整继承。真正的 spec-coding,spec 的身份证永远属于项目,而不是人。

\par\noindent\hrule\par

\subsection*{对话:如果这么设置,会不会不被当成 spec}

\subsubsection*{核心结论}

有可能。把文件放进 spec/ 目录 ≠ 系统就一定会"把它当成 spec"。spec 是否成立,取决于它是否被"强制作为真理源使用"。结构是必要条件,但不是充分条件。

\subsubsection*{不会被当成 spec 的风险清单}

\textbf{风险 1:agent 只是"看见",但不被强制遵守}

表现形式:agent 会参考 spec、复述 spec,但仍然根据"当前代码"、"常识"、"推测"做决定。本质问题是 spec 是 advisory(参考)而不是 authoritative(裁决)。这是 Claude Code/Copilot 类工具的默认状态。

\textbf{风险 2:spec 没有"引用义务"}

如果 agent 可以说"我认为这样更合理"而不需要说明"依据 spec/constraints.md\#HC-02",那在系统层面 spec 只是背景知识,不是约束。

\textbf{风险 3:spec 与代码冲突时,agent 选了代码}

最常见也最隐蔽的失效方式。当 spec 说 A、代码已经实现了 B,如果 agent 默认"那我沿着现有代码走吧",这一刻 spec 就已经失效了,被降级成"历史文档"。

\textbf{风险 4:spec 太"人类语言",不可定位}

例如"系统应该尽量简单、避免不必要的复杂性"这种 spec:没有 ID、无法被精确引用、无法判断是否违反。agent 只能凭感觉解释,自然不会当成硬 spec。

\subsubsection*{spec 成立的条件}

\textbf{条件 1:明确宣告「spec = 唯一真理源」}

必须在不可忽略的地方(通常是 CLAUDE.md)写清楚:

\begin{verbatim}
## Source of Truth
- All product behavior is defined by files under /spec
- If code conflicts with spec, spec wins
- If spec is silent, ask before proceeding
\end{verbatim}

没有这一步,spec 只是"资料"。

\textbf{条件 2:强制"引用义务"}

\begin{verbatim}
- Any implementation MUST cite a spec file and section
- Changes without spec citation are invalid
\end{verbatim}

这一步把 spec 从"被看"升级为"被调用"。

\textbf{条件 3:spec 可被 diff/review}

spec/目录结构(prd.md、constraints.md、decisions.md)已满足:可版本化、可审计、可 rollback。这是 spec 成立的工程基础。

\textbf{条件 4:冲突时"显式处理",而不是自动融合}

明确规定:

\begin{verbatim}
- If spec conflicts with implementation:
  - Stop
  - Report the conflict
  - Do not auto-resolve
\end{verbatim}

否则 agent 会"聪明地帮你调和",而调和 = drift 的开始。

\subsubsection*{认知转折点}

Spec 不是"放在哪里",而是"有没有否决权"。你已经解决了"spec 在哪里",但还需要解决"spec 能否说不"。

\subsubsection*{补最后一刀方案}

在 CLAUDE.md 里加这段:

\begin{verbatim}
## Spec Authority

- Files under /spec are the authoritative source of truth
- Code, tests, or conventions must not override spec
- Any change must reference at least one spec file and section
- If spec is unclear or missing, ask before implementing
\end{verbatim}

这段的作用是把 /spec 从"普通目录"升级为"裁决层"。

\subsubsection*{终极判断标准}

"如果 agent 生成了一段代码,但我删掉所有 spec 文件,它是否还能自信地继续开发?"不能 → spec 真正生效;能 → spec 只是背景噪音。

\subsubsection*{一句话总结}

把 spec 放进 repo 只是"存储正确",只有当 agent 被强制引用、服从、并在冲突时让 spec 拥有否决权,spec 才真正从"文档"升级为"工程真理"。

\par\noindent\hrule\par

\subsection*{对话:Droid 里的 spec 机制}

\subsubsection*{核心结论}

在 Droid 里,只要正确用法,spec 几乎不可能"只是被看见而不被当成 spec"。原因:在 Droid 中,spec 不是"上下文",而是"系统状态 + 执行约束"。

\subsubsection*{逐条对照风险}

\textbf{风险 1:spec 只是"看见",没有否决权}

Claude Code:spec = context,agent 可以"理解"、"参考"然后自己决定,所以必须靠 CLAUDE.md 去"喊停"。

Droid:spec = Authoritative State,agent 执行逻辑是 "Parse spec → validate plan → execute",不存在"我看了但不听"的分支。如果实现与 spec 冲突,agent 不能继续执行,必须报告冲突、请求 spec 变更。spec 天然拥有否决权。

\textbf{风险 2:没有"引用义务"}

Claude Code:agent 可以不引用、模糊转述、靠"感觉一致"。

Droid:内部是 "Spec node → Plan node → Code change",每一个动作都挂在 spec 节点下。会看到 "This change satisfies Spec S-12" 或在 UI/log 里显示依赖关系。引用不是规范,是结构本身。

\textbf{风险 3:spec 和代码冲突时,代码赢了}

Claude Code:默认行为是"现有代码已经这样了,我就顺着改吧"。

Droid:代码 ≠ 真理、spec ≠ 建议。当发生冲突,agent 不会自动 reconcile,而是进入 blocked/conflict 状态。spec 永远优先于代码。

\textbf{风险 4:spec 太自然语言,被自由解释}

Claude Code:这是 drift 的主要来源之一。

Droid:会把 spec 结构化、标号、转成内部 schema。模糊 spec 会被标记为 ambiguous/incomplete,并要求补充,而不是"帮你猜"。猜 = 被禁止的行为。

\subsubsection*{Droid 里的 spec 形态}

不是"不用 spec",而是"不一定用你手写的 Markdown 作为 spec 本体"。spec 可以来自 UI 表单、结构化输入、repo 中的 spec 文件(同步进系统),但最终都会变成系统内部的 spec state。真正的 spec = 系统状态,而不是 md 文件本身。

\subsubsection*{spec 退化场景}

\textbf{退化场景 1}:用 Droid 当 Copilot(不维护 spec、直接让 agent"改代码"、接受自由发挥)。这不是 Droid 的问题,是使用模式退化。

\textbf{退化场景 2}:spec 没有 project-level 绑定(spec 只存在于个人 workspace、临时 session)。连续性会退化成账号级,失去 Droid 的核心优势。

\subsubsection*{验证标准}

"如果我要求一个明显违反 spec 的改动,agent 会不会直接拒绝?"会拒绝 → spec 是真理;会照做 → 这是 Copilot 模式。

\subsubsection*{对照总结}

Claude Code 需要你"不断提醒 agent 这是 spec";Droid 的设计假设是"agent 默认不可信,因此 spec 必须拥有系统级否决权"。

\subsubsection*{终极总结}

在 Droid 中,spec 是否成立,不取决于放不放在 repo,而取决于是否使用它的"spec-native 执行路径";一旦进入该路径,spec 不可能只是背景文档。
