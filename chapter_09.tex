% chapter_09.tex - Spec Coding 与 Vibe Coding 的范式对话
\small

\chapter{Spec Coding 与 Vibe Coding 的范式对话}

\subsection*{对话一:从 Vibe 到 Spec 的真实阻力}

\subsubsection*{团队协作中的隐性假设}

在多人协作中,Vibe coding 最大的问题是「隐性假设不兼容」。每个开发者对需求的理解都有细微差异,而这些差异在快速迭代中会被放大。

\textbf{场景示例:}
\begin{itemize}
    \item 产品经理说:「用户需要一个简单的搜索功能」
    \item 前端工程师理解为:「一个输入框 + 搜索按钮」
    \item 后端工程师理解为:「支持模糊匹配的接口」
    \item 测试工程师理解为:「验证搜索结果的正确性」
\end{itemize}

这些差异在 Spec coding 中会被明确化为:
\begin{itemize}
    \item 输入:支持哪些搜索关键词格式
    \item 输出:搜索结果的排序规则
    \item 边界:搜索失败时的提示信息
    \item 性能:搜索响应时间要求
\end{itemize}

\subsubsection*{Spec 不是"更多文档",而是"更少歧义"}

很多人反感 Spec coding 是因为他们把 Spec 等同于"写更多文档"。但实际上,Spec coding 是"写更少但更精确的文档"。

\textbf{传统文档 vs Spec 文档:}
\begin{itemize}
    \item \textbf{传统文档:}「用户可以搜索产品」→「搜索结果应该准确」→「支持模糊搜索」
    \item \textbf{Spec 文档:}「用户可以通过产品名称、SKU、分类进行搜索」→「搜索结果应在 500ms 内返回,准确率 > 95%」→「支持拼音首字母、部分匹配、容错输入」
\end{itemize}

\par\noindent\hrule\par

\subsection*{对话二:Spec Coding 的 ROI 误区}

\subsubsection*{短期成本 vs 长期收益}

很多团队拒绝 Spec coding 是因为"看起来慢"。但这种判断忽略了长期维护成本。

\textbf{案例分析:}
\begin{itemize}
    \item \textbf{项目初期}:Vibe coding 比 Spec coding 快 30%
    \item \textbf{项目中期}:Spec coding 开始显现优势,维护成本降低 50%
    \item \textbf{项目后期}:Spec coding 的优势扩大到 200%,重构风险降低 80%
\end{itemize}

\subsubsection*{Spec 的"隐性价值"}

Spec coding 的价值不仅在于"写代码更快",更在于:
\begin{itemize}
    \item 降低沟通成本:团队成员基于同一套规范工作
    \item 减少返工:明确的需求减少了理解偏差
    \item 提高质量:明确的边界条件提高了代码的鲁棒性
    \item 增强可维护性:新成员可以快速理解代码意图
\end{itemize}

\par\noindent\hrule\par

\subsection*{对话三:Spec Coding 的正确姿势}

\subsubsection*{不是"先写 Spec,再写代码"}

Spec coding 不是僵化的"先写完整 Spec,再写代码"流程。正确的做法是:

\textbf{迭代式 Spec 开发:}
\begin{enumerate}
    \item 写核心功能的最小 Spec
    \item 实现核心功能
    \item 测试验证
    \item 完善 Spec,添加边界条件
    \item 重构实现
    \item 重复上述过程
\end{enumerate}

\subsubsection*{Spec 的"可承受范围"}

Spec 应该写得"刚好足够",而不是"面面俱到"。判断标准:
\begin{itemize}
    \item 核心功能必须有 Spec
    \item 复杂逻辑必须有 Spec
    \item 边界条件必须有 Spec
    \item 简单功能可以不写 Spec
\end{itemize}

\par\noindent\hrule\par

\subsection*{对话四:AI 时代的 Spec Coding}

\subsubsection*{AI 改变了什么}

AI 工具(如 Claude Code、Droid)让 Spec coding 变得更容易,但并没有改变其本质。

\textbf{AI 对 Spec Coding 的影响:}
\begin{itemize}
    \item 降低了写 Spec 的成本:AI 可以帮助生成和完善 Spec
    \item 提高了 Spec 的质量:AI 可以检查 Spec 的完整性和一致性
    \item 增强了 Spec 的可读性:AI 可以将复杂 Spec 可视化
    \item 自动化了验证过程:AI 可以自动验证代码是否符合 Spec
\end{itemize}

\subsubsection*{AI 时代的新挑战}

AI 工具也带来了新的挑战:
\begin{itemize}
    \item Spec 的"漂移风险":AI 可能会误解 Spec 并生成不符合要求的代码
    \item Spec 的"维护负担":随着需求变化,Spec 需要频繁更新
    \item Spec 的"质量要求":AI 对 Spec 的质量要求更高,模糊的 Spec 会导致错误的代码
\end{itemize}

\par\noindent\hrule\par

\subsection*{终极总结}

Spec Coding 不是"写文档",而是"建立工程真理"。它的价值在于:
\begin{itemize}
    \item 减少歧义:明确的 Spec 消除了团队成员之间的理解差异
    \item 提高效率:长期维护阶段的效率提升远远超过启动阶段的成本
    \item 降低风险:明确的约束减少了 bug 和需求漂移的风险
    \item 增强协作:团队成员基于同一套 Spec 工作,减少了沟通成本
\end{itemize}

在 AI 时代,Spec Coding 变得更加重要,因为它为智能体提供了明确的工作边界和验证标准。真正的 Spec-coding agent(如 Droid)不是"让 AI 写代码",而是"让 AI 维护 Spec $\leftrightarrow$ Code 一致性"。

\vspace{1em}
\normalsize
