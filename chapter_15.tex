\small

\section*{Chapter 15: Spec-Driven 工作流与 Git 实践}

\subsection*{15.1 Pull Request 的本质与定位}

\subsubsection*{PR 的定义}

Pull Request = 代码合并请求。在 Git 工作流中,PR 是将 feature 分支的变更合并到目标分支(通常是 main)的正式机制。PR 不仅包含代码差异(diff),还关联了上下文信息:spec、plan、tasks、commit 历史。

\subsubsection*{PR 在 Spec-Driven 中的角色}

在 spec-driven 开发中,PR 承载三重功能:

\textbf{(1) 代码审查载体}:Review 者通过 PR 查看 spec 定义与实际实现的对应关系,验证是否按 spec 执行。

\textbf{(2) 决策边界}:PR 是"是否接受此 feature 进入主线"的人类决策点。AI 可以执行 implement,但无法决定"这段代码是否值得进 main"。

\textbf{(3) 历史记录单元}:PR 合并后,其 title、description、讨论内容成为永久可追溯的工程文档。

\subsubsection*{PR 与 Spec 的关系}

Spec 文件(.specify/specs/XXX/spec.md)与代码在同一 feature 分支中创建和提交。PR 打开时,spec 已作为 commit 的一部分存在。Review 者在 PR 中同时看到:spec 内容(为什么这么做)+ 代码变更(怎么做的)+ 测试结果(是否有效)。

\subsection*{15.2 Spec-Driven 工作流的四个阶段}

\subsubsection*{阶段 1:Spec / Plan / Tasks(决策阶段)}

位置:feat/* 分支

操作:定义需求、约束实现路径、拆解任务

状态:可反复修改、不产生代码、仅存在规划文档

\subsubsection*{阶段 2:/speckit.implement(执行阶段)}

位置:同一 feat/* 分支

AI 被允许:按 tasks 写代码、创建/修改文件

AI \textbf{不被允许}:commit、merge、push

结果:代码仅存在于 working tree,未进入 Git 历史

\subsubsection*{阶段 3:人类 Review(控制阶段)}

操作:执行 \texttt{git diff} 查看 spec、plan、tasks 与 code diff

决策:接受 / 拒绝 / 修改 spec 后重新 implement

此阶段是 spec-driven 的核心安全闸门。

\subsubsection*{阶段 4:Merge(生效阶段)}

操作:手动 \texttt{git merge} 或点击 PR merge 按钮

这是唯一影响 main 分支的时刻,必须由人类触发。

\subsection*{15.3 关键设计原则:执行与生效的分离}

\subsubsection*{为什么 implement 不自动 commit?}

Commit 是"责任边界",不能由 AI 自动跨越。原因:

\textbf{(1)} Commit 代表对历史节点的认可:哪些文件算完成、commit message 怎么写、是一个 commit 还是多个——这些都是工程判断,非执行判断。

\textbf{(2)} Commit 是 review 之后的动作,非 implement 的一部分。正确顺序:implement $\rightarrow$ review $\rightarrow$ commit。若 implement 自动 commit,review 变成事后补救,历史会被污染。

\subsubsection*{为什么 spec 同意不自动 merge?}

AI 可以执行,但不能决定"什么时候对世界负责"。在工程中:commit = 对历史负责、merge = 对主线负责、release = 对用户负责。这些责任点不能自动化。

用户同意 spec 的真实含义是:"我授权 AI 按此计划尝试实现,但结果是否可接受,仍由我决定。"而非:"我同意这段代码直接进 main。"

\subsection*{15.4 Commit 策略:从防丢存档到决策节点}

\subsubsection*{Spec-Driven 下 Commit 的角色变化}

传统 vibe coding:高频、碎片化 commit,作用是"防丢代码"。

Spec-driven:低频、语义化 commit,作用是"标记决策节点"。

\subsubsection*{Feat 分支仍需人工 Commit}

原因:(1) commit 是 review 的最小单位;(2) commit 控制"回滚粒度";(3) merge 的前提是 commit(merge 的对象是 commit,非工作区)。

\subsubsection*{推荐 Commit 原则}

\textbf{1} 决策变成现实 $\rightarrow$ commit

\textbf{2} 一个 commit = 一个"可解释状态"

\textbf{3} commit 不是越多越好,而是越"语义完整"越好

\textbf{4} 判断口诀:"如果现在让我向别人解释当前状态,我是否有一句话能概括?"有则 commit,无则等待。

\subsubsection*{健康的 Commit 节奏示例}

\begin{verbatim}
feat/todo-cli 分支:

commit 1: docs(todo): finalize spec and plan
commit 2: feat(todo): implement todo CLI
\end{verbatim}

spec/plan 单独 commit,implement 是一个或少量 commit,中间不频繁碎 commit。

\subsection*{15.5 .specify/ 目录的 Git 生命周期}

\subsubsection*{核心原则}

.specify/ 与代码在 Git 里是同一等级。Git 眼里只有"哪些文件在 commit 里",对 .specify/specs/001-xxx/spec.md 与 src/foo.ts 一视同仁。

\subsubsection*{Spec 进入 Main 的时机}

.specify/ 并非"自动等到 merge 才进入 main",而是:始终存在于 feat 分支;是否进入 main,完全取决于 merge 时是否保留它。没有任何"延迟进入 main"的特殊机制。

\subsubsection*{推荐模式:Spec 跟 Feature 一起 Merge}

\begin{verbatim}
feat/xxx branch:
  - commit A: docs: add spec/plan/tasks
  - commit B: feat: implement xxx

merge → main

结果:main 包含代码 + .specify/specs/001-xxx/
\end{verbatim}

这是 spec-driven 的"完整形态":feature 的"做什么"和"为什么"都被永久记录在 main 历史中。

\subsubsection*{其他模式}

\textbf{Squash merge}:.specify/ 是否进入 main 取决于 squash 的 commit 里是否包含它。若 squash 时只选代码或手动删 .specify/,则不会进 main——这是人的选择,非工具规则。

\textbf{删除 spec(不推荐)}:有些团队在 feature 完成后删除 .specify/ 再 merge。技术上合法,但工程上是损失:删除了决策依据、设计上下文、未来重构的参考。

\subsubsection*{分支删除与文件保留}

\begin{verbatim}
feat branch: .specify/ + src/

↓ merge

main: .specify/ + src/

↓ delete feat branch

main: .specify/ + src/  (仍存在)
\end{verbatim}

删除分支 $\neq$ 删除文件。只要文件被 commit 且 commit 被 merge,文件就进入了 main。

\subsection*{15.6 Squash:压扁 Commit 的艺术}

\subsubsection*{定义}

Squash = 把多个 commit"压扁"为一个 commit。

\begin{verbatim}
squash 前:
  commit A: add spec
  commit B: fix typo
  commit C: implement feature
  commit D: fix bug

squash 后:
  commit X: implement feature with spec
\end{verbatim}

\subsubsection*{Squash 的目的}

不 squash 的问题:main 上充满"fix typo"、"wip"等噪音,历史难读。

Squash 的目的:main 只保留"做了什么",而非"我中间怎么折腾的"。这是对"主线历史可读性"的优化手段。

\subsubsection*{Squash 与 Spec-Driven 的关系}

在 spec-driven 中,squash 非常常见甚至推荐。因为 spec/plan 已记录过程,main 更适合只保留:spec 定义 + 最终实现。

健康的结果:
\begin{verbatim}
commit X:
  docs(todo): spec + plan + tasks
  feat(todo): implement todo CLI
\end{verbatim}

或干脆:
\begin{verbatim}
commit X: feat(todo): implement todo CLI (per spec)
\end{verbatim}

\subsubsection*{Squash $\neq$ 删除历史}

原 commit 仍在 feat 分支和 reflog 中,只是不进入 main。Squash 是"选择性公开历史",非抹除历史。

\subsubsection*{何时不用 Squash}

不适合 squash 的情况:release/hotfix 分支、需要精确 bisect 的历史、每个 commit 都是稳定点。

判断口诀:"main 要不要看懂你怎么一步步写的?"不需要则 squash,需要则不 squash。

\subsubsection*{操作方式}

本地:\texttt{git rebase -i main},把多个 commit 标记为 squash。

团队:99\% 情况直接用 PR 的"Squash and merge"按钮。

\subsubsection*{本质}

Squash 的本质:把"开发过程"留在 feature 分支,把"工程结果"放进 main。

\subsection*{15.7 推荐工作流总结}

\subsubsection*{绝对安全的工作流}

\begin{verbatim}
feat branch:
  ✅ spec / plan / tasks(commit)
  ✅ /speckit.implement
  ✅ review(git diff / PR)
  ✅ commit implementation
  ✅ open PR
  ✅ merge

main:
  ✅ 只接受 review 通过的结果
\end{verbatim}

在此流程中:没有任何"自动 merge"步骤;每个不可逆动作前都有人工确认。

\subsubsection*{核心心智模型}

/speckit.implement = "一个严格按 spec 干活、但没有提交权限的工程师。"

你是:唯一能 commit 的人、唯一能 merge 的人。

\subsubsection*{定锚原则}

spec-driven 的自动化只覆盖"执行","是否生效"永远是人类的权力。Commit 和 merge 永远是显式的人类动作。

.specify/ 始终存在于 feature 分支;是否进入 main,完全由 merge 决定;merge 的不是"分支",而是"提交过的设计与实现"。
